\documentclass[a4paper]{scrartcl}
\usepackage[ngerman]{babel}
\usepackage[utf8]{inputenc}
\usepackage{amssymb,amsmath}
\usepackage{graphicx}
\usepackage[inline]{enumitem}
\setlist{noitemsep}
\usepackage[binary-units=true]{siunitx}
\usepackage{hyperref}
\usepackage{parskip}
\usepackage[nameinlink,noabbrev,ngerman]{cleveref} % has to be after hyperref
\usepackage[colorinlistoftodos]{todonotes}

\usepackage{listings}% http://ctan.org/pkg/listings
\lstset{
  basicstyle=\ttfamily,
  mathescape
}

\setcounter{secnumdepth}{2}
\setcounter{tocdepth}{2}

\begin{document}
 \selectlanguage{ngerman}
 \title{2013 SS}
 \author{Martin Thoma}

 \setcounter{section}{1}
 %%%%%%%%%%%%%%%%%%%%%%%%%%%%%%%%%%%%%%%%%%%%%%%%%%%%%%%%%%%%%%%%%%%%%%%%%%%%%%
 \section*{Aufgabe 1}
 \subsection*{Teilaufgabe 1.1}
 \subsubsection*{1.1 a}
  \begin{enumerate*}[label=(\arabic*)]
      \item Schnelligkeit
      \item Genauigkeit
      \item Stabilität
  \end{enumerate*}

  \subsubsection*{1.1 b}
  \begin{align}
      X_1 &= G_1 \cdot G_2\\
      X_2 &= G_3\\
      X_3 &= G_4 + G_5
  \end{align}

  \subsubsection*{1.1 c}
  \begin{align}
      L(2 \ddot{x} + 3 \dot{x} - 4x) &= L(\dot{w} + w)\\
      \Leftrightarrow 2 \cdot L(\ddot{x}) + 3 L(\dot{x}) - 4 L(x) &= L(\dot{w}) + L(w)\\
      \overset{?}{\Leftrightarrow} 2 (s^2 X(s) - X(0) - x(0)) + 3(X(s) - x(0)) - 4 X(s) &= s W(s) - w(0) + W(s)\\
      \Leftrightarrow X(s) (2s^2 - 1 - 2 X(0) - 2x(0)) + 3(X(s) - x(0) - 4 X(s)) &= W(s)(s+1) - w(0)
  \end{align}

  \underline{Annahme:} $x(0) = 0, X(0) = 0, w(0) = 0$ \todo{Ist das notwendig / gerechtfertigt?}

  \begin{align}
      X(s) (2s^2 - 1) &= W(s) (s+1)\\
      \Leftrightarrow \frac{X(s)}{W(s)} &= \frac{s+1}{2s^2 - 1} = G(s)
  \end{align}

  \subsubsection*{1.1 d}
  Hurwitz-Kriterium

  \subsubsection*{1.1 e}
  \begin{itemize}
      \item Das D-Glied \textbf{differenziert} die \textbf{Regeldifferenz}.
      \item Durch Betrachtung der Änderung des Signals wird ein zukünftiger
            Trend berücksichtigt. Die D-Reglerstrategie ist: Je stärker
            die Änderung der Regelabweichung ist, desto stärker muss das
            Stellsignal verändert werden.
      \item D-Glieder \textbf{verbessern} gewöhnlich die
            \textbf{Regelgeschwindigkeit} und die dynamische Regelabweichung.
      \item D-Gleider verstärken besonders hochfrequente (verrauschte)
            Anteile des Eingangssignals. Dies erhöht die Neigung zu
            Schwingungen.
  \end{itemize}


  \subsection*{Teilaufgabe 1.2}
  \subsubsection*{1.2 a}
  \begin{itemize}
      \item Einmaliger Hardware-Entwicklungsaufwand, dann sind parametrisierte
            Systemänderungen per Software möglich.
      \item Realisierung komplexer Reglerstrukturen
  \end{itemize}

  \subsubsection*{1.2 b}
  Bild mit Sollwert, Istwert, Regler $R(s)$, Taster $T$, Stracke $G(s)$
  (TODO)

  \subsubsection*{1.2 c}
  (Bild, TODO)

  \begin{itemize}
      \item 0T - 1T: 1
      \item 1T - 2T: 7
      \item 2T - 3T: 1
      \item 3T - 4T: 4
      \item 4T - 5T: 2
  \end{itemize}


  \subsubsection*{1.2 d}
  \textbf{Differenzialgleichung (DGL):} $2 \ddot{x}(t) + 3 \dot{x}(t) - x(t) = w(t)$\\
  \textbf{Differenzengleichung}:
  \begin{align}
      2 \frac{x(k) - 2x(k-1) + x(k-2)}{T_A^2} + 3 \frac{x(k) - x(k-1)}{T_A} - x(k) &= w(k)\\
      x(k) (\frac{2}{T_A^2 + \frac{3}{T_A} - 1}) + x(k-1)(-\frac{2}{T_A^2} - \frac{3}{T_A}) + \frac{2}{T_A^2} x(k-2) &= w(k)
  \end{align}

  \textbf{Z-Transformierte}
  \[X(z) (\frac{2}{T_A^2} + \frac{3}{T_A} - 1 + z^{-1} (- \frac{2}{T_A^2} - \frac{3}{T_A} + z^{-2} \frac{2}{T_A^2})) = W(z)\]

  \subsubsection*{1.2 e}
  Z-Transformierte der DGL:
  \begin{align}
      (z^{-1} + 1) X(z) &= W(z) + Z(-w_{k+1})\\
      \overset{Linksverschiebung}{\Leftrightarrow} (z^{-1} + 1) X(z) &= W(z) - z(W(z) - w_0 z^0)\\
      &= W(z)- z W(z)\\
      &= W(z) (1-z)\\
      \Leftrightarrow \frac{X(z)}{W(z)} &= \frac{1-z}{1+z^{-1}}
  \end{align}

  \subsubsection*{1.2 f}
  Für die Polstellen $p_1 = 0.7$ und $p_2 = -0.25$ gilt: $|p_1| < 1$ und $|p_2| < 1$
  $\Rightarrow$ Es ist stabil.

  Eine mögliche zugehörige Übertragungsfunktion ist:
  \[G(z) = \frac{(z+0.5)(z+1.5)}{(z-0.7) (z+0.25)}\]
  Diese kann mit beliebigen, von 0 verschiedenen, Konstanten multipliziert
  werden und würde immer noch im selben Pol-Nullstellendiagramm resultieren.

%%%%%%%%%%%%%%%%%%%%%%%%%%%%%%%%%%%%%%%%%%%%%%%%%%%%%%%%%%%%%%%%%%%%%%%%%%%%%%%
  \section*{Aufgabe 2}
  \subsection*{Teilaufgabe 2.1}
  \subsubsection*{2.1 a}
  %\todo{Wurde das in der Vorlesung gesagt? Steht das in den Folien?}
  \begin{itemize}
      \item Datenspeicher
      \item Programm- und Konstantenspeicher
      \item nonvolatile RAM            
      \item Wie kann nichtflüchtiger Speicher weiter unterschieden werden?
            \textbf{Einmal beschreibbar / Wiederbeschreibbar}
  \end{itemize}

  \subsubsection*{2.1 b}
  Folgende Techniken beschleunigen die Programmausführung in Mikroprozessoren:
  \begin{enumerate*}[label=(\arabic*)]
      \item Pipelining erhöht Befehlsdurchsatz
      \item Spekulation
      \item Speicherhierarchien
  \end{enumerate*}
  
  Kennzahlen und Argumentation (je weiter best und worst case auseinanderliegen, desto weniger echtzeitfähig):
  \begin{itemize}
  	\item ohne Pipeline: best = worst = 15 Taktzyklen - echtzeitfähig
	\item einfache Pipeline: best = worst = 12 Taktzyklen - echtzeitfähig
	\item Pipeline + Spekulation: best = 9, worst = 15 - weniger echtzeitfähig
	\item Pipeline + Cache + Spekulation: best = 5, worst = 15 - noch weniger echtzeitfähig
  \end{itemize}

  \subsubsection*{2.1 c}
  Zur Wahrung der Echzeitfähgikeit von System mit mehr als einem Busmaster
  müssen folgende Eigenschaften erfüllt sein:
  \begin{enumerate}
      \item Prioritäten: Jedem Busmaster muss eindeutig eine Priorität zugeordnet werden.
      \item Preemtion: Verlangt ein Busmaster mit höherer Priorität den Bus, so wird der mit niederer Priorität unterbrochen.
      \item Unterbrechbarkeit von Blocktransfers: Zur Vermeidung von Prioritäteninversion müssen Blocktransfers, d.h. Datentransfers in langen Blücken unterbrochen werden können.
      \item Busmonitor: Dient der Überwachung der Buszuteilungsregeln. Überschreiten ein Master das ihm zugeteilte Zeitkontingent, so muss der Busmonitor dies feststellen und Gegenmaßnahmen (z.B. Bus Error) einleiten.
  \end{enumerate}

  \subsubsection*{2.1 d}
  Der PCI-Bus ist Echtzeitfähig,
  Synchron (Takt: \SIrange[range-phrase = --]{33}{66}{\mega\hertz}),
  hat einen gemultiplexten Adress- und Datenbus und erlaubt Burst-Transfers.

  \subsubsection*{2.1 e}
  Da die PCI-Bridge wie ein normales PCI-Device behandelt wird, können
  PCI-Bussysteme hierarchisch kaskadiert werden.

  \subsection*{Teilaufgabe 2.2}
  \subsubsection*{Teilaufgabe 2.2 a}
  \begin{figure}[h]
      \centering
      \includegraphics*[width=0.5\linewidth, keepaspectratio]{op-verstaerker.png}
      \caption{Quelle: \href{https://commons.wikimedia.org/wiki/File:Op-amp_symbol.svg}{commons.wikimedia.org/wiki/File:Op-amp\_symbol.svg}}
      \label{fig:operationsverstaerker}
  \end{figure}

  In \cref{fig:operationsverstaerker} ist ein Operationsverstäker mit dem
  invertierenden Eingang $V_{-}$, dem nicht-invertierendem Eingang $V_{+}$,
  sowie der positiven und negativen Versorgungsspannung ($V_{S+}$ und $V_{S-}$)
  und der Ausgangsspannung $V_{\text{out}}$ dargestellt.\\
  \begin{enumerate*}[label=(\roman*)]
      \item Eingangswiderstand $\rightarrow \infty$
      \item Ausgangswiderstand $\rightarrow 0$
      \item Verstärkung $\rightarrow \infty$
  \end{enumerate*}

  \subsubsection*{Teilaufgabe 2.2 b}
  TODO: Bild

  \subsubsection*{Teilaufgabe 2.2 c}
  TODO

  \subsubsection*{Teilaufgabe 2.2 d}
  TODO

%%%%%%%%%%%%%%%%%%%%%%%%%%%%%%%%%%%%%%%%%%%%%%%%%%%%%%%%%%%%%%%%%%%%%%%%%%%%%%%
  \section*{Aufgabe 3}
  \subsection*{Teilaufgabe 3.1}
  \subsubsection*{Teilaufgabe 3.1 a}
  \begin{enumerate*}[label=(\arabic*)]
      \item TODO
      \item TODO
      \item TODO
  \end{enumerate*}

  \subsubsection*{Teilaufgabe 3.1 b}
  Maximale MMAT bei Token Passing: TODO

  \subsubsection*{Teilaufgabe 3.1 c}
  Dekodiertes Signal: 0011 1010 10

  \subsubsection*{Teilaufgabe 3.1 d}
  \begin{enumerate*}[label=(l\arabic*)]
      \item 2
      \item 3
      \item 4
      \item 2
      \item 3
      \item sendet
      \item 5
  \end{enumerate*}

  \subsubsection*{Teilaufgabe 3.1 e}
  Maximaler Füllstand des Paketpuffers bei TCP/IP-Kommunikation: TODO

  \subsection*{Teilaufgabe 3.2}
  \subsubsection*{Teilaufgabe 3.2 a}
  Asynchrone und Synchrone Programmierung.
  Synchrone Programmierung wird für rein zeitbasierte Systeme eingesetzt.

  \begin{itemize}
      \item Vorteil Synchroner Programmierung:
      \begin{itemize}
          \item Festes, vorhersagbares Zeitverhalten
          \item Einfache Tests und Analyse des Systems
          \item Rechtzeitigkeit und Gleichzeitigkeit können leicht garantiert
                werden.
      \end{itemize}
      \item Nachteil Synchroner Programmierung: Nicht ereignisbasiert\todo{also: nicht asynchron?}
  \end{itemize}


  \subsubsection*{Teilaufgabe 3.2 b}
  Ein Schedulingverfahren heißt \textit{optimal}, wenn es einen gültigen
  Schedule findet, falls dieser exisitert.

  \begin{itemize}
      \item Optimal auf Einprozessorsystemen\todo{also nicht Mehrprozessorsysteme?}: GPS (Guaranteed Percentage Scheduling),
            LLF (Least Laxity First),
            EDF (Earliest Deadline First)
      \item Nicht optimal: FPP (Fixed Priority Preemptive) bei $H_n > n \cdot (2^{\frac{1}{n}} - 1)$,
            FPN (Fixed Priority Non-Preemtpive)
  \end{itemize}


  \subsubsection*{Teilaufgabe 3.2 c}
  Nicht-Echtzeitsysteme: Logische Korrektheit

  Echtzeitsysteme: Zusätzlich Zeitliche Korrektheit


  \subsubsection*{Teilaufgabe 3.2 d}
  TODO

  \subsubsection*{Teilaufgabe 3.2 e}
  \begin{enumerate*}[label=(\arabic*)]
      \item TODO
      \item TODO
  \end{enumerate*}

  \subsubsection*{Teilaufgabe 3.2 f}
  \begin{enumerate*}[label=(\arabic*)]
      \item Gleichzeitigkeit
      \item Verfügbarkeit
  \end{enumerate*}

%%%%%%%%%%%%%%%%%%%%%%%%%%%%%%%%%%%%%%%%%%%%%%%%%%%%%%%%%%%%%%%%%%%%%%%%%%%%%%%
  \section*{Aufgabe 4}
  \subsection*{Teilaufgabe 4.1}
  \subsubsection*{Teilaufgabe 4.1 a}
  \begin{enumerate*}[label=(\arabic*)]
      \item Elementare Taskverwaltung
      \item Interprozesskommunikation
      \item Synchronisation
  \end{enumerate*}

  \subsubsection*{Teilaufgabe 4.1 b}
  RT-Linux \todo{Ist das ein Betriebssystemaufsatz? Was ist ein Betriebssystemaufsatz?}

  \subsubsection*{Teilaufgabe 4.1 c}
  TODO: Ablauf Sperrsynchronisation / Reihenfolgensynchronisation

  \subsubsection*{Teilaufgabe 4.1 d}
  Middleware ist Software oberhalb des Betriebssystems. Sie sorgt in
  heterogenen und/oder verteilten Systemen für Transparenz.

  \subsubsection*{Teilaufgabe 4.1 e}
  \begin{itemize}
      \item $\SI{32}{\bit}\text{Prozessor} \Rightarrow$ Logische
            Adresse ist \SI{32}{\bit} lang.
      \item $\SI{1}{\giga\byte}=2^{30}\text{ Byte} \Rightarrow$ Physikalische
            Adresse ist \SI{30}{\bit} lang.
      \item $\SI{4}{\kilo\byte}=2^{12}\text{ Byte} \Rightarrow$ Beide Offsets
            sind \SI{12}{\bit}
      \item[$\Rightarrow$] \SI{20}{\bit} Seitenadresse, \SI{18}{\bit} Basisadresse
  \end{itemize}

  \subsection*{Teilaufgabe 4.2}
  \subsubsection*{4.2 a}
  Anzahl Gatter: $2$
  \begin{lstlisting}
            +---+
E0.2 +-----O+   |
            | & +--+
E0.1 +------+   |  |  +----+
            +---+  +--+    |
                      | $\geq$1 +------+ A1.0
E0.3 +---------------O+    |
                      +----+
  \end{lstlisting}

  \subsubsection*{4.2 b}
  \begin{lstlisting}
     E0.2      E0.1        A1.0
     |  /|     |   |       |- -|
+----| / |-----|   |---+---|   |----+
     |/  |     |   |   |   |- -|
                       |
           E0.3        |
           |  /|       |
+----------| / |-------+
           |/  |
  \end{lstlisting}
  \subsubsection*{4.2 c}
  A1.0 $:=$ ((E0.1 $\&\texttt{NOT}$ E0.2)  \texttt{OR NOT} E0.3)

  \subsubsection*{4.2 d}
  \begin{itemize}
	\item Einlesen der Sensorsignale in Eingangsspeicher
	\item Aufrufen des Automatisierungsprogramms; Abarbeiten der Anweisungen
	\item Eingangssignale mit internen Zuständen und Ausgängen zu Ausgangssignalen verknüpfen
  \end{itemize}
\end{document}
