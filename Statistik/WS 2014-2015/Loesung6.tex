\section*{Lösung}
\subsection*{Teilaufgabe a}
Die Kovarianz zweier Zufallsvariablen $X$ und $Y$ ist definiert als
\[Cov(X, Y) := E((X - EX) (Y - EY))\]

Kovarianz-Matrizen sind positiv semidefinit, d.h.

\[|\Sigma| = 1 - c^2 \geq 0\]

$|c| \leq 1$. Also $c \in [-1, 1]$.

Der Korrelationskoeffizient ist definiert als
\[\rho(X, Y) := \frac{Cov(X, Y)}{V(X) V(Y)}\]
Es gilt $\rho(X, Y) \in [-1, 1]$ aufgrund der Cauchy-Schwarz-Ungleichung:
\[|Cov(X, Y)| \leq \sqrt{V(X) V(Y)}\]

\subsection*{Teilaufgabe b}
Sei $A = \begin{pmatrix}1 & -1\end{pmatrix}$. Dann gilt $V = A X$ und daher
$$A \Sigma A^T = 5 - c$$ und damit
$$V \sim \mathcal{N}(0, 5 - c)$$

\subsection*{Teilaufgabe c}
Es gilt:
\[U = \underbrace{\begin{pmatrix}1&k\end{pmatrix}}_{=: B} X \sim \mathcal{N}(k+1, B\Sigma B^T) \stackrel{D}{=} \mathcal{N}(k+1, c + 2k + k^2 c)\]
und nach (b)
\[V = AX \sim \mathcal{N}(0, 5-c)\]

Sowie
\[\begin{pmatrix}U\\V\end{pmatrix} = \begin{pmatrix}B\\A\end{pmatrix}X = \underbrace{\begin{pmatrix}1&k\\1&-1\end{pmatrix}}_{=: C}X \sim N_2(\begin{pmatrix}k+1\\0\end{pmatrix}, C^T \Sigma C)\]

Es gilt nun
\begin{align}
C^T \Sigma C &= \begin{pmatrix}1&k\\1&-1\end{pmatrix} \cdot \begin{pmatrix}1&c\\c&1\end{pmatrix}\begin{pmatrix}1&1\\k&-1\end{pmatrix}\\
&= \begin{pmatrix}1+kc&c+k\\1-c&c-1\end{pmatrix} \cdot \begin{pmatrix}1&1\\k&-1\end{pmatrix}\\
&= \begin{pmatrix}1+kc+kc+k^2&1+kc-c-k\\1-c+kc-k&1-c-c+1\end{pmatrix}\\
&= \begin{pmatrix}1+2kc+k^2&1+kc-c-k\\1-c+kc-k&2(1-c)\end{pmatrix}
\end{align}

Wähle nun $k$ so, dass die Covarianz gleich 0 ist:

\begin{align}
 0&\stackrel{!}{=} 1+kc-c-k\\
 \Leftrightarrow k - kc &\stackrel{!}{=} 1-c
\end{align}

Fall 1: $c = 1$. Dann spielt $k$ keine Rolle.
Fall 2: $c \neq 1$. Dann gilt
\begin{align}
 k &\stackrel{!}{=} \frac{1-c}{1-c} = 1
\end{align}