\section*{Aufgabe 3}
\subsection*{Teilaufgabe i}
relativer Fehler:

\begin{align}
	\frac{ | \frac{x}{y} - \frac{x \cdot (1 + \varepsilon_x)}{y \cdot (1 + \varepsilon_y)}|}{|\frac{x}{y}|}
	&= \frac{| \frac{x(1+\varepsilon_y) - x (1+\varepsilon_x)}{y(1+\varepsilon_y)} |}{|\frac{x}{y}|} \\
	&= \frac{| \frac{x(\varepsilon_y-\varepsilon_x)}{y(1+\varepsilon_y)} |}{|\frac{x}{y}|} \\
    &= \left |\frac{\varepsilon_y - \epsilon_x }{1 + \varepsilon_y} \right |\\
	&\le \frac{|\varepsilon_y | + | \varepsilon_x |}{|1 + \varepsilon_y|} \le \frac{2 \cdot \text{eps}}{|1 + \varepsilon_y|} \\
    &\approx 2 \cdot \text{eps}
\end{align}

Der letzte Ausdruck ist ungefähr gleich $2 \cdot \text{eps}$, da $1 + \epsilon_y$ ungefähr gleich $1$ ist.

Der relative Fehler kann sich also maximal verdoppeln.

\subsection*{Teilaufgabe ii}
Die zweite Formel ist vorzuziehen, also $f(x) = -\ln (x + \sqrt{x^2-1})$, da es bei Subtraktion zweier annähernd gleich-großer Zahlen zur Stellenauslöschung kommt. Bei der ersten Formel, also $f(x) = \ln (x - \sqrt{x^2-1})$, tritt genau dieses Problem auf: $x$ und $\sqrt{x^2-1}$ sind für große $x$ ungefähr gleich groß. \\
Bei der zweiten Formel tritt das Problem nicht auf: $x$ ist positiv und $\sqrt{x^2 - 1}$ auch, also gibt es in dem Ausdruck keine Subtraktion zweier annähernd gleich-großer Zahlen.

Außerdem ändert sich $\ln(x)$ stärker, je näher $x$ bei 0 ist. Es ist
also auch wegen der Ungenauigkeit der Berechnung des $\ln$ besser,
weiter von $0$ entfernt zu sein.
