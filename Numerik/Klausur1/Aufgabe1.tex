\section*{Aufgabe 1}
\subsection*{Teilaufgabe a}
\textbf{Gegeben:}

\[A =
\begin{pmatrix}
    3 & 15 & 13 \\
    6 & 6  & 6  \\
    2 & 8  & 19
\end{pmatrix}\]

\textbf{Aufgabe:} LR-Zerlegung von $A$ mit Spaltenpivotwahl

\textbf{Lösung:}

\begin{align*}
	&
	&
    A^{(0)} &= \begin{gmatrix}[p]
		3 & 15 & 13 \\
		6 & 6  & 6  \\
		2 & 8  & 19
	 \rowops
	 \swap{0}{1}
	\end{gmatrix}
	&\\
    P^{(1)} &= \begin{pmatrix}
		0 & 1 & 0\\
		1 & 0 & 0\\
     	0 & 0 & 1
	\end{pmatrix},
	&
    A^{(1)} &= \begin{gmatrix}[p]
		6 & 6  & 6  \\
		3 & 15 & 13 \\
		2 & 8  & 19
	 \rowops
	 \add[\cdot (-\frac{1}{2})]{0}{1}
	 \add[\cdot (-\frac{1}{3})]{0}{2}
	\end{gmatrix}
	&\\
	L^{(1)} &= \begin{pmatrix}
		1 & 0 & 0\\
		-\frac{1}{2} & 1 & 0\\
     	-\frac{1}{3} & 0 & 1
	\end{pmatrix},
	&
    A^{(2)} &= \begin{gmatrix}[p]
		6 & 6  & 6  \\
		0 & 12 & 10 \\
		0 & 6  & 17
	 \rowops
	 \add[\cdot (-\frac{1}{2})]{1}{2}
	\end{gmatrix}
	&\\
	L^{(2)} &= \begin{pmatrix}
		1 & 0 & 0\\
		0 & 1 & 0\\
     	0 & -\frac{1}{2} & 1
	\end{pmatrix},
	&
    A^{(3)} &= \begin{gmatrix}[p]
		6 & 6  & 6  \\
		0 & 12 & 10 \\
		0 & 0  & 12
	\end{gmatrix}
\end{align*}

Es gilt:

\begin{align}
	L^{(2)} \cdot L^{(1)} \cdot \underbrace{P^{(1)}}_{=: P} \cdot A^{0} &= \underbrace{A^{(3)}}_{=: R}\\
	\Leftrightarrow P A &= (L^{(2)} \cdot L^{(1)})^{-1} \cdot R \\
	\Rightarrow L &= (L^{(2)} \cdot L^{(1)})^{-1}\\
	&= \begin{pmatrix}
		1 & 0 & 0\\
		\frac{1}{2} & 1 & 0\\
		\frac{1}{3} & \frac{1}{2} & 1
	\end{pmatrix}
\end{align}

Nun gilt: $P A = L R = A^{(1)}$ (Kontrolle mit \href{http://www.wolframalpha.com/input/?i=%7B%7B1%2C0%2C0%7D%2C%7B0.5%2C1%2C0%7D%2C%7B1%2F3%2C0.5%2C1%7D%7D*%7B%7B6%2C6%2C6%7D%2C%7B0%2C12%2C10%7D%2C%7B0%2C0%2C12%7D%7D}{Wolfram|Alpha})

\subsection*{Teilaufgabe b}

\textbf{Gegeben:}

\[A =
\begin{pmatrix}
    9 & 4 & 12 \\
    4 & 1  & 4 \\
   12 & 4  & 17
\end{pmatrix}\]

\textbf{Aufgabe:} $A$ auf positive Definitheit untersuchen, ohne Eigenwerte zu berechnen.

\textbf{Vorüberlegung:}
Eine Matrix $A \in \mathbb{R}^{n \times n}$ heißt positiv definit $\dots$
\begin{align*}
  \dots & \Leftrightarrow \forall x \in \mathbb{R}^n \setminus \Set{0}: x^T A x > 0\\
	& \Leftrightarrow \text{Alle Eigenwerte sind größer als 0}
\end{align*}

Falls $A$ symmetrisch ist, gilt:
\begin{align*}
 \text{$A$ ist positiv definit} & \Leftrightarrow \text{alle führenden Hauptminore von $A$ sind positiv}\\
	& \Leftrightarrow \text{es gibt eine Cholesky-Zerlegung $A=GG^T$}\\
\end{align*}

\subsubsection*{Lösung 1: Hauptminor-Kriterium}

\begin{align}
	\det(A_1) &= 9 > 0\\
	\det(A_2) &=
		\begin{vmatrix}
			9 & 4 \\
			4 & 1 \\
		\end{vmatrix} = 9 - 16 < 0\\
	&\Rightarrow \text{$A$ ist nicht positiv definit}
\end{align}

\subsubsection*{Lösung 2: Cholesky-Zerlegung}
\begin{align}
	l_{11} &= \sqrt{a_{11}} = 3\\
	l_{21} &= \frac{a_{21}}{l_{11}} = \frac{4}{3}\\
	l_{31} &= \frac{a_{31}}{l_{11}} = \frac{12}{3} = 4\\
	l_{22} &= \sqrt{a_{22} - {l_{21}}^2} = \sqrt{1 - \frac{16}{9}}= \sqrt{-\frac{7}{9}} \notin \mathbb{R}\\
 & \Rightarrow \text{Es ex. keine Cholesky-Zerlegung, aber $A$ ist symmetrisch}\\
 & \Rightarrow \text{$A$ ist nicht positiv definit}
\end{align}
