\documentclass[a4paper]{scrartcl}
\usepackage{amssymb, amsmath} % needed for math
\usepackage[utf8]{inputenc} % this is needed for umlauts
\usepackage[ngerman]{babel} % this is needed for umlauts
\usepackage[T1]{fontenc}    % this is needed for correct output of umlauts in pdf
\usepackage{pdfpages}       % Signatureinbingung und includepdf
\usepackage{geometry}       % [margin=2.5cm]layout
\usepackage[pdftex]{hyperref}       % links im text
\usepackage{color}
\usepackage{framed}
\usepackage{enumerate}      % for advanced numbering of lists
\usepackage{marvosym}       % checkedbox
\usepackage{wasysym}
\usepackage{braket}         % for \Set{}
\usepackage{pifont}% http://ctan.org/pkg/pifont
\usepackage{gauss}
\usepackage{algorithm,algpseudocode}
\usepackage{parskip}
\usepackage{lastpage}
\usepackage{gauss}
\usepackage{units}
\usepackage{amsthm}
\allowdisplaybreaks

\newcommand{\cmark}{\ding{51}}%
\newcommand{\xmark}{\ding{55}}%

\title{Numerik Klausur 6 - Musterlösung}
\makeatletter
\AtBeginDocument{
	\hypersetup{
	  pdfauthor   = {Martin Thoma, Peter, Felix},
	  pdfkeywords = {Numerik, KIT, Klausur},
	  pdftitle    = {\@title}
  	}
	\pagestyle{fancy}
	\lhead{\@title}
	\rhead{Seite \thepage{} von \pageref{LastPage}}
}
\makeatother

\usepackage{fancyhdr}
\fancyfoot[C]{}

\begin{document}
	\section*{Aufgabe 1}
\textbf{Gegeben:}

\[
A = \begin{pmatrix}
    1 & 2 & 3\\
    2 & 8 & 14\\
    3 & 14 & 34
\end{pmatrix}\]

\textbf{Aufgabe:} Durch Gauß-Elimination die Cholesky-Zerlegung $A = \overline{L} \overline{L}^T$
berechnen

\begin{align*}
    A &=
	\begin{gmatrix}[p]
        1 & 2 & 3\\
        2 & 8 & 14\\
        3 & 14 & 34
        \rowops
        \add[\cdot (-2)]{0}{1}
        \add[\cdot (-3)]{0}{2}
    \end{gmatrix}\\
    \leadsto
    L^{(1)} &=
    \begin{pmatrix}
		1 & 0 & 0\\
	   -2 & 1 & 0\\
       -3 & 0 & 1
	\end{pmatrix},&
    A^{(1)} &=
	\begin{gmatrix}[p]
        1 & 2 & 3\\
        0 & 4 & 8\\
        0 & 8 & 25
        \rowops
        \add[\cdot (-2)]{1}{2}
    \end{gmatrix}\\
    \leadsto
    L^{(2)} &=
    \begin{pmatrix}
		1 & 0 & 0\\
	    0 & 1 & 0\\
        0 & -2 & 1
	\end{pmatrix},&
    A^{(2)} &=
	\begin{gmatrix}[p]
        1 & 2 & 3\\
        0 & 4 & 8\\
        0 & 0 & 9
    \end{gmatrix} =: R\\
    L &= (L^{(2)} \cdot L^{(1)})^{-1}\footnotemark
    &L &= \begin{pmatrix}
        1 & 0 & 0\\
        2 & 1 & 0\\
        3 & 2 & 1
    \end{pmatrix}
\end{align*}
\footnotetext{Da dies beides Frobeniusmatrizen sind, kann einfach die negierten Elemente unter der Diagonalmatrix auf die Einheitsmatrix addieren um das Ergebnis zu erhalten}

Nun gilt:
\begin{align}
    A &= LR = L (DL^T)\\
\Rightarrow A &= \underbrace{(L D^\frac{1}{2})}_{=: \overline{L}} (D^\frac{1}{2} L^T)\\
    \begin{pmatrix}d_1 &0&0\\0&d_2&0\\0&0&d_3\end{pmatrix} \cdot
\begin{pmatrix}
        1 & 2 & 3\\
        0 & 1 & 2\\
        0 & 0 & 1
    \end{pmatrix}
 &= \begin{pmatrix}
        1 & 2 & 3\\
        0 & 4 & 8\\
        0 & 0 & 9
    \end{pmatrix}\\
\Rightarrow D &= \begin{pmatrix}1 &0&0\\0&4&0\\0&0&9\end{pmatrix}\\
\Rightarrow D^\frac{1}{2} &= \begin{pmatrix}1 &0&0\\0&2&0\\0&0&3\end{pmatrix}\\
\overline{L} &= \begin{pmatrix}
        1 & 0 & 0\\
        2 & 1 & 0\\
        3 & 2 & 1
    \end{pmatrix} \cdot \begin{pmatrix}1 &0&0\\0&2&0\\0&0&3\end{pmatrix}\\
    &= \begin{pmatrix}
        1 & 0 & 0\\
        2 & 2 & 0\\
        3 & 4 & 3
    \end{pmatrix}
\end{align}
\clearpage
	\section*{Aufgabe 2}
\subsection*{Teilaufgabe a)}
\begin{align}
	r_{ij} = a_{ij} - \sum_{k=1}^{i-1} l_{ik} \cdot r_{kj} \\ %TODO: Korrektheit überprüfen
	l_{ij} = \frac{a_{ij} - \sum_{k=1}^{j-1} l_{ik} \cdot r_{kj}}{r_{jj}}
\end{align}


\begin{algorithm}
    \begin{algorithmic}
    	\For{$d \in \Set{1, \dots n}$}
    		\State berechne d-te Zeile von $R$
    		\State berechne d-te Spalte von $L$
	\EndFor
    \end{algorithmic}
\end{algorithm}

\clearpage
	\section*{Aufgabe 3}

\begin{table}[H]
    \begin{tabular}{l|l|l|l|l}
    $f_i$ & 8 & 3 & 4 & 8 \\ \hline
    $x_i$ & -1 & 0 & 1 & 3 \\
    \end{tabular}
\end{table}

\subsection*{Teilaufgabe i}
\begin{align}
	p(x) = \sum_{i=0}^3 f_i \cdot L_i(x)
\end{align}
mit
\begin{align}
	L_0(x) &= \frac{(x-x_1)(x-x_2)(x-x_3)}{(x_0 - x_1)(x_0-x_2)(x_0-x_3)}
	= \ldots = \frac{x^3 - 4x^2 + 3x}{-8} \\
	L_1(x) &= \frac{x^3 - 3x^2 - x + 3}{3} \\
	L_2(x) &= \frac{x^3 - 2x^2 - 3x}{-4} \\
	L_3(x) &= \frac{x^3 - x}{24}
\end{align}

\subsection*{Teilaufgabe ii}
Anordnung der dividierten Differenzen im so genannten Differenzenschema:
\begin{table}[H]
    \begin{tabular}{llll}
    $f[x_0]=f_0=8$ & ~                                                   & ~             & ~                \\
    $f[x_1]= 3$    & $f[x_0,x_1] = \frac{f[x_0] - f[x_1]}{x_0-x_1} = -5$ & ~             & ~                \\
    $f[x_2] = 4$   & $1$                                                 & $3$           & ~                \\
    $f[x_3] = 8$   & $2$                                                 & $\frac{1}{3}$ & $- \frac{2}{3} $ \\
    \end{tabular}
\end{table}
Also:
\begin{align}
	p(x) &= f[x_0] + f[x_0,x_1] \cdot (x-x_0) + f[x_0, x_1, x_2] \cdot (x-x_0) \cdot (x-x_1) \\ & + f[x_0, x_1, x_2, x_3] \cdot (x-x_0) \cdot (x-x_1) \cdot (x-x_2) \\
	&= 8 - 5 \cdot (x-x_0) + 3 \cdot (x-x_0) \cdot (x-x_1) \\ & - \frac{2}{3} \cdot (x-x_0) \cdot (x-x_1) \cdot (x-x_2)
\end{align}

	\section*{Aufgabe 4}

Diese Aufgabe ist identisch zu Aufgabe 3, Klausur2.
Die Lösung ist bei Klausur2 zu finden.

	\section*{Aufgabe 5}

Zunächst ist nach der Familie von Quadraturformeln gefragt, für die gilt: ($p := $ Ordnung der QF)
\begin{align}
	s = 3 \\
	0 = c_1 < c_2 < c_3 \\
	p \ge 4
\end{align}

Nach Satz 29 sind in der Familie genau die QFs, für die gilt: \\
Für alle Polynome $g(x)$ mit Grad $\le m-1 = 0$ gilt:
\begin{align}
	 \int_0^1 M(x) \cdot g(x) \mathrm{d}x = 0 \label{a3}
\end{align}

Da eine Quadraturformel höchstens Grad $2s=6$ (Satz 30) haben kann und es wegen
$c_1 = 0$ nicht die Gauss-Quadratur sein kann (Satz 31), kommt nur Ordnung $p=4$
und $p=5$ in Frage.

In dieser Aufgabe sind nur die symmetrischen QF, also die von Ordnung
$p=4$ explizit anzugeben. Für die QF von Ordnung $p=5$ hätte man nur
die Gewichte in Abhängigkeit der Knoten darstellen müssen und
eine Bedinung nur an die Knoten herleiten müssen.

\subsection*{Ordnung 4}
Es gilt $g(x) = c$ für eine Konstante $c$, da $\text{Grad}(g(x))=0$ ist.
Also ist \ref{a3} gleichbedeutend mit:
\begin{align}
	 \int_0^1 M(x) \cdot c \mathrm{d}x &= 0 \\
	 \Leftrightarrow c \cdot \int_0^1 M(x) \mathrm{d}x &= 0 \\
 	 \Leftrightarrow \int_0^1 M(x) \mathrm{d}x &= 0 \\
 	 \Leftrightarrow \int_0^1 (x-c_1)(x-c_2)(x-c_3) \mathrm{d}x &= 0 \\
 	 \Leftrightarrow \frac{1}{4} - \frac{1}{3} \cdot (c_2 + c_3) + \frac{1}{2} \cdot c_2 \cdot c_3 &= 0 \\
 	 \Leftrightarrow \frac{\frac{1}{4} - \frac{1}{3} \cdot c_3}
 	                      {\frac{1}{3} - \frac{1}{2} \cdot c_3} &= c_2
\end{align}

Natürlich müssen auch die Gewichte optimal gewählt werden. Dafür wird Satz 28 genutzt:
Sei $b^T = (b_1, b_2, b_3)$ der Gewichtsvektor. Sei zudem $C :=
\begin{pmatrix}
    {c_1}^0 & {c_2}^0 & {c_3}^0 \\
    {c_1}^1 & {c_2}^1 & {c_3}^1 \\
    {c_1}^2 & {c_2}^2 & {c_3}^2
\end{pmatrix}
$. \\
Dann gilt: $C$ ist invertierbar und $b = C^{-1} \cdot
\begin{pmatrix}
    1 \\
    \frac{1}{2} \\
    \frac{1}{3}
\end{pmatrix}
$.

Es gibt genau eine symmetrische QF in der Familie. Begründung: \\
Aus $c_1 = 0 $ folgt, dass $c_3 = 1$ ist. Außerdem muss $c_2 = \frac{1}{2} $ sein. Also sind die Knoten festgelegt. Da wir die Ordnung $\ge s = 3$ fordern, sind auch die Gewichte eindeutig. \\
Es handelt sich um die aus der Vorlesung bekannte Simpsonregel.

\subsection*{Ordnung 5}
Es gilt $g(x) = ax+c$ für Konstanten $a \neq 0, c$, da $\text{Grad}(g(x))=1$ ist.
Also ist \ref{a3} gleichbedeutend mit:
\begin{align}
	 \int_0^1 M(x) \cdot (ax+c) \mathrm{d}x &= 0 \\
    \Leftrightarrow a \int_0^1 x M(x) \mathrm{d}x + c \int_0^1 M(x) \mathrm{d}x &= 0 \\
    \Leftrightarrow a \int_0^1 x (x-c_1)(x-c_2)(x-c_3) \mathrm{d}x + c \int_0^1 (x-c_1)(x-c_2)(x-c_3) \mathrm{d}x &= 0 \\
    \stackrel{c_1=0}{\Leftrightarrow} a \int_0^1 x^2(x-c_2)(x-c_3) \mathrm{d}x + c \int_0^1 x(x-c_2)(x-c_3) \mathrm{d}x &= 0 \\
    \Leftrightarrow a \left (\frac{c_2 c_3}{3}-\frac{c_2}{4}-\frac{c_3}{4}+\frac{1}{5} \right ) + c \left ( \frac{c_2 c_3}{2}-\frac{c_2}{3}-\frac{c_3}{3}+\frac{1}{4} \right ) &= 0 \\
    \Leftrightarrow \left (\frac{c_2 c_3}{3}-\frac{c_2}{4}-\frac{c_3}{4}+\frac{1}{5} \right ) + \underbrace{\frac{c}{a}}_{=: d} \left ( \frac{c_2 c_3}{2}-\frac{c_2}{3}-\frac{c_3}{3}+\frac{1}{4} \right ) &= 0
\end{align}

Nun habe ich \href{http://www.wolframalpha.com/input/?i=(1%2F5+-+c%2F4+%2B+(b+(-3+%2B+4+c))%2F12)%2B+d*(3+-+4+c+%2B+b+(-4+%2B+6+c))%2F12%3D0}{Wolfram|Alpha} lösen lassen:
\begin{align}
    c_2 &= \frac{6-\sqrt{6}}{10} \approx 0.355\\
    c_3 &= \frac{6+\sqrt{6}}{10} \approx 0.845
\end{align}

Wegen der Ordnungsbedingungen gilt nun:
\begin{align}
    1 &= b_1 + b_2 + b_3\\
    \frac{1}{2} &= b_2 \cdot \frac{6-\sqrt{6}}{10} + b_3 \cdot \frac{6+\sqrt{6}}{10}\\
    \frac{1}{3} &= b_2 \cdot \left (\frac{6-\sqrt{6}}{10} \right )^2 + b_3 \cdot \left (\frac{6+\sqrt{6}}{10} \right )^2\\
    \Leftrightarrow \frac{1}{3} - b_3 \cdot \left (\frac{6+\sqrt{6}}{10} \right )^2 &= b_2 \cdot \left (\frac{6-\sqrt{6}}{10} \right )^2\\
    \Leftrightarrow \frac{\frac{1}{3} - b_3 \cdot \left (\frac{6+\sqrt{6}}{10} \right )^2}{\left (\frac{6-\sqrt{6}}{10} \right )^2} &= b_2\\
    \Leftrightarrow b_2 &= \frac{100}{3 \cdot (6-\sqrt{6})^2} - b_3 \cdot \frac{(6+\sqrt{6})^2}{(6-\sqrt{6})^2}\\
    \Rightarrow \frac{1}{2} &= \left ( \frac{100}{3 \cdot (6-\sqrt{6})^2} - b_3 \cdot \frac{(6+\sqrt{6})^2}{(6-\sqrt{6})^2} \right ) \cdot \frac{6-\sqrt{6}}{10} + b_3 \cdot \frac{6+\sqrt{6}}{10}\\
    &= \left (\frac{10}{3 \cdot (6 - \sqrt{6})} - b_3 \cdot \frac{(6+\sqrt{6})^2}{10 \cdot (6 - \sqrt{6})} \right ) + b_3 \cdot \frac{6+\sqrt{6}}{10}\\
    &= b_3 \cdot \left (\frac{6+\sqrt{6}}{10} - \frac{(6+\sqrt{6})^2}{10 \cdot (6 - \sqrt{6})} \right ) + \frac{10}{3 \cdot (6 - \sqrt{6})}\\
    &= b_3 \cdot \left (\frac{30-(6+\sqrt{6})^2}{10 \cdot (6 - \sqrt{6})} \right ) + \frac{10}{3 \cdot (6 - \sqrt{6})}\\
\Leftrightarrow \frac{1}{2} - \frac{10}{3 \cdot (6 - \sqrt{6})} &= b_3 \cdot \left (\frac{30-(6+\sqrt{6})^2}{10 \cdot (6 - \sqrt{6})} \right )\\
\Leftrightarrow \frac{3 \cdot (6 - \sqrt{6}) - 20}{6\cdot (6 - \sqrt{6})} &= b_3 \cdot \left (\frac{30-(6+\sqrt{6})^2}{10 \cdot (6 - \sqrt{6})} \right )\\
\Leftrightarrow b_3 &= \frac{(3 \cdot (6 - \sqrt{6}) - 20) \cdot 10 \cdot (6 - \sqrt{6})}{6\cdot (6 - \sqrt{6}) \cdot (30-(6+\sqrt{6})^2)}\\
&= \frac{(3 \cdot (6 - \sqrt{6}) - 20) \cdot 5}{3 \cdot (30-(6+\sqrt{6})^2)}\\
&= \frac{15 \cdot (6 - \sqrt{6}) - 100}{90-3 \cdot (6+\sqrt{6})^2}\\
   \Aboxed{b_3 &= \frac{16-\sqrt{6}}{36}} \approx 0.3764\\
   \Aboxed{b_2 &= \frac{16+\sqrt{6}}{36}} \approx 0.5125\\
   \stackrel{\text{Ordnungsbedinung 1}}{\Rightarrow} \Aboxed{b_1 &= \frac{1}{9}}\\
    \frac{1}{4} &\stackrel{?}{=} \frac{16+\sqrt{6}}{36} \cdot \left (\frac{6-\sqrt{6}}{10} \right )^3 + \frac{16-\sqrt{6}}{36} \cdot \left (\frac{6+\sqrt{6}}{10} \right)^3 \text{ \cmark}\\
    \frac{1}{5} &\stackrel{?}{=} \frac{16+\sqrt{6}}{36} \cdot \left (\frac{6-\sqrt{6}}{10} \right )^4 + \frac{16-\sqrt{6}}{36} \cdot \left (\frac{6+\sqrt{6}}{10} \right)^4 \text{ \cmark}\\
    \frac{1}{6} &\stackrel{?}{=} \frac{16+\sqrt{6}}{36} \cdot \left (\frac{6-\sqrt{6}}{10} \right )^5 + \frac{16-\sqrt{6}}{36} \cdot \left (\frac{6+\sqrt{6}}{10} \right)^5 = \frac{33}{200} \text{ \xmark}
\end{align}

\end{document}
