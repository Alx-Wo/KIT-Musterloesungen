\section*{Aufgabe 3}

\begin{table}[H]
    \begin{tabular}{l|l|l|l|l}
    $f_i$ & 8 & 3 & 4 & 8 \\ \hline
    $x_i$ & -1 & 0 & 1 & 3 \\
    \end{tabular}
\end{table}

\subsection*{Teilaufgabe i}
\begin{align}
	p(x) = \sum_{i=0}^3 f_i \cdot L_i(x)
\end{align}
mit
\begin{align}
	L_0(x) &= \frac{(x-x_1)(x-x_2)(x-x_3)}{(x_0 - x_1)(x_0-x_2)(x_0-x_3)}
	= \ldots = \frac{x^3 - 4x^2 + 3x}{-8} \\
	L_1(x) &= \frac{x^3 - 3x^2 - x + 3}{3} \\
	L_2(x) &= \frac{x^3 - 2x^2 - 3x}{-4} \\
	L_3(x) &= \frac{x^3 - x}{24}
\end{align}

\subsection*{Teilaufgabe ii}
Anordnung der dividierten Differenzen im so genannten Differenzenschema:
\begin{table}[H]
    \begin{tabular}{llll}
    $f[x_0]=f_0=8$ & ~                                                   & ~             & ~                \\
    $f[x_1]= 3$    & $f[x_0,x_1] = \frac{f[x_0] - f[x_1]}{x_0-x_1} = -5$ & ~             & ~                \\
    $f[x_2] = 4$   & $1$                                                 & $3$           & ~                \\
    $f[x_3] = 8$   & $2$                                                 & $\frac{1}{3}$ & $- \frac{2}{3} $ \\
    \end{tabular}
\end{table}
Also:
\begin{align}
	p(x) &= f[x_0] + f[x_0,x_1] \cdot (x-x_0) + f[x_0, x_1, x_2] \cdot (x-x_0) \cdot (x-x_1) \\ & + f[x_0, x_1, x_2, x_3] \cdot (x-x_0) \cdot (x-x_1) \cdot (x-x_2) \\
	&= 8 - 5 \cdot (x-x_0) + 3 \cdot (x-x_0) \cdot (x-x_1) \\ & - \frac{2}{3} \cdot (x-x_0) \cdot (x-x_1) \cdot (x-x_2)
\end{align}
