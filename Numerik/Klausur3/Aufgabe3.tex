\section*{Aufgabe 3}
\subsection*{Teilaufgabe a)}

\begin{enumerate}
\item Selbstabbildung: \\
	Sei $x \in D := [1.75 , 2] = [\frac{7}{4}, \frac{8}{4}]$.

	Dann:
	\begin{align}
		F(x) = 1 + \frac{1}{x} + \frac{1}{x^2} \le 1 + \frac{1}{1.75} + \frac{1}{1.75^2} = 1 + \frac{44}{49} \le 2 %TODO: schöner formulieren
	\end{align}
	und: \\
	\begin{align}
		F(x) = 1 + \frac{1}{x} + \frac{1}{x^2} \ge 1 + \frac{1}{2} + \frac{1}{4} = 1.75
	\end{align}

\item Abgeschlossenheit: $D$ ist offentsichtlich abgeschlossen.
\item Kontraktion: \\ %TODO:
    %\textbf{Behauptung:} $F(x)$ ist auf $A$ eine Kontraktion.
    %\textbf{Beweis:}
    %z.Z.: $\exists L \in [0, 1): \forall x,y \in A: || F(x) - F(y) || \leq L \cdot || x - y||$
    Hier ist der Mittelwertsatz der Differentialrechnung von Nutzen.\\
	$F$ ist Lipschitz-stetig auf $D$ und für alle $x \in D$ gilt: \\
	\begin{align}
		|F'(x)| = |-\frac{1}{x^2}-2 \cdot \frac{1}{x^3}| \le \frac{240}{343} =: \theta < 1
	\end{align}
	Also gilt auch $\forall x,y \in D $:
	\begin{align}
		|F(x) - F(y)| \le \theta \cdot |x - y|
	\end{align}
	Somit ist die Lipschitz- bzw. Kontraktions-Konstante $\theta$.
\end{enumerate}
Insgesamt folgt, dass $F$ die Voraussetzungen des Banachschen Fixpunktsatzes erfüllt.
