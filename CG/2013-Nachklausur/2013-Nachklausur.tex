\documentclass[a4paper]{scrartcl}
\usepackage[ngerman]{babel}
\usepackage[utf8]{inputenc}
\usepackage{amssymb,amsmath}
\usepackage{graphicx}
\usepackage[inline]{enumitem}
\setlist{noitemsep}
\usepackage[binary-units=true]{siunitx}
\usepackage{hyperref}
\usepackage{parskip}
\usepackage[nameinlink,noabbrev,ngerman]{cleveref} % has to be after hyperref
\usepackage{nicefrac}  % for \nicefrac{1}{3}
\usepackage{csquotes}  % for \enquote{what you want to quote}
\usepackage{booktabs}  % for \toprule, \midrule and \bottomrule
\usepackage{minted} % needed for the inclusion of source code

% for \begin{enumerate}[label=(\Alph*)], see http://tex.stackexchange.com/a/129960/5645
\usepackage{enumitem}

\setcounter{secnumdepth}{2}
\setcounter{tocdepth}{2}

\usepackage{wasysym}  % For \CheckedBox
\usepackage{microtype}

\begin{document}
\selectlanguage{ngerman}
\title{2013 Nachklausur (WS 2012/13)}

\setcounter{section}{1}
%%%%%%%%%%%%%%%%%%%%%%%%%%%%%%%%%%%%%%%%%%%%%%%%%%%%%%%%%%%%%%%%%%%%%%%%%%%%%%
\section*{Aufgabe 1: Raytracing}
\subsection*{Teilaufgabe 1a}
\begin{figure}[h]
    \centering
    \includegraphics*[width=0.8\linewidth, keepaspectratio]{1a.png}
    \caption{Reflexionsstrahl, Schattenstrahlen und Transmissionstrahl}
    \label{fig:1a}
\end{figure}

\subsection*{Teilaufgabe 1b}
\textit{Wie nennt man das physikalische Gesetz oder Prinzip, welches die Richtungsänderung
eines Lichtstrahls beim Übergang in ein anderes Medium beschreibt?}

Snellsches Gesetz ($\eta_0 \cdot \sin \theta_0 = \theta_1 \cdot \sin \delta_1$)

\subsection*{Teilaufgabe 1c}
\textit{Welche Bedingung muss gelten, damit beim Übergang eines Lichtstrahls
von einem Medium mit Refraktionsindex $\eta_0$ in ein Medium mit
Refraktionsindex $\eta_1$ Totalreflexion auftreten kann?}

Der Einfallswinkel muss einen Grenzwinkel
$\theta = \arcsin \frac{\eta_1}{\eta_0}$
überschreiten (also besonders flach auf das Material sein).


\section*{Aufgabe 2}
TODO

\section*{Aufgabe 3}
TODO

\section*{Aufgabe 4}
TODO

\section*{Aufgabe 5}
TODO

\section*{Aufgabe 6}
TODO

\section*{Aufgabe 7}
TODO

\section*{Aufgabe 8}
TODO

\section*{Aufgabe 9}
TODO

\section*{Aufgabe 10}
TODO

\section*{Aufgabe 11}
\inputminted[linenos, numbersep=5pt, tabsize=4, frame=lines, label=shader.frag]{glsl}{shader.frag}

\section*{Aufgabe 12}
TODO

\end{document}
