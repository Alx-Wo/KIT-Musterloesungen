\documentclass[a4paper]{scrartcl}
\usepackage[ngerman]{babel}
\usepackage[utf8]{inputenc}
\usepackage{amssymb,amsmath}
\usepackage{graphicx}
\usepackage[inline]{enumitem}
\setlist{noitemsep}
\usepackage[binary-units=true]{siunitx}
\usepackage{hyperref}
\usepackage{parskip}
\usepackage[nameinlink,noabbrev,ngerman]{cleveref} % has to be after hyperref
\usepackage{nicefrac}  % for \nicefrac{1}{3}
\usepackage{csquotes}  % for \enquote{what you want to quote}
\usepackage{booktabs}  % for \toprule, \midrule and \bottomrule
\usepackage{minted} % needed for the inclusion of source code

% for \begin{enumerate}[label=(\Alph*)], see http://tex.stackexchange.com/a/129960/5645
\usepackage{enumitem}

\setcounter{secnumdepth}{2}
\setcounter{tocdepth}{2}

\usepackage{wasysym}  % For \CheckedBox
\usepackage{microtype}

\begin{document}
\selectlanguage{ngerman}
\title{2013 Hauptklausur (WS 2012/13)}

\setcounter{section}{1}
%%%%%%%%%%%%%%%%%%%%%%%%%%%%%%%%%%%%%%%%%%%%%%%%%%%%%%%%%%%%%%%%%%%%%%%%%%%%%%
\section*{Aufgabe 1: Raytracing}
\subsection*{Teilaufgabe 1a}
\textit{Raytracing nach Whitted, wie Sie es in der Vorlesung kennengelernt
haben, folgt den Gesetzen der geometrischen Optik. Ergänzen Sie die folgende
Liste um die 3 weiteren Strahltypen, die bei diesem Raytracing-Verfahren
vorkommen!}

\begin{enumerate*}[label=(\arabic*)]
    \item Primärstrahlen
    \item Reflektionsstrahlen (rekursiv)
    \item Transmissionsstrahlen (rekursiv)
    \item Schattenstrahlen
\end{enumerate*}

\subsection*{Teilaufgabe 1b}
\textit{Die folgenden Skizzen zeigen zwei Lichtstrahlen mit unterschiedlichem
Einfallswinkel die an einer spekularen Glasoberfläche reflektiert werden (der
Vektor N ist die Oberflächennormale).}

In Bild 2, da dort der Winkel des Strahls auf die Oberfläche flacher ist.

\subsection*{Teilaufgabe 1c}
\textit{Wie nennt man das physikalische Gesetz oder Prinzip, welches den
Zusammenhang zwischen Einfallswinkel und Reflektivität beschreibt?}

Snelliussches Brechungsgesetz. Es lautet
\[n_1 \cdot \sin(\theta_1) = n_1 \cdot \sin(\theta_2)\]
wobei die Winkel von der Oberflächennormale aus gemessen werden. $n_1, n_2$
sind Materialkonstanten.

\clearpage
\section*{Aufgabe 2: Beleuchtung, Licht und Wahrnehmung}
\subsection*{Teilaufgabe 2a}
\textit{Ergänzen Sie die Skizze und zeichnen Sie die 4~Vektoren ein, die im
Phong-Beleuchtungsmodell für die Beleuchtungsberechnung benötigt werden!
Verwenden Sie für die Skizze die Betrachterposition $B_1$ und den
Oberflächenpunkt $x_1$}

\includegraphics*[width=0.8\linewidth, keepaspectratio]{2a.png}

Die 4 Vektoren sind:

\begin{itemize}
    \item View-Vektor $V$
    \item Normale $N$,
    \item Licht-Vektor $L$ und
    \item Reflektionsvektor $R_L$
\end{itemize}

\[I = \underbrace{k_a \cdot I_L}_{\text{ambient}} + \underbrace{k_d \cdot I_L \cdot (N \cdot L)}_{\text{diffus}} + \underbrace{k_s \cdot I_L (R_L \cdot V)^n}_{\text{spekular}}\]

\subsection*{Teilaufgabe 2b}
\textit{Der Wert welcher Komponente(n) des Phong-Beleuchtungsmodells verändert
bzw. verändern sich, wenn in der obigen Situation\dots}

\begin{enumerate}[label=(\roman*)]
    \item \textit{\dots der Punkt $x_2$ statt $x_1$ betrachtet wird?}\\
          $L, R_L, V$: Spekular und diffus
    \item \textit{\dots die Szene aus der Position $B_2$ statt $B_1$ betrachtet wird?}\\
          $V$: diffus
\end{enumerate}

\subsection*{Teilaufgabe 2c}
\textit{In welcher Komponente taucht der sogenannte Phong-Exponent auf und
welchen Einfluss hat er auf die Erscheinung einer Oberfläche? Wie ändert sich
das Aussehen, wenn der Phong-Exponent größer gewählt wird?}

Diffuse Komponente

Ein großes $n$ führt dazu, dass \textbf{Glanzlichter} kleiner, aber intensiver
werden. Die reflektion wird \enquote{perfekter}.

\subsection*{Teilaufgabe 2d}
\begin{tabular}{cp{12cm}cc}\toprule
\# & Aussage                                                                                                                                      & Wahr & Falsch \\\midrule
1  & Zu drei gewählten Primärfarben gibt es immer Spektralfarben, die durch die Kombination dieser drei Farben nicht realisierbar sind.           & x    & ~      \\
2  & Menschen können geringe Helligkeitsunterschiede im Bereich niedriger Lichtintensität besser wahrnehmen als im Bereich hoher Lichtintensität. & x    & ~      \\
3  & Es gibt keinen linearen Zusammenhang zwischem dem CIE-XYZ- und dem RGB-Modell.                                                               & ~    & x      \\
4  & Gammakorrektur ist nur bei Röhrenmonitoren notwendig.                                                                                        & ~    & x      \\\bottomrule
\end{tabular}


\section*{Aufgabe 3: Transformationen}
\subsection*{Teilaufgabe 3a}
\textit{Gegeben sind Vektoren in homogenen Koordinaten. Kreuzen Sie jeweils an,
ob es sich um einen Punkt oder eine Richtung handelt. Geben Sie außerdem die
dazugehörigen kartesischen Koordinaten an.}

Ein Punkt hat als letzte Komponente einen Wert $\neq 0$, eine Richtung hat dort
$= 0$.

\begin{table}
    \begin{tabular}{lccl}
    Vektor                                  & Punkt       & Richtung    & Kartesische Koordinaten \\
    $\begin{pmatrix}1,2,3,1\end{pmatrix}$   & \CheckedBox & $\square$   & $(1,2,3)$               \\
    $\begin{pmatrix}1,2,3,0.1\end{pmatrix}$ & \CheckedBox & $\square$   & $(1,2,30)$              \\
    $\begin{pmatrix}1,2,3,0\end{pmatrix}$   & $\square$   & \CheckedBox & $(1,2,3)$               \\
    \end{tabular}
\end{table}

\clearpage
\subsection*{Teilaufgabe 3b}
Korrekt sind:
\begin{enumerate}
    \item dreht die x-Achse in Richtung y-Achse
    \item Scherung um Faktor a
    \item erhält die Parallelität von Linien (bei den anderen beiden ist die Scherung ein Gegenbeispiel)
\end{enumerate}

\section*{Aufgabe 4}
\subsection*{Teilaufgabe 4a}
TODO
\subsection*{Teilaufgabe 4b}
TODO

\section*{Aufgabe 5}
\subsection*{Teilaufgabe 5a}
TODO
\subsection*{Teilaufgabe 5a}
TODO

\section*{Aufgabe 6}
\subsection*{Teilaufgabe 6a}
TODO
\subsection*{Teilaufgabe 6b}
TODO
\subsection*{Teilaufgabe 6c}
\inputminted[linenos, numbersep=5pt, tabsize=4, frame=lines, label=spheretracing.frag]{glsl}{spheretracing.frag}

\section*{Aufgabe 7}
\subsection*{Teilaufgabe 7a}
TODO
\subsection*{Teilaufgabe 7b}
TODO

\section*{Aufgabe 8}
\subsection*{Teilaufgabe 8a}
TODO
\subsection*{Teilaufgabe 8b}
TODO
\subsection*{Teilaufgabe 8c}
TODO

\section*{Aufgabe 9}
\inputminted[linenos, numbersep=5pt, tabsize=4, frame=lines, label=keyframing.vert]{glsl}{keyframing.vert}

\section*{Aufgabe 10}
\subsection*{Teilaufgabe 10a}
\inputminted[linenos, numbersep=5pt, tabsize=4, frame=lines, label=shader.frag]{glsl}{shader.frag}

\subsection*{Teilaufgabe 10b}
TODO

\subsection*{Teilaufgabe 10c}
TODO

\section*{Aufgabe 11: Bézierkurven}
\subsection*{Teilaufgabe 11a}
TODO
\subsection*{Teilaufgabe 11b}
TODO

\end{document}
