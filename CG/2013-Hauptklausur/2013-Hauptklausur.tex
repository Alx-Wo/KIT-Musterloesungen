\documentclass[a4paper]{scrartcl}
\usepackage[ngerman]{babel}
\usepackage[utf8]{inputenc}
\usepackage{amssymb,amsmath}
\usepackage{graphicx}
\usepackage[inline]{enumitem}
\setlist{noitemsep}
\usepackage[binary-units=true]{siunitx}
\usepackage{hyperref}
\usepackage{parskip}
\usepackage[nameinlink,noabbrev,ngerman]{cleveref} % has to be after hyperref
\usepackage{nicefrac}
\usepackage{csquotes}
\usepackage{booktabs}  % for \toprule, \midrule and \bottomrule

\usepackage{minted} % needed for the inclusion of source code

\setcounter{secnumdepth}{2}
\setcounter{tocdepth}{2}

\usepackage{microtype}

\begin{document}
\selectlanguage{ngerman}
\title{2013 Hauptklausur (WS 2012/13)}

\setcounter{section}{1}
%%%%%%%%%%%%%%%%%%%%%%%%%%%%%%%%%%%%%%%%%%%%%%%%%%%%%%%%%%%%%%%%%%%%%%%%%%%%%%
\section*{Aufgabe 1: Raytracing}
\subsection*{Teilaufgabe 1a}
\textit{Raytracing nach Whitted, wie Sie es in der Vorlesung kennengelernt
haben, folgt den Gesetzen der geometrischen Optik. Ergänzen Sie die folgende
Liste um die 3 weiteren Strahltypen, die bei diesem Raytracing-Verfahren
vorkommen!}

\begin{enumerate*}[label=(\arabic*)]
    \item Primärstrahlen
    \item Reflektionsstrahlen (rekursiv)
    \item Transmissionsstrahlen (rekursiv)
    \item Schattenstrahlen
\end{enumerate*}

\subsection*{Teilaufgabe 1b}
\textit{Die folgenden Skizzen zeigen zwei Lichtstrahlen mit unterschiedlichem
Einfallswinkel die an einer spekularen Glasoberfläche reflektiert werden (der
Vektor N ist die Oberflächennormale).}

In Bild 2, da dort der Winkel des Strahls auf die Oberfläche flacher ist.

\subsection*{Teilaufgabe 1c}
\textit{Wie nennt man das physikalische Gesetz oder Prinzip, welches den
Zusammenhang zwischen Einfallswinkel und Reflektivität beschreibt?}

Snelliussches Brechungsgesetz. Es lautet
\[n_1 \cdot \sin(\theta_1) = n_1 \cdot \sin(\theta_2)\]
wobei die Winkel von der Oberflächennormale aus gemessen werden. $n_1, n_2$
sind Materialkonstanten.

\section*{Aufgabe 2}
\subsection*{Teilaufgabe 2a}
\begin{figure}[h]
    \centering
    \includegraphics*[width=0.8\linewidth, keepaspectratio]{1a.png}
    \caption{Whatever}
    \label{fig:1a}
\end{figure}
TODO

\subsection*{Teilaufgabe 2b}
TODO

\subsection*{Teilaufgabe 2c}
TODO

\subsection*{Teilaufgabe 2d}
TODO

\section*{Aufgabe 3}
\subsection*{Teilaufgabe 3a}
TODO

\subsection*{Teilaufgabe 3b}
TODO

\section*{Aufgabe 4}
\subsection*{Teilaufgabe 4a}
TODO
\subsection*{Teilaufgabe 4b}
TODO

\section*{Aufgabe 5}
\subsection*{Teilaufgabe 5a}
TODO
\subsection*{Teilaufgabe 5a}
TODO

\section*{Aufgabe 6}
\subsection*{Teilaufgabe 6a}
TODO
\subsection*{Teilaufgabe 6b}
TODO
\subsection*{Teilaufgabe 6c}
\inputminted[linenos, numbersep=5pt, tabsize=4, frame=lines, label=spheretracing.frag]{glsl}{spheretracing.frag}

\section*{Aufgabe 7}
\subsection*{Teilaufgabe 7a}
TODO
\subsection*{Teilaufgabe 7b}
TODO

\section*{Aufgabe 8}
\subsection*{Teilaufgabe 8a}
TODO
\subsection*{Teilaufgabe 8b}
TODO
\subsection*{Teilaufgabe 8c}
TODO

\section*{Aufgabe 9}
\inputminted[linenos, numbersep=5pt, tabsize=4, frame=lines, label=keyframing.vert]{glsl}{keyframing.vert}

\section*{Aufgabe 10}
\subsection*{Teilaufgabe 10a}
\inputminted[linenos, numbersep=5pt, tabsize=4, frame=lines, label=shader.frag]{glsl}{shader.frag}

\subsection*{Teilaufgabe 10b}
TODO

\subsection*{Teilaufgabe 10c}
TODO

\section*{Aufgabe 11: Bézierkurven}
\subsection*{Teilaufgabe 11a}
TODO
\subsection*{Teilaufgabe 11b}
TODO

\end{document}
