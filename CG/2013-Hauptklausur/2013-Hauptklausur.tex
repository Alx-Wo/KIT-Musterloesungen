\documentclass[a4paper]{scrartcl}
\usepackage[ngerman]{babel}
\usepackage[utf8]{inputenc}
\usepackage{amssymb,amsmath}
\usepackage{graphicx}
\usepackage[inline]{enumitem}
\setlist{noitemsep}
\usepackage[binary-units=true]{siunitx}
\usepackage{hyperref}
\usepackage{parskip}
\usepackage[nameinlink,noabbrev,ngerman]{cleveref} % has to be after hyperref
\usepackage{nicefrac}  % for \nicefrac{1}{3}
\usepackage{csquotes}  % for \enquote{what you want to quote}
\usepackage{booktabs}  % for \toprule, \midrule and \bottomrule
\usepackage{minted} % needed for the inclusion of source code

% for \begin{enumerate}[label=(\Alph*)], see http://tex.stackexchange.com/a/129960/5645
\usepackage{enumitem}

\setcounter{secnumdepth}{2}
\setcounter{tocdepth}{2}

\usepackage{wasysym}  % For \CheckedBox
\usepackage{microtype}

\begin{document}
\selectlanguage{ngerman}
\title{2013 Hauptklausur (WS 2012/13)}

\setcounter{section}{1}
%%%%%%%%%%%%%%%%%%%%%%%%%%%%%%%%%%%%%%%%%%%%%%%%%%%%%%%%%%%%%%%%%%%%%%%%%%%%%%
\section*{Aufgabe 1: Raytracing}
\subsection*{Teilaufgabe 1a}
\textit{Raytracing nach Whitted, wie Sie es in der Vorlesung kennengelernt
haben, folgt den Gesetzen der geometrischen Optik. Ergänzen Sie die folgende
Liste um die 3 weiteren Strahltypen, die bei diesem Raytracing-Verfahren
vorkommen!}

\begin{enumerate*}[label=(\arabic*)]
    \item Primärstrahlen
    \item Reflektionsstrahlen (rekursiv)
    \item Transmissionsstrahlen (rekursiv)
    \item Schattenstrahlen
\end{enumerate*}

\subsection*{Teilaufgabe 1b}
\textit{Die folgenden Skizzen zeigen zwei Lichtstrahlen mit unterschiedlichem
Einfallswinkel die an einer spekularen Glasoberfläche reflektiert werden (der
Vektor N ist die Oberflächennormale).}

In Bild 2, da dort der Winkel des Strahls auf die Oberfläche flacher ist.

\subsection*{Teilaufgabe 1c}
\textit{Wie nennt man das physikalische Gesetz oder Prinzip, welches den
Zusammenhang zwischen Einfallswinkel und Reflektivität beschreibt?}

Die \textbf{Fresnelsche Formeln} beschreiben die Beziehung zwischen
Transmission und Reflexion in Abhängigkeit des Winkels.

\clearpage
\section*{Aufgabe 2: Beleuchtung, Licht und Wahrnehmung}
\subsection*{Teilaufgabe 2a}
\textit{Ergänzen Sie die Skizze und zeichnen Sie die 4~Vektoren ein, die im
Phong-Beleuchtungsmodell für die Beleuchtungsberechnung benötigt werden!
Verwenden Sie für die Skizze die Betrachterposition $B_1$ und den
Oberflächenpunkt $x_1$}

\includegraphics*[width=0.8\linewidth, keepaspectratio]{2a.png}

Die 4 Vektoren sind:

\begin{itemize}
    \item View-Vektor $V$
    \item Normale $N$,
    \item Licht-Vektor $L$ und
    \item Reflektionsvektor $R_L$
\end{itemize}

\[I = \underbrace{k_a \cdot I_L}_{\text{ambient}} + \underbrace{k_d \cdot I_L \cdot (N \cdot L)}_{\text{diffus}} + \underbrace{k_s \cdot I_L (R_L \cdot V)^n}_{\text{spekular}}\]

\subsection*{Teilaufgabe 2b}
\textit{Der Wert welcher Komponente(n) des Phong-Beleuchtungsmodells verändert
bzw. verändern sich, wenn in der obigen Situation\dots}

\begin{enumerate}[label=(\roman*)]
    \item \textit{\dots der Punkt $x_2$ statt $x_1$ betrachtet wird?}\\
          $L, R_L, V$: Spekular und diffus
    \item \textit{\dots die Szene aus der Position $B_2$ statt $B_1$ betrachtet wird?}\\
          $V$: Spekular (Glanzlichter sind abhängig vom Winkel zwischen $V$ und $R_L$)
\end{enumerate}

\subsection*{Teilaufgabe 2c}
\textit{In welcher Komponente taucht der sogenannte Phong-Exponent auf und
welchen Einfluss hat er auf die Erscheinung einer Oberfläche? Wie ändert sich
das Aussehen, wenn der Phong-Exponent größer gewählt wird?}

Spekulare Komponente

Ein großes $n$ führt dazu, dass \textbf{Glanzlichter} kleiner, aber intensiver
werden. Die reflektion wird \enquote{perfekter}.

\subsection*{Teilaufgabe 2d}
\begin{tabular}{cp{12cm}cc}\toprule
\# & Aussage                                                                                                                                      & Wahr & Falsch \\\midrule
1  & Zu drei gewählten Primärfarben gibt es immer Spektralfarben, die durch die Kombination dieser drei Farben nicht realisierbar sind.           & x    & ~      \\
2  & Menschen können geringe Helligkeitsunterschiede im Bereich niedriger Lichtintensität besser wahrnehmen als im Bereich hoher Lichtintensität. & x    & ~      \\
3  & Es gibt keinen linearen Zusammenhang zwischem dem CIE-XYZ- und dem RGB-Modell.                                                               & ~    & x      \\
4  & Gammakorrektur ist nur bei Röhrenmonitoren notwendig.                                                                                        & ~    & x      \\\bottomrule
\end{tabular}


\section*{Aufgabe 3: Transformationen}
\subsection*{Teilaufgabe 3a}
\textit{Gegeben sind Vektoren in homogenen Koordinaten. Kreuzen Sie jeweils an,
ob es sich um einen Punkt oder eine Richtung handelt. Geben Sie außerdem die
dazugehörigen kartesischen Koordinaten an.}

Ein Punkt hat als letzte Komponente einen Wert $\neq 0$, eine Richtung hat dort
$= 0$.

\begin{table}
    \begin{tabular}{lccl}
    Vektor                                  & Punkt       & Richtung    & Kartesische Koordinaten \\
    $\begin{pmatrix}1,2,3,1\end{pmatrix}$   & \CheckedBox & $\square$   & $(1,2,3)$               \\
    $\begin{pmatrix}1,2,3,0.1\end{pmatrix}$ & \CheckedBox & $\square$   & $(10,20,30)$            \\
    $\begin{pmatrix}1,2,3,0\end{pmatrix}$   & $\square$   & \CheckedBox & $(1,2,3)$               \\
    \end{tabular}
\end{table}

\clearpage
\subsection*{Teilaufgabe 3b}
Korrekt sind:
\begin{enumerate}
    \item dreht die y-Achse in Richtung z-Achse
    \item Scherung um Faktor a
    \item ist teilverhältnistreu; erhält die Parallelität von Linien
\end{enumerate}

\section*{Aufgabe 4: Texturen und Texture-Mapping}
\subsection*{Teilaufgabe 4a}
\textit{Was versteht man unter Mip-Mapping? Welches Problem beim Texture Mapping soll
damit gelöst werden und wann tritt dieses Problem auf? Wie erzeugt man Mip-Maps?}

Mip-Mapping ist eine Vorverarbeitung der Textur, um das \textbf{Aliasing}-Problem
bei \textbf{Minification} zu behandeln.

Es werden kleiner skalierte, vorverarbeitete Versionen der Textur erstellt
(Stufe $i$: um Faktor $2^i$ pro Achse kleiner).

\subsection*{Teilaufgabe 4b}
\textit{Was versteht man unter einer Environment Map? Nennen Sie eine Anwendung
von Environment Mapping. Wie wird auf die Environment Map zugegriffen und
welche vereinfachende Annahme wird dabei gemacht?}

\underline{Was:} Eine Environment-Map ist eine Textur zur Darstellung der Umgebung.

\underline{Anwendung:} Durch eine Environment-Map kann die
Reflektion/Beleuchtung eines Objekts bestimmt werden, ohne aufwendiges
Ray-Tracing zu betreiben.

Konkret: Spiegelungen auf \enquote{flüssigem} Terminator (vgl. Folien)

\underline{Annahme:} Die Umgebung ist weit genug entfernt, sodass die Position
keine Rolle spielt. Es wird nur die Blickrichtung verwendet. Man bestimmt für
Cube-Maps also die Richtung (die Fläche des Würfels) und greift nur
entsprechend dieser auf die Textur der betreffenden Würfelfläche zu.

\section*{Aufgabe 5: Räumliche Datenstrukturen}
\subsection*{Teilaufgabe 5a}
\begin{tabular}{cp{8cm}llll}\toprule
\#& Aussage                                                                                                                                                                                                & BVH     & Octree & kD-Baum & Gitter \\\midrule
1 & Die Struktur eignet sich gut in Fällen, in denen Primitive gehäuft auftreten und große Leerräume zwischen den Häufungen existieren.                                                                    & x       & x      & ~       & ~      \\
2 & In einer Hierarchieebene können sich die Zellen der Struktur überlappen.                                                                                                                               & x       & ~      & ~       & ~      \\
3 & Gehen Sie nun davon aus, dass Primitive nicht unterteilt werden und kein Mailboxing verwendet wird. Dann wird jedes Primitiv in jedem Fall höchstens einmal auf einen Schnitt mit dem Strahl getestet. & x       & ~      & ~       & ~      \\\bottomrule
\end{tabular}

\subsection*{Teilaufgabe 5b}

\begin{itemize}
    \item Am wenigsten aufwendig (1): Gitter
    \item (2): kd-Baum mit Objekt Median
    \item (3): kd-Baum mit Surface Area Heuristic
\end{itemize}

\section*{Aufgabe 6: Prozedurale Modellierung}
\subsection*{Teilaufgabe 6a}
\textit{Nennen Sie drei Vorteile (Stichpunkte) von prozeduralen Texturen!}

\begin{enumerate}[label=(\arabic*)]
    \item Geringerer Speicheraufwand
    \item Natürliche Strukturen lassen sich leicht beschreiben
    \item Beliebige Vergrößerungen sind möglich (TODO: War das so gewünscht?)
\end{enumerate}

Nachteile:

\begin{itemize}
    \item Hoher Rechenaufwand
\end{itemize}

\clearpage
\subsection*{Teilaufgabe 6b}
\textit{Im folgenden Beispiel sollen Sie Sphere-Tracing für Distanzfelder
illustrieren. Das Distanzfeld in dieser Szene beschreibt die grauen Körper. Der
Strahl, der auf Schnitt getestet werden soll, beginnt am Punkt \textbf{A} in
Richtung von Punkt \textbf{B}, wo er auch endet. Zeichnen Sie die Schritte
entlang des Strahls ein!}

\includegraphics*[width=0.8\linewidth, keepaspectratio]{2a.png}

TODO: Einzeichnen

\clearpage
\subsection*{Teilaufgabe 6c}
\textit{Sie sollen nun Sphere-Tracing in der OpenGL Shading Language
programmieren. Dabei soll der nächste Schnittpunkt eines Strahls mit der
Szenengeometrie gefunden werden. Als Abbruchkriterium für die Suche dient die
Distanz \texttt{tMax}. Ein Schnittpunkt ist gefunden, wenn die Distanzfunktion
einen kleineren Wert als epsilon liefert. In diesem Fall soll die Funktion
\texttt{sphereTrace} den Wert true und den Schnittpunkt pos zurückliefern. Die
Anzahl der Sphere-Tracing-Schritte wird immer in steps zurückgegeben. Ihnen
steht die Distanzfunktion \texttt{float DF( vec3 x )} zur Verfügung.}

\inputminted[linenos, numbersep=5pt, tabsize=4, frame=lines, label=spheretracing.frag]{glsl}{spheretracing.frag}

\clearpage
\section*{Aufgabe 7}
\subsection*{Teilaufgabe 7a}
\textit{i) Indexliste(n) für Primitivtyp \texttt{GL\_TRIANGLE\_STRIP}}

p1: (1, 2, 3, 4, 5, 6, 7, 8)

benötigt 1 Primitiv


\textit{ii) Indexliste(n) für Primitivtyp \texttt{GL\_TRIANGLE\_FAN}}

p1: (3, 2, 1, 4, 5)

p2: (6, 4, 5, 7, 8)

benötigt 2 Primitive


\subsection*{Teilaufgabe 7b}
\begin{tabular}{cp{8cm}p{1.5cm}p{1.5cm}p{1.5cm}}\toprule
\#& Aufgabe    
            &  Vertex-Shader     & Geometry-Shader   &   Fragment-Shader \\\midrule
1 & Den Primitivtyp von \texttt{GL\_POINT} auf \texttt{GL\_TRIANGLE} ändern.
            & ~                  & x                 & ~     \\
2 & Die Projektionstransformation auf Vertizes anwenden und das Ergebnis in \texttt{gl\_Position} speichern.
            & x                  & ~                 & ~     \\
3 & Die Fragmentfarbe ausgeben.
            & ~                  & ~                 & x     \\
4 & \texttt{uniform}-Variablen schreiben.
            & ~                  & ~                 & ~     \\
5 & Das Beleuchtungsmodell auswerten, wenn \textit{Phong-Shading} verwendet wird.
            & ~                  & ~                 & x     \\
6 & Aus Texturen lesen.
            & x                  & x                 & x     \\\bottomrule
\end{tabular}


\section*{Aufgabe 8}
\subsection*{Teilaufgabe 8a}
$\lambda_3 = \triangle(x, x_1, x_2)$
\subsection*{Teilaufgabe 8b}
Dreiecksnetz weiter unterteilen (Tesselieren)
\subsection*{Teilaufgabe 8c}

\begin{enumerate}[label=(\roman*)]
    \item Flat shading
        \begin{align}
        n &= \langle x_1-x_2, x_3-x_2\rangle\\
        f &= \langle n , L \rangle^+ \cdot (\lambda_1 c_1 + \lambda_2 c_2 + \lambda_3 c_3)
        \end{align}
    \item Gouraud-Shading
        \begin{align}
        c'_i &= c_i \langle n_i, L \rangle^+\\
        f &= \lambda_1 c'_1 + \lambda_2 c'_2 + \lambda_3 c'_3
        \end{align}
    \item Phong-Shading
        \begin{align}
        n &= \lambda_1 n_1 + \lambda_2 n_2 + \lambda_3 n_3\\
        f &=\langle n , L \rangle^+ \cdot (\lambda_1 c_1 + \lambda_2 c_2 + \lambda_3 c_3)
        \end{align}
\end{enumerate}

\section*{Aufgabe 9}
\inputminted[linenos, numbersep=5pt, tabsize=4, frame=lines, label=keyframing.vert]{glsl}{keyframing.vert}

\section*{Aufgabe 10}
\subsection*{Teilaufgabe 10a}
\inputminted[linenos, numbersep=5pt, tabsize=4, frame=lines, label=shader.frag]{glsl}{shader.frag}

\subsection*{Teilaufgabe 10b}
TODO

\subsection*{Teilaufgabe 10c}
TODO

\section*{Aufgabe 11: Bézierkurven}
\subsection*{Teilaufgabe 11a}
\begin{itemize}
    \item Falsch: Nur die Endpunkte werden interpoliert
    \item Falsch: Eine Bezierkurve vom Grad $N$ hat $N+1$ Kontrollpunkte
    \item Falsch: Sie bilden eine Basis des Polynomraums $\mathbb{R}[X]$
    \item Richtig, da Bézierkurven eine Basis des Polynomraums sind.
\end{itemize}

\subsection*{Teilaufgabe 11b}
\begin{itemize}
    \item Nein: Konvexe Hülle der Kontrollpunkte
    \item Ja
    \item Ja
    \item Nein: Variationsreduktion
\end{itemize}

\end{document}
