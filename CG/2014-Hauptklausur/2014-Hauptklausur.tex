\documentclass[a4paper]{scrartcl}
\usepackage[ngerman]{babel}
\usepackage[utf8]{inputenc}
\usepackage{amssymb,amsmath}
\usepackage{graphicx}
\usepackage[inline]{enumitem}
\setlist{noitemsep}
\usepackage[binary-units=true]{siunitx}
\usepackage{hyperref}
\usepackage{parskip}
\usepackage[nameinlink,noabbrev,ngerman]{cleveref} % has to be after hyperref
\usepackage[colorinlistoftodos]{todonotes}
\usepackage{nicefrac}
\usepackage{csquotes}
\usepackage{booktabs}  % for \toprule, \midrule and \bottomrule

\usepackage{minted} % needed for the inclusion of source code

\usepackage{xcolor}
\definecolor{magenta}{HTML}{FF00FF}

\setcounter{secnumdepth}{2}
\setcounter{tocdepth}{2}

\usepackage{wasysym}
\usepackage{microtype}

\begin{document}
\selectlanguage{ngerman}
\title{2014 Hauptklausur (WS 2013/14)}

\setcounter{section}{1}
%%%%%%%%%%%%%%%%%%%%%%%%%%%%%%%%%%%%%%%%%%%%%%%%%%%%%%%%%%%%%%%%%%%%%%%%%%%%%%
\section*{Aufgabe 1: Farben und Farbwahrnehmung}
\subsection*{Teilaufgabe 1a: Chromatizitätsdiagramm}
\begin{figure}[h]
    \centering
    \includegraphics*[width=0.8\linewidth, keepaspectratio]{1a.png}
    \caption{Aufgabe 1a}
    \label{fig:1a}
\end{figure}

\subsection*{Teilaufgabe 1b}
Alles auf der Purple line. Also insbesondere \textcolor{magenta}{\textbf{Magenta}}.

\subsection*{Teilaufgabe 1c}
\begin{align}
    x &= \frac{X}{X + Y + Z}\\
    y &= \frac{Y}{X + Y + Z}
\end{align}

\subsection*{Teilaufgabe 1d}
$(2) < (3) < (1)$, also\\
RGB $<$ Raum aller Farben die durch 100 monochromatische Leuchtdioden darstellbar sind $<$ XYZ

\subsection*{Teilaufgabe 1e}
\begin{table}
    \begin{tabular}{p{6cm}ccp{5cm}}\toprule
    Aussage  & Wahr & Falsch & Begründung \\\midrule
    Den Weißpunkt eines Farbraums bezeichnet man auch als Tristimulus\-wert. & $\square$ & \CheckedBox & Die RGB-Werte sind die Tristimulus-Werte. Der Weißpunkt heißt pblicherweise $D[\text{Zahl}]$, wobei die Zahl die Temperatur angibt. D65 hat eine Farbtemperatur von ca. 6504K.\\
    Die subjektiv empfundene Stärke von Sinneseindrücken ist proportional zum Logarithmus ihrer Intensität. & \CheckedBox &  $\square$    & ~          \\
    Jeder Farbeindruck für den Menschen kann mit drei Grundgrößen beschrieben werden. & $\square$ & \CheckedBox & vgl. 1 (b) \\\bottomrule
    \end{tabular}
\end{table}

\section*{Aufgabe 2: Whitted-Style Raytracing}
\subsection*{Teilaufgabe 2a-d}
Siehe \cref{fig:2a}.

\begin{figure}[h]
    \centering
    \includegraphics*[width=0.8\linewidth, keepaspectratio]{2a.png}
    \caption{Aufgabe 2a-d; $n_1 = 1, n_2 = 1.5$}
    \label{fig:2a}
\end{figure}

\subsection*{Teilaufgabe 2e}
\begin{align}
    \eta_i \sin \theta_i &= \eta_t \sin \theta_t\\
    1 \cdot \frac{4}{10} &= 1.5 \sin \theta_t\\
    \Leftrightarrow \sin \theta_t = \frac{4}{15} = \frac{2}{7.5}
\end{align}

\subsection*{Teilaufgabe 2f}
\begin{align}
    I_s    &= k_s \cdot I_L \cdot \cos^n \alpha\\
    \alpha &= r_L \cdot v
\end{align}

wobei $k_s$ ein Materialparameter und $I_L$ die intensität der Lichtquelle ist.
$n$ wird der Phong-Exponent genannt (TODO: woher kommt der?)

\subsection*{Teilaufgabe 2g}
Snellsches Brechungsgesetz

\[\eta_i \sin \theta_i = \eta_t \sin \theta_t\]

\section*{Aufgabe 3: Transformationen}
\[\begin{pmatrix}s_x & h_x & t_x\\h_y & s_y & t_y\\a & b & c\end{pmatrix}\]


\begin{itemize}
    \item Die Parameter $s_x, s_y$ skalieren in Richtung der $x$ bzw. $y$
          Achse.
    \item Die Parameter $h_x, h_y$ scheeren in Richtung der $x$ bzw. $y$ Achse.
    \item Die Parameter $t_x, t_y$ füren eine Translation in $x$ bzw. $y$
          Richtung aus.
    \item Die Parameter $a, b, c$ skalieren.
\end{itemize}


Die Matrix

\[\begin{pmatrix}\cos \theta & -\sin \theta & 0\\\sin \theta & \cos \theta & 0\\0 & 0 & 1\end{pmatrix}\]

rotiert um $\theta$ um den Ursprung (gegen den Uhrzeigersinn.)

\begin{itemize}
    \item Bild 1: Translation um 1 in $x$ und 3 in $y$-Richtung.
    \item Bild 2: Scherung im $-2$ in $y$-Richtung.
    \item Bild 3: Rotation um $45^\circ$ gegen den Urzeigersinn.
    \item Bild 4: In $x$-Richtung um $\nicefrac{1}{2}$ stauchen, in $y$-Richtung
                  um 3 Strecken und dann um 4 nach rechts verschieben.
    \item Bild 5: Projektion auf die zur $x$-Achse parallele Gerade durch $(0, 3)$.
\end{itemize}


\section*{Aufgabe 4}
\subsection*{Teilaufgabe 4a}
TODO
\subsection*{Teilaufgabe 4b}
TODO
\subsection*{Teilaufgabe 4c}
\subsubsection*{Teilaufgabe 4c (I)}
TODO
\subsubsection*{Teilaufgabe 4c (II)}
TODO
\subsubsection*{Teilaufgabe 4c (III)}
TODO

\subsection*{Teilaufgabe 4d}
\begin{tabular}{p{6cm}ccp{5cm}}\toprule
Aussage  & Wahr & Falsch & Begründung \\\midrule
Texturkoordinaten müssen sich immer im Intervall $[0; 1]$ befinden. & $\square$ & $\square$ & ~ \\
Texturkoordinaten können als Attribute der Eckpunkte (Vertizes) übergeben werden und werden als solche interpoliert.  & $\square$    & $\square$      & ~          \\
Texturkoordinaten müssen für die Darstellung wie Eckpunktkoordinaten der Model-View-Transformation unterzogen werden. & $\square$    & $\square$      & ~          \\\bottomrule
\end{tabular}


\section*{Aufgabe 5: Vorgefilterte Environment-Maps}
\subsection*{Teilaufgabe 5a}
TODO
\subsection*{Teilaufgabe 5b}
TODO

\section*{Aufgabe 6: Hierarchische Datenstrukturen}
\subsection*{Teilaufgabe 6a}
\includegraphics*[width=0.8\linewidth, keepaspectratio]{6a.png}

\subsection*{Teilaufgabe 6b}
Inklusive Schnittests der AABB Hüllkörper:

\begin{enumerate}
    \item 1
    \item 1.1
    \item 1.1.1
    \item 5, 6
    \item 1.1.2
    \item 1.2
    \item 1.2.1
    \item 3, 7
    \item 1.2.2
    \item 4, 8
\end{enumerate}

\subsection*{Teilaufgabe 6c}
\begin{tabular}{p{8cm}ccp{5cm}}\toprule
Aussage & Wahr & Falsch & Begründung \\\midrule
Beim Traversieren eines kD-Baums müssen immer beide Kinder in Betracht gezogen werden. & $\square$  & \CheckedBox & vgl. Folie 103\\
Das Traversieren einer Hüllkörperhierarchie mit achsenparallelen Boxen (Bounding Volume Hierarchy, BVH) erfordert Mailboxing, um mehrfache Schnitttests mit einem Dreieck zu verhindern. & \CheckedBox & $\square$     & ~          \\
Der Speicheraufwand einer BVH hängt logarithmisch von der Anzahl der Primitive ab. & \CheckedBox & $\square$      & ~          \\
kD-Bäume sind eine Verallgemeinerung von BSP-Bäumen. & $\square$    & \CheckedBox & Es ist genau anders herum. kD-Bäume müssen Achsenparallele Trennebenen haben, BSP-Bäume jedoch nicht. \\
BSP-Bäume sind adaptiv und leiden nicht unter dem \enquote{Teapot in a Stadium}-Problem. & \CheckedBox    & $\square$      & ~          \\
\end{tabular}

\subsection*{Teilaufgabe 6d}
\begin{tabular}{p{8cm}cccc}\toprule
Aussage  & BVH & Octree & kD-Baum & Gitter \\\midrule
Die Datenstruktur partitioniert den Raum. & $\square$   & \CheckedBox & \CheckedBox & \CheckedBox \\
Der Aufwand für den Aufbau der Datenstruktur ist linear in der Anzahl der Primitive. & $\square$ & $\square$ & $\square$ & $\square$ \\
Eine effizientere Traversierung wird erreicht, wenn die Surface Area Heuristic bei der Konstruktion verwendet wird. & $\square$ & $\square$ & \CheckedBox & $\square$ \\
Die Datenstruktur eignet sich am besten für Szenen, in denen die Geometrie gleichmäßig verteilt ist und kaum leere Zwischenräume vorhanden sind. & $\square$ & $\square$ & $\square$ & \CheckedBox \\\bottomrule
\end{tabular}

\section*{Aufgabe 7}
TODO

\section*{Aufgabe 8}
TODO

\section*{Aufgabe 9}
TODO

\section*{Aufgabe 10}
TODO

\section*{Aufgabe 11: Wasseroberfläche mit GLSL}
\subsection*{Teilaufgabe 11a}
\inputminted[linenos, numbersep=5pt, tabsize=4, frame=lines, label=shader.frag]{glsl}{shader.frag}

\subsection*{Teilaufgabe 11b}
TODO

\end{document}
