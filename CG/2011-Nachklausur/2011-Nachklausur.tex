\documentclass[a4paper]{scrartcl}
\usepackage[ngerman]{babel}
\usepackage[utf8]{inputenc}
\usepackage{amssymb,amsmath}
\usepackage{graphicx}
\usepackage[inline]{enumitem}
\setlist{noitemsep}
\usepackage[binary-units=true]{siunitx}
\usepackage{hyperref}
\usepackage{parskip}
\usepackage[nameinlink,noabbrev,ngerman]{cleveref} % has to be after hyperref
\usepackage[colorinlistoftodos]{todonotes}
\usepackage{nicefrac}  % for \nicefrac{1}{3}
\usepackage{csquotes}  % for \enquote{what you want to quote}
\usepackage{booktabs}  % for \toprule, \midrule and \bottomrule
\usepackage{minted} % needed for the inclusion of source code

% for \begin{enumerate}[label=(\Alph*)], see http://tex.stackexchange.com/a/129960/5645
\usepackage{enumitem}

\setcounter{secnumdepth}{2}
\setcounter{tocdepth}{2}

\usepackage{microtype}

% \begin{figure}[h]
%     \centering
%     \includegraphics*[width=0.8\linewidth, keepaspectratio]{1a.png}
%     \caption{Whatever}
%     \label{fig:1a}
% \end{figure}

\begin{document}
\selectlanguage{ngerman}
\title{2011 Nachklausur (WS 2010/11)}

\setcounter{section}{1}
%%%%%%%%%%%%%%%%%%%%%%%%%%%%%%%%%%%%%%%%%%%%%%%%%%%%%%%%%%%%%%%%%%%%%%%%%%%%%%
\section*{Aufgabe 1: Wahrnehmung, Farbe und Rasterbilder}
\subsection*{Teilaufgabe 1a}
\textit{Was versteht man unter Metamerie beim Farbsehen des Menschen?}

Metamerie ist das Phänomen, dass verschiedene Spektren den selben Farbeindruck
erzeugen können.

\subsection*{Teilaufgabe 1b}
\textit{Was versteht man unter Schwarzkörperstrahlung und Farbtemperatur?}

Ein Schwarzkörper ist eine idealisierte thermische Strahlungsquelle. Die
idealisierung besteht darin, dass der Körper die komplette auftretende
Strahlung vollständig absorbiert. Gleichzeitig sendet er Wärmestrahlung
(Schwarzkörperstrahlung) aus, welche nur von seiner Temperatur abhängig ist.

Die Farbtemperatur ist ein Maß, um einen jeweiligen Farbeindruck einer
Lichtquelle zu bestimmen.

\subsection*{Teilaufgabe 1c}
Siehe \href{https://martin-thoma.com/html5/graphic-filters/graphic-filters.htm}{martin-thoma.com/html5/graphic-filters} zum ausprobieren.

\begin{enumerate}[label=(\Alph*)]
    \item Hervorheben von horizontalen Kanten.
    \item Unschärfe / Weichzeichnen
    \item Hervorheben aller Kanten
    \item Hervorheben aller Kanten (Invertierter Laplace-Filter), entfernen vom
          Rest
\end{enumerate}

\subsection*{Teilaufgabe 1d}
\textit{Was versteht man unter einem normalisierten Filterkernel?}

Ein normalisierter Filterkernel hat als Summe der Element den Wert~1.

\textit{Welche globale Eigenschaft eines Bildes ändert sich, wenn ein Filterkernel nicht normalisiert ist?}

Die Helligkeit des Bildes ändert sich nicht.


\section*{Aufgabe 2}
TODO

\section*{Aufgabe 3}
TODO

\section*{Aufgabe 4}
TODO

\section*{Aufgabe 5}
TODO

\section*{Aufgabe 6}
TODO

\section*{Aufgabe 7}
TODO

\section*{Aufgabe 8}
TODO

\section*{Aufgabe 9}
TODO

% \inputminted[linenos, numbersep=5pt, tabsize=4, frame=lines, label=shader.frag]{glsl}{shader.frag}


\end{document}
