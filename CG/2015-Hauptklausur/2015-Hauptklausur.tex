\documentclass[a4paper]{scrartcl}
\usepackage[ngerman]{babel}
\usepackage[utf8]{inputenc}
\usepackage{amssymb,amsmath}
\usepackage{graphicx}
\usepackage[inline]{enumitem}
\setlist{noitemsep}
\usepackage[binary-units=true]{siunitx}
\usepackage{hyperref}
\usepackage{parskip}
\usepackage[nameinlink,noabbrev,ngerman]{cleveref} % has to be after hyperref
\usepackage[colorinlistoftodos]{todonotes}
\usepackage{nicefrac}
\usepackage{csquotes}
\usepackage{booktabs}  % for \toprule, \midrule and \bottomrule

\usepackage{minted} % needed for the inclusion of source code

\setcounter{secnumdepth}{2}
\setcounter{tocdepth}{2}

\usepackage{microtype}

\begin{document}
\selectlanguage{ngerman}
\title{2015 Hauptklausur (WS 2014/15)}

\setcounter{section}{1}
%%%%%%%%%%%%%%%%%%%%%%%%%%%%%%%%%%%%%%%%%%%%%%%%%%%%%%%%%%%%%%%%%%%%%%%%%%%%%%
\section*{Aufgabe 1: Raytracing}
\subsection*{Teilaufgabe 1a}
TODO
\subsection*{Teilaufgabe 1b}
TODO
\subsection*{Teilaufgabe 1c}
TODO
\subsection*{Teilaufgabe 1d}
\begin{enumerate}
    \item \textbf{Ray generation}: Erzeuge Sichtstrahlen durch jeden Pixel.
    \item \textbf{Ray intersection}: Schnittberechnung; also: Finde Objekt
          welches den Strahl schneidet und am nahesten zur Kamera ist.
    \item \textbf{Shading}: Schattierung / Beleuchtungsberechnung.
\end{enumerate}

\section*{Aufgabe 2: Farben}
\subsection*{Teilaufgabe 2a}
\textit{Wie nennt man die Funktionen, mit denen man Tristimulus-Werte zu einem gegebenen Spektrum berechnen kann?}

Color Matching Funktionen?

\subsection*{Teilaufgabe 2b}
\begin{itemize}
    \item \textit{Es gibt eine lineare Abbildung zwischen den Farbräumen XYZ und xyY.}
    \item[$\rightarrow$] TODO
    \item \textit{Es gibt eine lineare Abbildung zwischen den Farbräumen RGB und XYZ.}
    \item[$\rightarrow$] TODO
    \item \textit{Die subjektiv empfundene Stärke von Sinneseindrücken ist proportional zur Intensität des physikalischen Reizes.}
    \item[$\rightarrow$] TODO
\end{itemize}


\subsection*{Teilaufgabe 2c}
\textit{Welche Information beinhalten die x- und y-Komponenten einer Farbdarstellung im CIE-xyY-Farbraum zusammengenommen?}
TODO

\subsection*{Teilaufgabe 2d}
TODO

\section*{Aufgabe 3: Homogene Koordinaten}
\subsection*{Teilaufgabe 3a}
TODO
\subsection*{Teilaufgabe 3b}
TODO
\subsection*{Teilaufgabe 3c}
TODO

\section*{Aufgabe 4: Transformationen}
\subsection*{Teilaufgabe 4a}
TODO
\subsection*{Teilaufgabe 4b}
TODO

\section*{Aufgabe 5: Beschleunigungsstrukturen und Hüllkörper}
\subsection*{Teilaufgabe 5a}
TODO
\subsection*{Teilaufgabe 5b}
TODO
\subsection*{Teilaufgabe 5c}
TODO

\section*{Aufgabe 6: Texturen}
\subsection*{Teilaufgabe 6a}
TODO
\subsection*{Teilaufgabe 6b}
TODO

\section*{Aufgabe 7: Beleuchtung}
\subsection*{Teilaufgabe 7a}
TODO
\subsection*{Teilaufgabe 7b}
TODO
\subsection*{Teilaufgabe 7c}
TODO
\subsection*{Teilaufgabe 7d}
TODO
\subsection*{Teilaufgabe 7e}
TODO
\subsection*{Teilaufgabe 7f}
TODO

\section*{Aufgabe 8: Partikeleffekte und OpenGL-Blending}
\subsection*{Teilaufgabe 8a}
TODO
\subsection*{Teilaufgabe 8b}
TODO
\subsection*{Teilaufgabe 8c}
TODO
\subsection*{Teilaufgabe 8d}
TODO

\section*{Aufgabe 9: OpenGL}
TODO

\section*{Aufgabe 10: Reflexionen in OpenGL}
\inputminted[linenos, numbersep=5pt, tabsize=4, frame=lines, label=shader.frag]{glsl}{shader.frag}

\section*{Aufgabe 11: GLSL-Hatching}
\subsection*{Teilaufgabe 11a}

\[
	\left(1 - \frac{x - x_1}{x_2 - x_1}\right) y_1 + \frac{x - x_1}{x_2 - x_1} y_2
\]

oder

\[
	y_1 + \frac{y_2 - y_1}{x_2 - x_1} (x - x_1)
\]

\subsection*{Teilaufgabe 11b}
\inputminted[linenos, numbersep=5pt, tabsize=4, frame=lines, label=shader.frag]{glsl}{hatching.frag}

\section*{Aufgabe 12: Bézierkurven}
\subsection*{Teilaufgabe 12a}
TODO
\subsection*{Teilaufgabe 12b}
TODO
\subsection*{Teilaufgabe 12c}
\begin{enumerate}
    \item Ja
    \item Ja
    \item Nein, da die Kurve nicht innerhalb der konvexen Hülle der
          Kontrollpunkte ist.
\end{enumerate}

\end{document}
