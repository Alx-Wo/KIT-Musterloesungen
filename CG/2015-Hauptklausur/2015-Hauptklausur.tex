\documentclass[a4paper]{scrartcl}
\usepackage[ngerman]{babel}
\usepackage[utf8]{inputenc}
\usepackage{amssymb,amsmath}
\usepackage{graphicx}
\usepackage[inline]{enumitem}
\setlist{noitemsep}
\usepackage[binary-units=true]{siunitx}
\usepackage{hyperref}
\usepackage{parskip}
\usepackage[nameinlink,noabbrev,ngerman]{cleveref} % has to be after hyperref
\usepackage{nicefrac}  % for \nicefrac{1}{3}
\usepackage{csquotes}  % for \enquote{what you want to quote}
\usepackage{booktabs}  % for \toprule, \midrule and \bottomrule
\usepackage{minted} % needed for the inclusion of source code

% for \begin{enumerate}[label=(\Alph*)], see http://tex.stackexchange.com/a/129960/5645
\usepackage{enumitem}

\setcounter{secnumdepth}{2}
\setcounter{tocdepth}{2}

\usepackage{wasysym}  % For \CheckedBox
\usepackage{microtype}

\begin{document}
\selectlanguage{ngerman}
\title{2015 Hauptklausur (WS 2014/15)}

\setcounter{section}{1}
%%%%%%%%%%%%%%%%%%%%%%%%%%%%%%%%%%%%%%%%%%%%%%%%%%%%%%%%%%%%%%%%%%%%%%%%%%%%%%
\section*{Aufgabe 1: Raytracing}
\subsection*{Teilaufgabe 1a}
\textit{Zeichnen Sie in der folgenden Szene mit Liniensegmenten einen möglichst einfachen
Lichtpfad von der Kamera zur Punktlichtquelle ein, der zwar in der geometrischen
Optik auftreten kann, jedoch nicht mit Whitted-Style Raytracing erzeugt werden kann!}
\includegraphics*[width=\linewidth, keepaspectratio]{1a.jpg}

\subsection*{Teilaufgabe 1b}
\textit{Begründen Sie kurz, warum der Pfad nicht mit Whitted-Style Raytracing
erzeugt werden kann!}

Whitted-Style sendet nur von Kamera Strahlen aus und verfolgt Schatten-,
Transmission und Reflektionsstrahlen als Sekundärstrahlen. Da aber von der
Kamera aus verfolgt wird, kann (in diesem Fall) kein Strahl die Rückseite (aus
Sicht der Kamera) der diffusen Fläche treffen $\Rightarrow$ die indirekte
Beleuchtung die durch diffuses Abstrahlen von der Rückseite der diffusen Fläche
erfolgt, kann mit Whitted-Style Raytracing nicht verfolgt werden.

\subsection*{Teilaufgabe 1c}
\textit{In der folgenden Abbildung sind zwei Primärstrahlen gegeben. Führen Sie
mit diesen zeichnerisch Whitted-Style Raytracing durch! Kennzeichnen Sie
Reflexionsstrahlen mit \enquote{R}, Transmissionsstrahlen mit \enquote{T} und
Schattenstrahlen mit \enquote{S}!}

\includegraphics*[width=\linewidth, keepaspectratio]{1c.jpg}

\subsection*{Teilaufgabe 1d}
\begin{enumerate}
    \item \textbf{Ray generation}: Erzeuge Sichtstrahlen durch jeden Pixel.
    \item \textbf{Ray intersection}: Schnittberechnung; also: Finde Objekt
          welches den Strahl schneidet und am nahesten zur Kamera ist.
    \item \textbf{Shading}: Schattierung / Beleuchtungsberechnung.
\end{enumerate}

\section*{Aufgabe 2: Farben}
\subsection*{Teilaufgabe 2a}
\textit{Wie nennt man die Funktionen, mit denen man Tristimulus-Werte zu einem gegebenen Spektrum berechnen kann?}

Color Matching Funktionen

\subsection*{Teilaufgabe 2b}
\begin{itemize}
    \item \textit{Es gibt eine lineare Abbildung zwischen den Farbräumen XYZ und xyY.}
    \item[$\rightarrow$] Falsch, da bei der Umrechnung durch $X + Y + Z$ dividiert wird ($x = \frac{X}{X+Y+Z}$, $y = \frac{y}{X+Y+Z}$)
    \item \textit{Es gibt eine lineare Abbildung zwischen den Farbräumen RGB und XYZ.}
    \item[$\rightarrow$] Korrekt, der Farbraum wurden konstruiert, um eine möglichst einfache Umrechnung zu ermöglichen.
    \item \textit{Die subjektiv empfundene Stärke von Sinneseindrücken ist proportional zur Intensität des physikalischen Reizes.}
    \item[$\rightarrow$] Falsch, es besteht ein logarithmischer Zusammenhang.
\end{itemize}

\subsection*{Teilaufgabe 2c}
\textit{Welche Information beinhalten die $x$- und $y$-Komponenten einer Farbdarstellung im CIE-$xyY$-Farbraum zusammengenommen?}

Sie beschreiben die Farbe (Chromatizität) unabhängig von der Intensität.

\subsection*{Teilaufgabe 2d}
\textit{Ein RGB-Eingabebild wird in den CIE-$xyY$-Farbraum transformiert. Um Speicher zu
sparen werden die $x$- und $y$-Komponenten in geringerer Auflösung gespeichert, während
die Auflösung der $Y$-Komponente beibehalten wird.}

\textit{Warum ist dieses Vorgehen im Hinblick auf einen menschlichen Betrachter deutlich
besser, als die Auflösung der RGB-Komponenten zu reduzieren?}

Die Kontrastsentitivität für Chrominanz (Farbe) ist vor allem im hochfrequenten Bereich deutlich geringer als für Luminanz, daher kann man den Chrominanz-Anteil (xy-Werte) in geringerer Auflösung speichern (was dem Weglassen hoher Frequenzen entspricht), ohne dass mit den Augen ein großer Unterschied erkennbar ist.

\section*{Aufgabe 3: Homogene Koordinaten}
\textit{Beim Raytracing soll ein in Weltkoordinaten definierter Strahl mit einem Objekt in dessen
lokalen Koordinatensystem geschnitten werden. Die affine Transformation von lokalen in
Weltkoordinaten ist durch die homogene Transformationsmatrix $M \in \mathbb{R}^{4\times 4}$ gegeben.}
\subsection*{Teilaufgabe 3a}
\textit{Mit welcher Matrix kann der Strahl von Weltkoordinaten in lokale Koordinaten trans-
formiert werden?}

Mit der Matrix $M^{-1}$, da diese die inverse Transformation von Punkten aus \textit{lokalen} in \textit{Weltkoordinaten} darstellt.
\subsection*{Teilaufgabe 3b}
\textit{Der Strahl $\mathbf{R}(t) = \mathbf{E} + t \mathbf{d}, (\mathbf{E}, \mathbf{d} \in \mathbb{R}^3 )$ soll mit der Matrix aus der Teilaufgabe a) in das
lokale Koordinatensystem transformiert werden. Geben Sie dazu den transformierten
Punkt $\mathbf{E'}$ und die transformierte Richtung $\textbf{d'}$ an!}

Zur Vereinfachung geht man zu erweiterten Koordinaten $\begin{pmatrix}
\mathbf{E}\\1
\end{pmatrix}$ und $\begin{pmatrix}
\mathbf{d}\\0
\end{pmatrix}$ über (1 für Punkte, 0 für Richtungen, da diese Differenzen zweier Punkte darstellen).

Damit erhält man $\begin{pmatrix}
\mathbf{E'}\\1
\end{pmatrix} = M^{-1} \begin{pmatrix}
\mathbf{E}\\1
\end{pmatrix}$ und $\begin{pmatrix}
\mathbf{d'}\\0
\end{pmatrix} = M^{-1} \begin{pmatrix}
\mathbf{d}\\0
\end{pmatrix}$
\subsection*{Teilaufgabe 3c}
\textit{Gegeben sei nun}
$$M=\begin{pmatrix}0&0 &-1&5&\\
1&0 &0&0\\
0&-1&0 &0\\
0&0 &0 &1\\
\end{pmatrix}$$
\textit{Berechnen Sie für diese Werte die Matrix aus Aufgabe a)!}

Invertiere die Matrix: $$\left(\begin{array}{cccc|cccc}
0&0 &-1&5&1&0&0&0\\
1&0 &0&0&0&1&0&0\\
0&-1&0 &0&0&0&1&0\\
0&0 &0 &1&0&0&0&1\\
\end{array}\right)\longrightarrow\cdots\longrightarrow\left(\begin{array}{cccc|cccc}
1&0 &0&0&0&1&0&0\\
0&1&0 &0&0&0&-1&0\\
0&0 &1&0&-1&0&0&5\\
0&0 &0 &1&0&0&0&1\\
\end{array}\right)$$

Allgemein gilt für eine Matrix der Form $M = \begin{pmatrix}
A&t\\0&1\end{pmatrix}$: $M^{-1} = \begin{pmatrix}
A^{-1}&-A^{-1}t\\0&1
\end{pmatrix}$

\clearpage
\section*{Aufgabe 4: Transformationen}
\subsection*{Teilaufgabe 4a}
\textit{Gegeben sei folgendes Liniensegment $(\mathbf{p}_0 , \mathbf{p}_1)$ im
Clip-Space:}

\[\mathbf{p}_0 = \begin{pmatrix}-1\\1\\5\\10\end{pmatrix},\;\;\;
  \mathbf{p}_1 = \begin{pmatrix}0.5\\-0.5\\15\\5\end{pmatrix}\]

\subsubsection*{Teilaufgabe 4a (i)}
\textit{Wie lauten die Normalized Device Coordinates der Punkte $\mathbf{p}_0$
und $\mathbf{p}_0$?}

TODO

\subsubsection*{Teilaufgabe 4a (ii)}
\textit{Warum sollte man dieses Liniensegment vor der Normalisierungstransformation
(Dehomogenisierung) clippen?}

TODO

\subsection*{Teilaufgabe 4b}
\textit{Die folgende Abbildung zeigt ein Quadrat in 2D vor (links) und nach einer Scherung
(rechts).}

\includegraphics*[width=0.5\linewidth, keepaspectratio]{4b.png}

\begin{itemize}
\item[i)] \textit{Geben Sie die Transformationsmatrix $M\in \mathbb{R}^{2\times 2}$ an, die die Scherung beschreibt!}

Die Scherung hat zur Folge, dass beim Erhöhen der x-Koordinate um eine Einheit die y-Koordinate um eine Einheit verringert wird. Damit hat die Matrix die Form
$$M=\begin{pmatrix}
1&0\\-1&1
\end{pmatrix}$$
\item[ii)] \textit{Wie berechnet man allgemein die Matrix $N \in \mathbb{R}^{2\times 2}$ an, die zur Transformation
der Normalen verwendet werden muss? Bestimmen Sie diese Matrix!}

$$N = (M^{-1})^\top = \left(\begin{pmatrix}
1&0\\-1&1
\end{pmatrix}^{-1}\right)^{\top} = \begin{pmatrix}
1&0\\1&1
\end{pmatrix}^{\top} = \begin{pmatrix}
1&1\\0&1
\end{pmatrix}$$

\item[iii)] \textit{Berechnen Sie mithilfe von $N$ die transformierte Normale $\mathbf{n'}$ zur in der Abbildung
eingezeichneten Normalen $\mathbf{n}$!}

$$\mathbf{n'} = \frac{N\mathbf{n}}{||N\mathbf{n}||} = \frac{1}{\sqrt{2}}\begin{pmatrix}
1\\1
\end{pmatrix}$$
\end{itemize}

\section*{Aufgabe 5: Beschleunigungsstrukturen und Hüllkörper}
\subsection*{Teilaufgabe 5a}
\includegraphics*[width=\linewidth, keepaspectratio]{5a.png}
\textit{Welche der beiden Hierarchien kann von einem Raytracer effizienter
durchlaufen werden? Begründen Sie dies in Stichpunkten!}

TODO

\subsection*{Teilaufgabe 5b}
\begin{tabular}{cp{8cm}llp{4cm}}\toprule
\# & Aussage                                                                                                                                     & Wahr           & Falsch           & Kommentar \\\midrule
1  & Ein kD-Baum kann für $N$ Primitive den Aufwand für Strahlschnitte von $O(N)$ auf $O(\log(N))$ reduzieren.                                   & \CheckedBox    & $\square$        & average case\\
2  & Die Surface Area Heuristic sorgt dafür, dass beiden Kindknoten gleich viele Primitive zugeteilt werden.                                     & $\square$      & \CheckedBox      & minimale Kosten\\
3  & Für die Suche der Trennebene (Split Plane) mit Objektmittel (Object Median) ist immer eine vollständige Sortierung der Primitive notwendig. & $\square$      & \CheckedBox      & Sortiert werden die Objektmittelpunkte\\
4  & Teilen eines Knotens am Objektmittel (Object Median) führt stets zu den effizientesten Hüllkörperhierarchien.                               & $\square$      & \CheckedBox      & SAH\\
5  & Die Surface Area Heuristic macht die Annahme, dass Strahlen stets außerhalb des zu unterteilenden Hüllkörpers starten.                      & TODO           & TODO             & \\
6  & Es gibt Szenen, in denen ein kD-Baum keinen Vorteil bringt.                                                                                 & \CheckedBox    & $\square$        & Identische Dreiecke\\\bottomrule
\end{tabular}

\subsection*{Teilaufgabe 5c}
TODO: Bitte kontrollieren!

\begin{tabular}{cp{10cm}lll}\toprule
\# & Aussage                                                                                                                                                                                       & AABB           & OBB           & Kugel \\\midrule
1  & Der Hüllkörper kann einen beliebig orientierten Würfel optimal, also ohne freien Raum umschliessen.                                                                                           &$\square$       & \CheckedBox   & $\square$     \\
2  & Es gibt orthonormale Transformationen des eingeschlossenen Objektes, die das Volumen des optimalen Hüllkörpers verändern.                                                                     & \CheckedBox    & $\square$     & $\square$     \\
3  & Es gibt affine Transformationen des  eingeschlossenen Objektes, die das Volumen des Hüllkörpers verändern.\footnotemark                                                                       & \CheckedBox    & \CheckedBox   & \CheckedBox   \\
4  & Gegegeben sind zwei Objektmengen und zu jeder Menge ihr optimaler Hüllkörper. Der Aufwand, den optimalen Hüllkörper für alle Objekte zu bestimmen, ist unabhängig von der Anzahl der Objekte. & \CheckedBox    & $\square$     & \CheckedBox     \\\bottomrule
\end{tabular}
\footnotetext{Skalierung. Allerdings gibt es nur für AABBs affine transformationen, welche $\frac{V_\text{Hüllkörper} - V_\text{Objekt}}{V_\text{Hüllkörper}}$ verändert.}

\section*{Aufgabe 6: Texturen}
\subsection*{Teilaufgabe 6a}

\begin{tabular}{cp{8cm}lll}\toprule
\#   & Aussage                                                                                                                                                                        & Wahr         & Falsch      & Kommentar \\\midrule
I1   & Das Filtern von Texturen wird üblicherweise bei der Minification und der Magnification angewandt, um Aliasing-Artefakte zu vermeiden                                           & $\square$    & \CheckedBox & Nur Magnification \\
I2   & Das Filtern von Texturen benötigt bei Verfahren ohne Vorfilterung mehr Zugriffe auf den Texturspeicher als einfaches Auslesen.                                                 & TODO         & TODO        & ~         \\
II1  & Mipmapping bekämpft das Aliasing-Problem bei der Unterabtastung von Texturen.                                                                                                  & \CheckedBox  & $\square$   & ~         \\
II2  & Mipmapping berücksichtigt anisotrope Footprints.                                                                                                                               & TODO         & TODO        & ~         \\
III1 & Bei Einsatz von Summed-Area-Tables kann man mit nur vier Texturzugriffen die Summe über eine kreisförmige Fläche berechnen.                                                   & $\square$    & \CheckedBox & ~         \\
III2 & Bei Einsatz von Summed-Area-Tables benötigt man im Allgemeinen mehr Bits zur Repräsentation der enthaltenen Werte als bei gewöhnlichen Texturen mit dem gleichen Wertebereich. & \CheckedBox  & $\square$   & ~         \\\bottomrule
\end{tabular}

\clearpage
\subsection*{Teilaufgabe 6b}
\textit{Eine Textur wird mit einem trilinearen Filter abgetastet. Die folgende Abbildung zeigt
den Pixel-Footprint im $(s, t)$-Texturraum $(s, t \in [0, 1])$ und rechts
daneben die Textur:}

\includegraphics*[width=0.8\linewidth, keepaspectratio]{6b.png}

\textit{Welche Mipmap-Stufen werden für die trilineare Interpolation verwendet, wenn die
Textur eine Auflösung von $2048\times1024$ hat? (Mipmap-Stufe 0 hat die größte Auflösung).
Begründen Sie Ihre Antwort!}

Texturgröße:
\begin{align}
    0.005 \cdot 2048 &= 10.24\\
    0.01 \cdot 1024 &= 10.24\\
    \Rightarrow 10.24 \cdot 10.24 &= 104.8576
\end{align}

Stufen:
\begin{enumerate}
    \item $1024 \times 512$
    \item $512 \times 256$
    \item $256 \times 128$
    \item $128 \times 64$
    \item $64 \times 32$
    \item $32 \times 16$
    \item $16 \times 8 = 128$
    \item $8 \times 4 = 32$
\end{enumerate}

Es werden also die Stufen 7 und 8 verwendet.

\clearpage
\section*{Aufgabe 7: Beleuchtung}
\subsection*{Teilaufgabe 7a}
$\mathbf{P} = \lambda_1 \mathbf{P}_1 + \lambda_2 \mathbf{P}_2 + \lambda_3 \mathbf{P}_3$, wobei $\lambda_1 + \lambda_2 + \lambda_3 = 1$
und $\lambda_1, \lambda_2, \lambda_3 \geq 0$

\subsection*{Teilaufgabe 7b}
\begin{itemize}
    \item Berechne $I_i = k_d I_L (\mathbf{n}_i \cdot \text{normalize}(\mathbf{L}-\mathbf{P}_i))$ für $i = 1,2,3$
    \item Interpoliere $I = \lambda_1 I_1 + \lambda_2 I_2 + \lambda_3 I_3$
    \item Bemerkung: $\text{normalize}(\mathbf{x}) = \frac{\mathbf{x}}{||\mathbf{x}||}$
\end{itemize}

\subsection*{Teilaufgabe 7c}
\begin{itemize}
    \item Interpoliere $\mathbf{n} = \text{normalize}(\lambda_1 \mathbf{n}_1 + \lambda_2 \mathbf{n}_2 + \lambda_3 \mathbf{n}_3)$
    \item Berechne $I = k_d I_L (\mathbf{n} \cdot \text{normalize}(\mathbf{L}-\mathbf{P}))$
\end{itemize}

\subsection*{Teilaufgabe 7d}
In den Eckpunkten. Allgemein gilt im $i$-ten Eckpunkt: $\lambda_i = 1$ sowie $\lambda_j = 0 \, (\forall j \neq i)$.
Somit sind die Berechnungen in diesen Punkten äquivalent, wie man leicht aus den obrigen Formeln entnehmen kann.

\subsection*{Teilaufgabe 7e}
Im rechten Fall wird mehr Licht zum betrachter reflektiert.

\subsection*{Teilaufgabe 7f}
Fresnel-Effekt

\clearpage
\section*{Aufgabe 8: Partikeleffekte und OpenGL-Blending}
\subsection*{Teilaufgabe 8a}
\textit{Zunächst werden nur Feuerpartikel gerendert. Konfigurieren Sie OpenGL
so, dass die Feuerpartikel Licht in Richtung des Betrachters emittieren!}

\inputminted[linenos, numbersep=5pt, tabsize=4, frame=lines, label=8a.cpp]{cpp}{8a.cpp}

\subsection*{Teilaufgabe 8b}
\textit{Warum ist der Tiefentest aktiviert? Warum ist Schreiben in den
Tiefenpuffer deaktiviert?}

TODO

\subsection*{Teilaufgabe 8c}
\textit{Zunächst werden nur Feuerpartikel gerendert. Konfigurieren Sie OpenGL
so, dass die Feuerpartikel Licht in Richtung des Betrachters emittieren!}

\inputminted[linenos, numbersep=5pt, tabsize=4, frame=lines, label=8a.cpp]{cpp}{8c.cpp}
\clearpage
\subsection*{Teilaufgabe 8d}
\textit{Ihnen fällt auf, dass sich Rauch und Feuer nicht korrekt zusammenfügen, wenn Sie erst
Feuerpartikel wie in Aufgabe a) und dann Rauchpartikel wie in Aufgabe c) rendern.
Welche Schritte müssen Sie vornehmen, um das Problem zu beheben?}

Den Renderablauf verändern:\\
\begin{tabular}{cl}
 $\square$ \CheckedBox TODO & Zuerst die Rauchpartikel zeichnen, dann die Feuerpartikel.       \\
 $\square$ \CheckedBox TODO & \texttt{glDepthFunc(GL\_GREATER)}: Zeichenreihenfolge umkehren.           \\
 $\square$ \CheckedBox TODO & Rauchpartikel front-to-back sortieren.                           \\
 $\square$ \CheckedBox TODO & Feuerpartikel und Rauchpartikel zusammen sortieren und zeichnen. \\
\end{tabular}

OpenGL Blending-Konfigurationen:\\
\begin{tabular}{cl}
 $\square$ \CheckedBox TODO & \verb+glBlendFunc(GL_SRC_ALPHA, GL_ONE_MINUS_SRC_ALPHA)+\\
 $\square$ \CheckedBox TODO & \verb+glBlendFunc(GL_FUNC_ADD, GL_ONE)+\\
 $\square$ \CheckedBox TODO & \verb+glBlendFunc(GL_ONE_MINUS_DST_ALPHA, GL_SRC_ALPHA)+\\
 $\square$ \CheckedBox TODO & \verb+glBlendFunc(GL_ONE, GL_ONE_MINUS_SRC_ALPHA)+\\
\end{tabular}

Den Fragment-Shader verändern:\\
\begin{tabular}{cl}
 $\square$ \CheckedBox TODO & Alpha-Ausgabewert von Feuerpartikeln auf (1 - Texel-Alpha) setzen.\\
 $\square$ \CheckedBox TODO & Alpha-Ausgabewert von Feuerpartikeln auf 1 setzen.\\
 $\square$ \CheckedBox TODO & Alpha-Ausgabewert von Feuerpartikeln auf 0 setzen.\\
 $\square$ \CheckedBox TODO & Ausgabewerte unverändert beibehalten.\\
\end{tabular}

Vertex-Attribute hinzufügen:\\
\begin{tabular}{cl}
 $\square$ \CheckedBox TODO & zur feingranularen Anpassung der OpenGL Blending-Konfiguration,\\
 $\square$ \CheckedBox TODO & zur Unterscheidung von Feuer- und Rauchpartikeln,\\
 $\square$ \CheckedBox TODO & zur Einstellung der Zeichenebene.\\
\end{tabular}

\section*{Aufgabe 9: OpenGL}
\begin{tabular}{cp{10cm}ll}\toprule
\# & Aussage                                                                                             & Wahr      & Falsch \\\midrule
1  & Die Dehomogenisierung wird in der OpenGL-Pipeline nach dem Fragment-Shader durchgeführt.            & TODO      & TODO      \\
2  & Die Pipeline-Stufe \enquote{Primitive Assembly} ist programmierbar.                                 & $\square$ & \CheckedBox \\
3  & Clipping am Sichtvolumen wird von der OpenGL-Pipeline vor dem Fragment-Shader durchgeführt.         & TODO      & TODO\\
4  & Shader-Programme können den Zustand des OpenGL-Zustandsautomaten (State Machine) verändern.         & TODO      & TODO\\
5  & Die Beleuchtungsberechnung muss im Fragment-Shader immer in Kamera-Koordinaten durchgeführt werden. & TODO      & TODO\\
6  & Texturen können ausschließlich im Fragment-Shader gelesen werden.                                   & TODO      & TODO\\\bottomrule
\end{tabular}


\section*{Aufgabe 10: Reflexionen in OpenGL}
\inputminted[linenos, numbersep=5pt, tabsize=4, frame=lines, label=shader.frag]{glsl}{shader.frag}

\section*{Aufgabe 11: GLSL-Hatching}
\subsection*{Teilaufgabe 11a}

\[
	\left(1 - \frac{x - x_1}{x_2 - x_1}\right) y_1 + \frac{x - x_1}{x_2 - x_1} y_2
\]

oder

\[
	y_1 + \frac{y_2 - y_1}{x_2 - x_1} (x - x_1)
\]

\clearpage
\subsection*{Teilaufgabe 11b}
\inputminted[linenos, numbersep=5pt, tabsize=4, frame=lines, label=shader.frag]{glsl}{hatching.frag}

\clearpage
\section*{Aufgabe 12: Bézierkurven}
\subsection*{Teilaufgabe 12a}
\textit{Was versteht man unter affiner Invarianz bei Bézierkurven?}

Sei $F(u) = \sum_{i=0}^n \mathbf{b}_i B_i^n$ eine Bézierkurve und
$\varphi(x) = A x + t$ eine affine Funktion. Dann gilt:

\[\varphi(F(u)) = \sum_{i=0}^n B_i^n \varphi(\mathbf{b}_i)\]

\subsection*{Teilaufgabe 12b}
\includegraphics*[width=\linewidth, keepaspectratio]{12b.png}

\subsection*{Teilaufgabe 12c}
\textit{Geben Sie an, ob es sich bei den folgenden Kurven mit gegebenem
Kontrollpolygon um Bézierkurven handelt! Begründen Sie jeweils kurz Ihre
Antwort, falls es sich nicht um eine Bézierkurve handelt!}
\begin{enumerate}
    \item Nein, da die Kurve nicht symmetrisch ist. Insbesondere müsste sie
          durch die Mitte des zweiten und dritten Kontrollpunktes gehen.
    \item Ja
    \item Nein, da die Kurve nicht innerhalb der konvexen Hülle der
          Kontrollpunkte ist.
\end{enumerate}

\end{document}
