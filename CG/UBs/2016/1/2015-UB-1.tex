\documentclass[a4paper]{scrartcl}
\usepackage[ngerman]{babel}
\usepackage[utf8]{inputenc}
\usepackage{amssymb,amsmath}
\usepackage{graphicx}
\usepackage[inline]{enumitem}
\setlist{noitemsep}
\usepackage[binary-units=true]{siunitx}
\usepackage{hyperref}
\usepackage{parskip}
\usepackage[nameinlink,noabbrev,ngerman]{cleveref} % has to be after hyperref
\usepackage{nicefrac}  % for \nicefrac{1}{3}
\usepackage{csquotes}  % for \enquote{what you want to quote}
\usepackage{booktabs}  % for \toprule, \midrule and \bottomrule
\usepackage{minted} % needed for the inclusion of source code

% for \begin{enumerate}[label=(\Alph*)], see http://tex.stackexchange.com/a/129960/5645
\usepackage{enumitem}

\setcounter{secnumdepth}{2}
\setcounter{tocdepth}{2}

\usepackage{wasysym}  % For \CheckedBox
\usepackage{microtype}

\begin{document}
\selectlanguage{ngerman}
\title{2015, Übungsblatt 1}

\setcounter{section}{1}
%%%%%%%%%%%%%%%%%%%%%%%%%%%%%%%%%%%%%%%%%%%%%%%%%%%%%%%%%%%%%%%%%%%%%%%%%%%%%%
\section*{Aufgabe 1: Farben und Farbwahrnehmung}
vgl. Hauptklausur 2014
\subsection*{Teilaufgabe 1a}
\textit{Tragen Sie die Farben Grün, Rot, Gelb, Orange, Cyan, Magenta in die
        entsprechenden Felder im Chromatizitätsdiagramm ein.}

\includegraphics*[width=0.8\linewidth, keepaspectratio]{1a.png}

\subsection*{Teilaufgabe 1b}
\textit{Welcher der Farbeindrücke aus Aufgabe a) lässt sich nicht durch monochromatisches Licht
erzeugen?}

Alles auf der Purple line. Also insbesondere \textcolor{magenta}{\textbf{Magenta}}.

\subsection*{Teilaufgabe 1c}
\textit{Die Farbräume $xyY$ und $XYZ$ sind eng verwandt. Wie ist der
mathematische Zusammenhang zwischen der Chromatizität $(x, y)$ und der
passenden Farbe $(X, Y, Z)$?}

\begin{align}
    x &= \frac{X}{X + Y + Z}\\
    y &= \frac{Y}{X + Y + Z}
\end{align}

\subsection*{Teilaufgabe 1d}
\textit{Ordnen Sie diese Räume aufsteigend nach der Größe ihres für den
Menschen sichtbaren Gamut.}

$(2) < (3) < (1)$, also\\

RGB $<$ Raum aller Farben die durch 100 monochromatische Leuchtdioden
darstellbar sind $<$ XYZ

\end{document}
