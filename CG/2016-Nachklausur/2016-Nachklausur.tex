\documentclass[a4paper]{scrartcl}
\usepackage[ngerman]{babel}
\usepackage[utf8]{inputenc}
\usepackage{amssymb,amsmath}
\usepackage{graphicx}
\usepackage[inline]{enumitem}
\setlist{noitemsep}
\usepackage[binary-units=true]{siunitx}
\usepackage{hyperref}
\usepackage{parskip}
\usepackage[nameinlink,noabbrev,ngerman]{cleveref} % has to be after hyperref
\usepackage{csquotes}  % for \enquote{what you want to quote}
\usepackage{booktabs}  % for \toprule, \midrule and \bottomrule
\usepackage{minted} % needed for the inclusion of source code

\usepackage{tikz}
\usetikzlibrary{calc}

% for \begin{itemize}[label=(\Alph*)], see http://tex.stackexchange.com/a/129960/5645
\usepackage{enumitem}

\setcounter{secnumdepth}{2}
\setcounter{tocdepth}{2}

\usepackage{wasysym}  % For \CheckedBox
\usepackage{microtype}

\begin{document}
\selectlanguage{ngerman}
\title{2016 Nachklausur (SS 2016)}

\setcounter{section}{1}
%%%%%%%%%%%%%%%%%%%%%%%%%%%%%%%%%%%%%%%%%%%%%%%%%%%%%%%%%%%%%%%%%%%%%%%%%%%%%%
\section*{Aufgabe 1}
\subsection*{Teilaufgabe 1a}
Der Gamut ist der Raum aller vom Monitor darstellbaren Farben.

\subsection*{Teilaufgabe 1b}
\begin{tabular}{p{7cm}cccc}\toprule
	\textbf{Aussage}                                                          & \textbf{RGB} & \textbf{CMY} & \textbf{HSV} & \textbf{CIE xyY}\\\midrule
	Der Farbraum ist additiv.                                                 & \CheckedBox  & \Square      & \Square      & \Square         \\
	Der Farbraum ist subtraktiv.                                              & \Square      & \CheckedBox  & \Square      & \Square         \\
	Der Farbraum trennt Helligkeit von Farbe.                                 & \Square      & \Square      & \CheckedBox  & \CheckedBox     \\
	Der Farbraum kann alle für den Menschen sichtbaren Farben repräsentieren. & \Square      & \Square      & \Square      & \CheckedBox     \\
	Der Farbraum wird nativ auf Peripheriegeräten verwendet.                  & \CheckedBox  & \CheckedBox  & \Square      & \Square         \\\bottomrule
\end{tabular}

\subsection*{Teilaufgabe 1c}
Nein (negative Energieabstrahlung nicht möglich)

\section*{Aufgabe 2}
\subsection*{Teilaufgabe 2a}
TODO

\subsection*{Teilaufgabe 2b}
TODO

\subsection*{Teilaufgabe 2c}
\begin{tabular}{cp{12cm}ll}\toprule
	~ & \textbf{Aussage} & \textbf{Wahr} & \textbf{Falsch} \\\midrule
	1 & Wenn ein Lichtstrahl aus einem optisch dichteren Medium in ein optisch dünneres Medium übergeht, kann Totalreflexion auftreten.
	  & \CheckedBox & \Square     \\
	2 & Die Zeitkomplexität von Whitted-style Raytracing ist polynomial in der Rekursionstiefe.
	  & \Square     & \CheckedBox \\
	3 & Es gibt eine Klasse von Materialien, die beim Whitted-style Raytracing mehrere Sekundärstrahlen per Schnittpunkt erzeugen.
	  & \CheckedBox & \Square     \\
	4 & Der Whitted-style Raytracer überprüft die direkte Beleuchtung eines Oberflächenpunktes mithilfe von Schattenstrahlen.
	  & \CheckedBox & \Square     \\
	5 & Whitted-style Raytracing konvergiert immer zum physikalisch korrekten Ergebnis.
	  & \Square     & \CheckedBox \\
	6 & Whitted-style Raytracing ist das Bildsyntheseverfahren, das auf der GPU in Hardware implementiert ist.
	  & \Square     & \CheckedBox \\\bottomrule
\end{tabular}

\section*{Aufgabe 3}
\subsection*{Teilaufgabe 3ab}
\begin{tikzpicture}[scale=0.9,every label/.style={red,font=\small}]
	\draw[help lines] (0,0) grid (16,16);
	\draw (0,0) rectangle (16,16);
	\foreach \x in {4,8,12} {
		\draw (\x, -0.2) -- (\x, +0.2);
		\draw (\x, 15.8) -- (\x, 16.2);
	};
	\foreach \y in {4,8,12} {
		\draw (-0.2, \y) -- (+0.2, \y);
		\draw (15.8, \y) -- (16.2, \y);
	};
	
	\node[label= 93:1] at  (5.1,14.9) (a) {a};
	\node[label=180:3] at  (6.9,13.4) (b) {b};
	\node[label=  0:1] at  (9.1,14.7) (c) {c};
	\node[label=-50:1] at (11.1,13.0) (d) {d};
	\node[label=-35:1] at  (1.8,9.6)  (e) {e};
	\node[label= 95:4] at  (5.6,6.2)  (f) {f};
	\node[label=-90:1] at  (7.5,4.4)  (g) {g};
	\node[label=-90:1] at (10.7,2.3)  (h) {h};
	
	\draw  (4.8,15.8) --  (5.8,14.7) --  (4.6,14.2) -- cycle;
	\draw  (5.3,12.4) --  (7.6,12.4) --  (7.6,15.6) -- cycle;
	\draw  (8.4,14.2) --  (9.3,14.4) --  (9.3,15.6) -- cycle;
	\draw (10.2,12.4) -- (11.8,12.9) -- (11.1,13.6) -- cycle;
	\draw  (0.2,8.3)  --  (2.9,8.3)  --  (2.4,11.3) -- cycle;
	\draw  (4.3,5.2)  --  (7.4,5.6)  --  (5.3,7.2)  -- cycle;
	\draw  (7.1,4.2)  --  (7.9,4.2)  --  (7.6,4.9)  -- cycle;
	\draw  (9.8,2.0)  -- (11.0,3.2)  -- (11.6,0.9)  -- cycle;
	
	\begin{scope}[red]
		\draw[step=4] (0.2,0.2) grid (15.8,15.8);
		\draw  (4,14) -- (12,14);
		\draw  (6,12) --  (6,16);
		\draw (10,12) -- (10,16);
		\draw[step=2] (4,4) grid (8,8);
		\draw[step=1] (6,4) grid (8,6);
	\end{scope}
\end{tikzpicture}

\subsection*{Teilaufgabe 3c}
e, f, f, f, h

\subsection*{Teilaufgabe 3d}
\begin{tabular}{p{7cm}p{1.5cm}p{1.5cm}p{1.5cm}p{1.5cm}}\toprule
	\textbf{Aussage}                                                                                     & \textbf{BVH} & \textbf{kD-Baum} & \textbf{Gitter} & \textbf{Keine} \\\midrule
	Primitive können in mehr als einem Blattknoten/einer Gitterzelle vorhanden sein.                     & \CheckedBox  & \CheckedBox      & \CheckedBox     & \Square        \\
	Die Datenstruktur kann zur Beschleunigung von Nachbarschaftssuchen verwendet werden.                 & \Square      & \Square          & \CheckedBox     & \Square        \\
	Die Datenstruktur eignet sich besonders für Szenen, in denen die Geometrie gleichmäßig verteilt ist. & \Square      & \Square          & \CheckedBox     & \Square        \\
	Der Aufbau-Algorithmus passt die Datenstruktur an die Normalen der Geometrie an.                     & \Square      & \Square          & \Square         & \CheckedBox    \\\bottomrule
\end{tabular}

\subsection*{Teilaufgabe 3e}
\begin{tabular}{cp{12cm}ll}\toprule
	~ & \textbf{Aussage} & \textbf{Wahr} & \textbf{Falsch} \\\midrule
	1 & Die Surface Area Heuristic minimiert die Anzahl der Primitive, die sich in der Beschleunigungsstruktur befinden.
	  & \Square     & \CheckedBox \\
	2 & Der Aufbau einer BVH mithilfe der Surface Area Heuristic ist im Allgemeinen aufwändiger als der Aufbau mittels Objektmittel-Methode (object median).
	  & \CheckedBox & \Square     \\
	3 & BSP-Bäume sind kD-Bäume mit achsparallelen Unterteilungsebenen.
	  & \Square     & \CheckedBox \\
	4 & Die Objektmittel-Methode (object median) kann beim Erstellen von BVH und kD-Baum verwendet werden.
	  & \CheckedBox & \Square     \\
	5 & Die Raummittel-Methode (spatial median) kann beim Erstellen von BVH und kD-Baum verwendet werden.
	  & \CheckedBox & \Square     \\
	6 & Eine optimale objektorientierte Box (OOBBs) hat höchstens das Volumen der entsprechenden achsenparallelen Box (AABB).
	  & \CheckedBox & \Square     \\\bottomrule
\end{tabular}

\section*{Aufgabe 4}
\subsection*{Teilaufgabe 4a}
\[c = \lambda_1 c_1 + \lambda_2 c_2 + \lambda_3 c_3\]

\subsection*{Teilaufgabe 4b}
Die interpolierte Normale muss erneut normalisiert werden.

\subsection*{Teilaufgabe 4c}
\begin{itemize}
	\item $\lambda_1 = \frac{2}{3}$
	\item $\lambda_2 = 0$
	\item $\lambda_3 = \frac{1}{3}$
\end{itemize}

\subsection*{Teilaufgabe 4d}
Das Phong-Beleuchtungsmodell kann für alle Verfahren eingesetzt werden.


\section*{Aufgabe 5}
\subsection*{Teilaufgabe 5a}
\[c_{BL}(T_p) = 0.4 \cdot (100 \cdot 0.1) + 0.8 \cdot (100 \cdot 0.9) = 4 + 72 = 76\]

\subsection*{Teilaufgabe 5b}
\begin{tikzpicture}[every circle/.style={radius=2pt}]
	\fill (0,0.0) coordinate[label=left: $P_1$] (P1) circle;
	\fill (3,3.0) coordinate[label=above:$P_2$] (P2);
	\fill (7,1.5) coordinate[label=right:$P_3$] (P3) circle;
	
	\fill[gray] ($(P1)!0.5!(P2)$) -- (P2) -- ($(P2)!0.5!(P3)$) -- ($(P1)!0.5!(P3)$);
	\fill (P2) circle;
	\draw (P1) -- (P2) -- (P3) -- (P1);
\end{tikzpicture}

\subsection*{Teilaufgabe 5c (i)}
Stufe 1 (1 Texel): $\frac{1}{4} \cdot 0 + \frac{1}{4} \cdot 100 + \frac{1}{4} \cdot 0 + \frac{1}{4} \cdot 100 = 50$

\subsection*{Teilaufgabe 5c (ii)}
\[c_{MM}(T_p) = \frac{1}{2} \cdot 50 + \frac{1}{2} \cdot 76 = 25 + 38 = 63\]

\subsection*{Teilaufgabe 5c (iii)}
\begin{tabular}{lcccc}\toprule
	\textbf{Fall} & \textbf{Nearest} & \textbf{Bilinear} & \textbf{Mip-Map-Nearest} & \textbf{Mip-Map-Trilinear} \\\midrule
	Minification  & \Square          & \Square           & \Square                  & \CheckedBox                \\
	Magnification & \Square          & \CheckedBox       & \Square                  & \Square                    \\\bottomrule
\end{tabular}

\section*{Aufgabe 6}
\subsection*{Teilaufgabe 6a}
\begin{tabular}{ll}
	\verb|GL_TRIANGLES|      & 0,1,3,0,2,3,1,2,3 \\
	\verb|GL_TRIANGLE_STRIP| & 0,1,3,2,0         \\
	\verb|GL_TRIANGLE_FAN|   & 3,1,0,2,1         \\
\end{tabular}

\subsection*{Teilaufgabe 6b}
Vertex-Shader $\rightarrow$ Geometry-Shader $\rightarrow$ Fragment-Shader

\section*{Aufgabe 7}
\subsection*{Teilaufgabe 7a}
\inputminted[linenos, numbersep=5pt, tabsize=4, frame=lines, label=shader7a.frag, mathescape]{glsl}{shader7a.frag}

\subsection*{Teilaufgabe 7b}
TODO

\section*{Aufgabe 8}
\subsection*{Teilaufgabe 8a}
TODO

\subsection*{Teilaufgabe 8b}
\inputminted[linenos, numbersep=5pt, tabsize=4, frame=lines, label=blending.cpp]{cpp}{blending.cpp}

\subsection*{Teilaufgabe 8c}
\mint{cpp}|glBlendFunc(GL_SRC_ALPHA, GL_ONE_MINUS_SRC_ALPHA);|

\subsection*{Teilaufgabe 8d}
\mint{cpp}|glBlendFunc(GL_ONE, GL_ONE);|

\section*{Aufgabe 9}
\subsection*{Teilaufgabe 9a}
\inputminted[linenos, numbersep=5pt, tabsize=4, frame=lines, label=shader9a.frag]{glsl}{shader9a.frag}

\subsection*{Teilaufgabe 9b}
\inputminted[linenos, numbersep=5pt, tabsize=4, frame=lines, label=shader9b.frag]{glsl}{shader9b.frag}

\section*{Aufgabe 10}
\subsection*{Teilaufgabe 10a}
\begin{flalign*}
&  (1 - u^2) \mathbf{b}_0 + 2u (1 - u) \mathbf{b}_1 + u^2 \mathbf{b}_2\\
&= (1 - 2u + u^2) \mathbf{b}_0 + (2u - 2u^2) \mathbf{b}_1 + u^2 \mathbf{b}_2\\
&= \mathbf{b}_0 u^2 - 2 \mathbf{b}_1 u^2 + \mathbf{b}_2 u^2 - 2 \mathbf{b}_0 u + 2 \mathbf{b}_1 u + \mathbf{b}_0\\
&= (\underbrace{\mathbf{b}_0 - 2 \mathbf{b}_1 + \mathbf{b}_2}_{= \mathbf{a}_2}) u^2 + (\underbrace{2 \mathbf{b}_1 - 2 \mathbf{b}_0}_{= \mathbf{a}_1}) u + \underbrace{\mathbf{b}_0}_{= \mathbf{a}_0}
\end{flalign*}

\subsection*{Teilaufgabe 10b}
\begin{tikzpicture}[scale=0.75,every circle/.style={radius=2pt}]
\draw[help lines] (0,0) grid (15,15);
\draw[thick,->] (0,0) -- (16,0) node[below right] {x};
\draw[thick,->] (0,0) -- (0,16) node[left] {y};
\draw foreach \x in {1,...,14} {(\x,0.2) -- (\x,-0.2) node[below] {\x}};
\draw foreach \y in {1,...,14} {(0.2,\y) -- (-0.2,\y) node[left] {\y}};

\fill  (2,2)  coordinate[label=left: $\mathbf{b}_0$] (b0) circle;
\fill  (6,2)  coordinate[label=below:$\mathbf{b}_1$] (b1) circle;
\fill (14,14) coordinate[label=right:$\mathbf{b}_2$] (b2) circle;
\fill (14,2)  coordinate[label=right:$\mathbf{b}_3$] (b3) circle;

\begin{scope}[red]
	\draw[dashed] (b0) -- (b1) -- (b2) -- (b3);
	\fill ($(b0)!0.5!(b1)$) coordinate[label=below:     $\mathbf{b}_0^0$] (b00) circle;
	\fill ($(b1)!0.5!(b2)$) coordinate[label=above left:$\mathbf{b}_1^0$] (b10) circle;
	\fill ($(b2)!0.5!(b3)$) coordinate[label=right:     $\mathbf{b}_2^0$] (b20) circle;
	
	\draw[dotted] (b00) -- (b10) -- (b20);
	\fill ($(b00)!0.5!(b10)$) coordinate[label=above left:$\mathbf{b}_0^1$] (b01) circle;
	\fill ($(b10)!0.5!(b20)$) coordinate[label=above:     $\mathbf{b}_1^1$] (b11) circle;
\end{scope}

\draw[dotted] (b01) -- (b11);
\fill ($(b01)!0.5!(b11)$) coordinate[label=below right:$\mathbf{b}\left(\frac{1}{2}\right)$] circle;

\draw (b0) .. controls (b1) and (b2) .. (b3);
\end{tikzpicture}


\end{document}
