\documentclass[a4paper]{scrartcl}
\usepackage[ngerman]{babel}
\usepackage[utf8]{inputenc}
\usepackage{amssymb,amsmath}
\usepackage{graphicx}
\usepackage[inline]{enumitem}
\setlist{noitemsep}
\usepackage[binary-units=true]{siunitx}
\usepackage{hyperref}
\usepackage{parskip}
\usepackage[nameinlink,noabbrev,ngerman]{cleveref} % has to be after hyperref
\usepackage{nicefrac}  % for \nicefrac{1}{3}
\usepackage{csquotes}  % for \enquote{what you want to quote}
\usepackage{booktabs}  % for \toprule, \midrule and \bottomrule
\usepackage{minted} % needed for the inclusion of source code

% for \begin{enumerate}[label=(\Alph*)], see http://tex.stackexchange.com/a/129960/5645
\usepackage[inline]{enumitem}

\setcounter{secnumdepth}{2}
\setcounter{tocdepth}{2}

\usepackage{wasysym}  % For \CheckedBox
\usepackage{microtype}

\begin{document}
\selectlanguage{ngerman}
\title{2011 Hauptklausur (WS 2010/11)}


\setcounter{section}{1}
%%%%%%%%%%%%%%%%%%%%%%%%%%%%%%%%%%%%%%%%%%%%%%%%%%%%%%%%%%%%%%%%%%%%%%%%%%%%%%
\section*{Aufgabe 1: Wahrnehmung und Farbräume}
\subsection*{Teilaufgabe 1a}
\textit{Welche Eigenschaft der menschlichen Wahrnehmung wird durch das Weber-Fechner-Gesetz beschrieben?}

Das Weber-Fechner-Gesetz macht eine Aussage über die subjektiv empfundene
Stärke von Sinneseindrücken im Abhängigkeit von der Intensität des
Helligkeitsunterschiedes.

Es wird die Eigenschaft, dass die Stärke des Sinneseindrucks von der Intensität
logarithmisch abhängt beschrieben.


\subsection*{Teilaufgabe 1b}
\textit{Was ist der Gamut eines Monitors?}

Der Gamut eines Monitors entspricht dem Spektrum der darauf darstellbaren
Farben.

\subsection*{Teilaufgabe 1c}
\begin{tabular}{p{8cm}llll}\toprule
Aussage                                                  & RGB & CMY & HSV & CIE xyY \\\midrule
Der Farbraum ist additiv.                                & \CheckedBox & $\square$   & $\square$   & $\square$ \\
Der Farbraum ist subtraktiv.                             & $\square$   & \CheckedBox & $\square$   & $\square$ \\
Der Farbraum ist multiplikativ.                          & $\square$   & \CheckedBox & $\square$   & $\square$ \\
Der Farbraum trennt Luminanz von Chrominanz.             & $\square$   & $\square$   & \CheckedBox & $\square$ \\
Der Farbraum kann alle sichtbaren Farben repräsentieren. & $\square$   & $\square$   & ~           & \CheckedBox       \\
Der Farbraum wird nativ auf Peripheriegeräten verwendet. & \CheckedBox & \CheckedBox & $\square$   & $\square$ \\\bottomrule
\end{tabular}

\clearpage
\section*{Aufgabe 2: Prozedurale Modellierung}
\subsection*{Teilaufgabe 2a}
\textit{Nennen Sie einen Vorteil und einen Nachteil prozeduraler Beschreibungen! Nennen Sie
zwei Beispiele für deren Modellierung prozedurale Modelle gut geeignet sind!}

\begin{itemize}
    \item \textbf{Vorteil}: Wenig Speicherplatz wird benötigt
    \item \textbf{Nachteil}: Komplexe Beschreibung
    \item \textbf{Beispiele}
    \begin{itemize}
        \item Bäume, Gebirge, Vegetation
        \item Hypertextures (Wolken)
        \item Landschaften
    \end{itemize}
\end{itemize}

\subsection*{Teilaufgabe 2b}
\textit{Was versteht man unter Rauschtexturen nach Perlin?}

Überlagerung verschiedener Frequenzbereiche (spektrale Synthese) zur Erzeugung
von Rauschfunktionen für Texturen.

\textit{Was sind wichtige Eigenschaften dieser Rauschtexturen?}

\begin{enumerate*}[label=(\arabic*)]
    \item Reproduzierbarkeit
    \item Keine sichtbare Periodizität
\end{enumerate*}

\textit{Geben Sie eine einfache Möglichkeit an, um 2D-Rauschtexturen zu berechnen.}

Rauschtexturen können anhand eines 2D-Gitters mit Zufallswerten und
interpolation berechnet werden.

\section*{Aufgabe 3: Ray-Tracing}
\subsection*{Teilaufgabe 3a}
\textit{Nennen Sie die vier Arten von Strahlen, die beim
Whitted-Style-Ray-Tracing auftreten können. Welche dieser Strahlen benötigen
zur Berechnung Rekursion?}

\begin{enumerate*}[label=(\arabic*)]
    \item Primärstrahlen
    \item Reflektionsstrahlen (rekursiv)
    \item Transmissionsstrahlen (rekursiv)
    \item Schattenstrahlen
\end{enumerate*}

\clearpage
\subsection*{Teilaufgabe 3b}
\textit{Nennen Sie die zwei Abbruchkriterien für die Rekursion, die Sie in der
Vorlesung kennengelernt haben! Welchen Vorteil haben die Kriterien jeweils?}
\begin{enumerate}[label=(\arabic*)]
    \item \textbf{Maximale Rekursionstiefe}: Ist einfach zu implementieren
    \item \textbf{Schwellwertunterschreitung} (der Zusätzliche Beitrag
    unterschreitet einen vorgegebenen Schwellwert): Ist adaptiv
\end{enumerate}

\subsection*{Teilaufgabe 3c}
\textit{In welchem Fall ist keine (weitere) Rekursion notwendig, nachdem ein
Schnittpunkt gefunden wurde?}

Schnittpunkt mit einer Lichtquelle


\section*{Aufgabe 4: Texturen}
\subsection*{Teilaufgabe 4a}
\textit{Was versteht man unter Magnification und Minification bei der Texturierung? Nennen
und erläutern Sie kurz je eine Möglichkeit, wie Sie den hierbei auftretenden Artefakten
begegnen können.}

\begin{itemize}
    \item \textbf{Magnification}: Ein Texel wird auf viele Pixel abgebildet.
    \begin{itemize}
        \item Problem: Bild ist unschaft
        \item Lösung: Bilineare Interpolation
    \end{itemize}
    \item \textbf{Minification}: Viele Texel werden auf einen Pixel abgebildet.
    \begin{itemize}
        \item Problem: Aliasing durch Überabtastung
        \item Lösung: Supersampling oder Vorfilterung (z.B. Mipmaps, ripemaps)
    \end{itemize}
\end{itemize}

\subsection*{Teilaufgabe 4b}
\textit{Nennen Sie zwei Parametrisierungen für Environment Maps und für jede angegebene
Parametrisierung einen Vorteil oder Nachteil.}\\
\begin{itemize}
    \item \textbf{Sphere-Map} hat den Nachteil, dass am Rand eine Singularität
          ist.
    \item \textbf{Cube-Map} hat den Vorteil, sie bei korrekter Filterung
          Nahtlos ist.
    \item \textbf{LatLong-Map} hat den Nachteil, dass die Pole ungleichmäßig
          abgetastet werden.
\end{itemize}

\section*{Aufgabe 5: Räumliche Datenstrukturen}
\subsection*{Teilaufgabe 5a}
\textit{Begründen Sie, warum und wofür räumliche Datenstrukturen für
Ray-Tracing von komplexen Szenen besonders wichtig sind!}

Zur Beschleunigung der Schnittpunktberechnung bei Szenen mit vielen Objekten /
Primitiven.

\subsection*{Teilaufgabe 5b}
\begin{tabular}{p{8cm}llll}\toprule
Aussage                                                                                                 & BVH           & Octree           & Gitter           & BSP \\\midrule
Der Aufbau-Algorithmus ist adaptiv und passt die Datenstruktur deshalb automatisch an die Geometrie an. & \CheckedBox   & \CheckedBox      & $\square$        & \CheckedBox   \\
Die Datenstruktur wird durch einen Binärbaum repräsentiert.                                             & \CheckedBox   & $\square$        & $\square$        & \CheckedBox   \\
Objekte werden bei der Traversierung potentiell mehrfach von demselben Strahl geschnitten.              & $\square$     & $\square$        & \CheckedBox      & $\square$     \\
Bei der Traversierung wird leerer Raum effizient übersprungen.                                          & \CheckedBox   & \CheckedBox      & $\square$        & \CheckedBox  \\
Der Raum wird durch die Datenstruktur immer achsenparallel unterteilt.                                  & $\square$     & \CheckedBox      & \CheckedBox      & $\square$   \\
Bei der Konstruktion kann die Surface-Area-Heuristik sinnvoll eingesetzt werden.                        & \CheckedBox   & $\square$        & $\square$        & \CheckedBox   \\\bottomrule
\end{tabular}

\section*{Aufgabe 6: Clipping}
\subsection*{Teilaufgabe 6a}
\textit{Mit welchen Kanten der Clipping-Region wird die Strecke $\overline{P_{1}P_{2}}$ beim Cohen-Sutherland-
Clipping-Algorithmus potenziell geschnitten?}

\begin{enumerate}
    \item Outcode($P_1$): 1010; Outcode($P_2$): 0100
    \item Outcode($P_1$) $|$ Outcode($P_2$) $ = 1110 \neq 0000 \Rightarrow$ kein trivial accept.
    \item Outcode($P_1$) \& Outcode($P_2$) $ = 0000 \Rightarrow$ kein trivial reject.
    \item Es werden potentiell die linke (10xx), untere (xx10) und rechte
          (01xx) Kante geschnitten, da die entsprechenden Optcodes gesetzt sind
\end{enumerate}

\textit{Wovon hängt es ab, mit welchen Kanten die Strecke tatsächlich
geschnitten wird?}

TODO

\subsection*{Teilaufgabe 6b}

\textit{In welchen Fällen kann der Algorithmus ausschließlich anhand der Outcodes eine
beliebige Strecke $\overline{AB}$ eliminieren?}

In den \textit{trivial accept} und \textit{trivial reject} Fällen kann der
Alglorithmus ausschlieslich anhand der Optcodes eine Strecke eliminieren.

\textit{Welche Bitoperation wird mit den Outcodes hierzu durchgeführt?}\\

Bit-weises und (\&) für trivial reject und bit-weises oder ($|$) für trivial
accept.


\section*{Aufgabe 7: Shading}
\subsection*{Teilaufgabe 7a}
\begin{tabular}{p{8cm}lll}\toprule
Aussage                                                                                                                                                                                                                   & Flat           & Gouraud           & Phong \\\midrule
Die Normale wird für jeden Pixel aus Normalen der Eckpunkte interpoliert.                                                                                                                                                 & $\square$      & $\square$         & \CheckedBox     \\
Der Mach-Band-Effekt (Machsche Streifen) kann sichtbar werden / auftreten.                                                                                                                                                & $\square$      & \CheckedBox       & $\square$       \\
Die Beleuchtung wird ausschließlich an den Vertex-Positionen berechnet und anschließend interpoliert.                                                                                                                     & $\square$      & \CheckedBox       & $\square$       \\
Für die Berechnung der Beleuchtung wird die Flächennormale des Dreiecks verwendet.                                                                                                                                        & \CheckedBox    & $\square$         & $\square$       \\
Die Ebene in Abbildung 1, repräsentiert durch ein Dreiecksnetz, soll diffuse und spekulare Reflexionseigenschaften aufweisen. Dann ändert sich durch eine feinere Unterteilung der Ebene die berechnete Beleuchtung nicht.& $\square$      & $\square$         & \CheckedBox     \\
Wird in der Fixed-Function-Pipeline von OpenGL unterstützt.                                                                                                                                                               & \CheckedBox    & \CheckedBox       & $\square$       \\\bottomrule
\end{tabular}

\subsection*{Teilaufgabe 7b}
\textit{Wie werden Normalenvektoren beim Phong-Shading interpoliert?}

Linear mit baryzentrischen Koordinaten.

\textit{Warum muss im Allgemeinen nach der Interpolation erneut normalisiert werden?}

Die ermittelte normale kann eine Länge $\neq 1$ aufweisen.


\subsection*{Teilaufgabe 7c}
\textit{Kann man das Blinn-Phong-Beleuchtungsmodell mit Flat-Shading kombinieren?}

Ja. Man kann die Dreiecksnormale für die Berechnung des Diffusanteils pro Pixel
verwenden.

\section*{Aufgabe 8: Rasterisierung}
\begin{tabular}{p{11cm}ll}\toprule
Aussage                                                                                                                  & Wahr           & Falsch \\\midrule
Sichtbarkeit wird vor der Rasterisierung berechnet.                                                                      & $\square$      & \CheckedBox      \\
Opake Primitive können mit korrekter Verdeckungsberechnung rasterisiert werden, ohne vorher explizit sortiert zu werden. & \CheckedBox    & $\square$        \\
Der Tiefenwert (z-Wert) kann mit Hilfe von baryzentrischen Koordinaten korrekt im Bildraum interpoliert werden.          & \CheckedBox    & $\square$        \\
Der Tiefenwert, der im Tiefenpuffer gespeichert wird, hängt von den Abständen der Near- und Far-Plane zur Kamera ab.     & \CheckedBox    & $\square$        \\
Transparente Oberflächen können ohne vorherige Sortierung korrekt dargestellt werden.                                    & $\square$      & \CheckedBox      \\
In OpenGL wird der Tiefentest vor den Alpha- und Stenciltests durchgeführt.                                              & $\square$      & \CheckedBox      \\\bottomrule
\end{tabular}


\section*{Aufgabe 9: OpenGL}
\subsection*{Teilaufgabe 9a}
\textit{Welchen Vorteil haben Dreiecksstreifen (\texttt{GL\_TRIANGLE\_STRIP})
im Vergleich zu isolierten Dreiecken (\texttt{GL\_TRIANGLES})?}\\
Beim Triangle-Strip werden die Vertices für mehrere Dreiecke verwendet. Es muss
also weniger gespeichert werden.

\subsection*{Teilaufgabe 9b}
\textit{Was versteht man unter einem Indexed Face Set? Welche Optimierung bei
der Geometrieverarbeitung wird dadurch erst ermöglicht?}\\
Liste von Indices für die Vertices in einem Polygonmesh. Die Vertices werden
also indirekt referenziert. Dadurch wird die Verwendung eines Vertexcache
ermöglicht.

\subsection*{Teilaufgabe 9c}
\textit{Was versteht man bei OpenGL unter \enquote{Blending}? Nennen Sie eine
wichtige Anwendung wofür es eingesetzt werden kann!}\\
Blending bezeichnet die Farbkombination (z.B. Überlagerung) von erzeugten
Fragmenten und Pixeln im Framebuffer. Dadurch lassen sich semi-transparente
Objekte darstellen.

\subsection*{Teilaufgabe 9d}
\textit{Geben Sie zwei kommutative Blending-Einstellungen mittels OpenGL-Befehlen an.
Gehen Sie dabei davon aus, dass als Blending-Verknüpfungsoperation die Addition
eingestellt ist.}\\

\begin{itemize}
    \item \texttt{glBlendFunc(GL\_ONE, GL\_ONE)}
    \item \texttt{glBlendFunc(GL\_ZERO, GL\_ZERO)}
\end{itemize}

\subsection*{Teilaufgabe 9e}
\textit{Erläutern Sie die Aufgaben eines Vertex-Shaders bzw. Vertex-Programms
und welche Ein- und Ausgabedaten zur Verfügung stehen und erzeugt werden müssen
bzw. können!}\\
Ein Vertex-Shader transformiert einen Vertex und verarbeitet dessen Attribute
(Position, Normale, \dots).

\begin{itemize}
    \item Input: \texttt{gl\_Vertex}, Matrizen (MVP-Matrix)
    \item Output: \texttt{gl\_Position}, \texttt{gl\_Normal}
\end{itemize}


\section*{Aufgabe 10: Bézierkurven}
\subsection*{Teilaufgabe 10a}
\textit{Wie nennt man den Algorithmus zur rekursiven Auswertung von
Bézierkurven, der auch eine grafische Repräsentation hat?}

Algorithmus von De Casteljau.

\clearpage
\subsection*{Teilaufgabe 10b}
\textit{Zählen Sie vier wichtige Eigenschaften von Bézierkurven, die Sie in der
Vorlesung kennengelernt haben, auf.}

\begin{itemize}
    \item \textbf{Tangentenbedingung}:
          $c_0 c_1$ ist Tangential an die Bezierkurve am Anfang,
          $c_2 c_3$ ist Tangential an die Bezierkurve am Ende.
    \item \textbf{Wertebereich}: Bézierkurven liegen innerhalb der konvexen
          Hülle, die durch die 4~Kontrollpunkte gebildet werden.
    \item \textbf{Endpunktinterpolation}: Bézierkurven beginnen immer beim
          ersten Kontrollpunkt und enden beim letzten Kontrollpunkt.
    \item \textbf{Variationsredukion}: Eine Bézierkurve $F$ wackelt nicht stärker
          als ihr Kontrollpolygon $B$ ($\sharp (H \cap F) \leq \sharp (H \cap B)$).
    \item \textbf{Affine Invarianz}
\end{itemize}

\subsection*{Teilaufgabe 10c}
\begin{figure}[h]
    \centering
    \includegraphics*[width=0.7\linewidth, keepaspectratio]{10c-bezier.png}
    \caption{Skizze zu Aufgabe 10c}
    \label{fig:10c}
\end{figure}

\end{document}
