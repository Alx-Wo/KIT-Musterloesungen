\documentclass[a4paper]{scrartcl}
\usepackage[ngerman]{babel}
\usepackage[utf8]{inputenc}
\usepackage{amssymb,amsmath}
\usepackage{graphicx}
\usepackage[inline]{enumitem}
\setlist{noitemsep}
\usepackage[binary-units=true]{siunitx}
\usepackage{hyperref}
\usepackage{parskip}
\usepackage[nameinlink,noabbrev,ngerman]{cleveref} % has to be after hyperref
\usepackage{nicefrac}
\usepackage{csquotes}
\usepackage{booktabs}  % for \toprule, \midrule and \bottomrule

\usepackage{minted} % needed for the inclusion of source code

\setcounter{secnumdepth}{2}
\setcounter{tocdepth}{2}

\usepackage{microtype}

\begin{document}
\selectlanguage{ngerman}
\title{2014 Nachklausur (WS 2013/14)}

\setcounter{section}{1}
%%%%%%%%%%%%%%%%%%%%%%%%%%%%%%%%%%%%%%%%%%%%%%%%%%%%%%%%%%%%%%%%%%%%%%%%%%%%%%
\section*{Aufgabe 1: Das Phong-Beleuchtungsmodell}
\subsection*{Teilaufgabe 1a}
\begin{figure}[h]
    \centering
    \includegraphics*[width=0.8\linewidth, keepaspectratio]{1a.png}
    \caption{Skizze zu Aufgabe 1}
    \label{fig:1a}
\end{figure}

\subsection*{Teilaufgabe 1b}
\textit{Geben Sie mit Hilfe der benannten Vektoren die Formel an, mit der die
Farbe der Oberfläche im Punkt $P_1$ bestimmt werden kann! Unterstreichen Sie
dabei die Vektoren, die normiert sein müssen!}

\[I = \sum_{i=1}^2 \overbrace{k_a \cdot I_{L_i}}^{=0} + k_d \cdot I_{L_i} \cdot (\underbrace{N \cdot L_{i}}_{\cos \Theta({L_i})}) + \overbrace{k_s \cdot I_{L_i} \cdot (R_{L_i} \cdot V)^n}^{=0}\]

Normalisierte Vektoren:

\begin{itemize}
    \item TODO: Welche Vektoren müssen normalisiert sein? Nicht alle?
\end{itemize}


\subsection*{Teilaufgabe 1c}
\textit{Welche der beiden Lichtquellen trägt mehr zur Intensität der Oberfläche
an $P_1$ bei?}

$L_1$ trägt mehr zur Intensität $I$ von $P_1$ bei, weil $\cos \Theta(L_1)$ größer
ist als  $\cos \Theta(L_2)$.

\subsection*{Teilaufgabe 1d}
\textit{Wie verändert sich die Intensität von $P_1$, wenn er statt von $A_1$
nun von $A_2$ betrachtet wird?}

$I$ ändert sich nicht, weil keiner der Terme die zu $I$ in diesem Fall beitragen
vom $A_1$ bzw. $A_2$ abhängig ist.

\subsection*{Teilaufgabe 1e}
\textit{Es soll nun die Farbe des Punktes $P_2$ bestimmt werden. Dieser
befindet sich auf der glänzenden Oberfläche. Zeichnen Sie für $P_2$ die
Vektoren ein, die zur Bestimmung seiner Farbe benötigt werden! Beschränken Sie
sich dabei auf die Vektoren für $L_1$.}

TODO

\subsection*{Teilaufgabe 1f}
\textit{Wie verändert sich die Intensität von $P_2$, wenn er statt von $A_1$
von $A_2$ betrachtet wird? Begründen Sie Ihre Antwort kurz!}

TODO

\section*{Aufgabe 2: Raytracing}
\subsection*{Teilaufgabe 2a}
\begin{itemize}
    \item Anstelle einen Punkt für einen Pixel abzutasten, tastet man
          $k^2$ mal in äquidistanten Intervallen ab.
    \item Aliasing wird dadurch verringert.
\end{itemize}

\subsection*{Teilaufgabe 2b}
\begin{itemize}
    \item Maximale Rekursionstiefe erreicht
    \item Rekursion bis der Beitrag zur Farbe vernachlässigbar wird
\end{itemize}

\subsection*{Teilaufgabe 2c}

\textit{Was ist der Unterschied zwischen Distributed Raytracing und Whitted-Style Raytracing?}

Beim Distributed Raytracing wird nicht nur ein Schattenstrahl verschickt, sondern
viele. Damit sollen zu perfekte Spiegelungen / Transmissionen vermieden werden.

\textit{Welchen Lichttransport kann man durch Distributed Raytracing berechnen, den
Whitted-Style Raytracing nicht erfassen kann?}

Siehe Kapitel~2 (Raytracing), Folie 147ff:

\begin{itemize}
    \item Kaustiken
    \item Weiche Schatten
    \item Tiefenunschärfe
    \item Bewegungsunschärfe
\end{itemize}

\subsection*{Teilaufgabe 2d}
\textit{Nennen Sie kurz und stichpunktartig die zwei Schritte, die zur Berechnung von Vertex-
Normalen bei einem Dreiecksnetz notwendig sind! Gehen Sie dabei davon aus, dass nur
die Vertex-Positionen und die Topologie des Netzes gegeben sind!}

TODO

\clearpage
\section*{Aufgabe 3: Farben und Farbwahrnehmung}
\subsection*{Teilaufgabe 3a}
\subsubsection*{Teilaufgabe 3a (I)}
\textit{Wie berechnet man die Sensorantwort $a$ für ein Spektrum $S(\lambda)$?}
\[a(S(\lambda)) = \int_\lambda E(\lambda) \cdot S(\lambda) \mathrm{d} \lambda \]

\subsubsection*{Teilaufgabe 3a (II)}
Unter einem \textit{Metamerismus} versteht man das Phänomen, das
unterschiedliche Spektren den selben Farbeindruck vermitteln können. Es muss
also
\[a_1 = a_2\]
gelten, damit $S_1(\lambda)$ und $S_2(\lambda)$ bzgl. der gegebenen Kamera
Metamere sind.

\subsection*{Teilaufgabe 3b}
\begin{enumerate}
    \item Das HSV-Farbmodell trennt Farbton von Helligkeit.
    \item[$\Rightarrow$] Richtig (\textit{H}ue (Farbton), \textit{S}aturation (Sättigung),
                         \textit{V}alue (Hellwert)).
    \item Der Farbeindruck einer additiv gemischten Farbe hängt nicht vom Farbeindruck der Ausgangsfarben ab.
    \item[$\Rightarrow$] Falsch. Vgl. Superpositionsprinzip, Kapitel 1, Teil 2, Folie 6.
    \item Farbige Flächen werden unabhängig von ihrer Umgebung vom menschlichen Auge immer gleich wahrgenommen.
    \item[$\Rightarrow$] Falsch. (TODO: Welche Folie?)
    \item Der Machsche Bandeffekt ist vor allem bei Phong-Shading ein Problem.
    \item[$\Rightarrow$] Falsch. Machsche Bänder entstehen an den Grenzen von Flächen, die jeweils keine Farbgraduierung haben. Beim Phong Shading gibt es wegen der interpolierten Normalen keine solchen Flächen. Der Effekt tritt z.B. beim Flat Shading auf, ist also vor allem dort problematisch. 
\end{enumerate}

\clearpage
\section*{Aufgabe 4: Bézier-Kurven}
\subsection*{Teilaufgabe 4a}
\textit{Gegeben sei die Bézier-Kurve $\mathbf{b}(u) = \sum_{i=0}^3 \mathbf{b}_i B_i^3(u)$ mit den Kontrollpunkten $\mathbf{b}_i$, wobei
$u \in [0, 1]$ und $B_i^3$ das $i$-te Bernstein-Polynom vom Grad 3 ist.}

\subsubsection*{Teilaufgabe 4a (I)}
\textit{Werten Sie die Bézier-Kurve zeichnerisch mit dem de-Casteljau-Algorithmus
an der Stelle $u = \nicefrac{1}{3}$ aus! Markieren Sie den Punkt $\mathbf{b}(\nicefrac{1}{3})$!}

\begin{figure}[h]
    \centering
    \includegraphics*[width=0.8\linewidth, keepaspectratio]{4ab.png}
    \caption{Skizze zu Aufgabe 4a und 4b}
    \label{fig:4ab}
\end{figure}

\subsubsection*{Teilaufgabe 4a (II)}
vgl \cref{fig:4ab}

\subsection*{Teilaufgabe 4b}
Siehe Nachklausur 2015, Aufgabe 11b für eine detailierte Erklärung.

\begin{enumerate}
    \item Nein, da die Kontrollpunkte auf den Ecken eines Rechtecks liegen,
          aber die Kurve nicht symmetrisch ist.
    \item Nein, da die Kurve nicht in der konvexen Hülle der Kontrollpunkte
          liegt.
    \item Ja
    \item Nein, da die Kurve nicht tangential an $b_0 b_1$ ist.
\end{enumerate}

\section*{Aufgabe 5: Transformationen}
% Gesucht ist eine homogene Matrix $M \in \mathbb{R}^{4 \times 4}$, für die gilt:

% \begin{align}
%     M \cdot \begin{pmatrix}-2\\1\\-1\\1\end{pmatrix} &= \begin{pmatrix}0\\0\\0\\c\end{pmatrix}\\

% \end{align}

Der Basiswechsel ist eine Verschiebung um $\begin{pmatrix}2\\-1\\1\end{pmatrix}$
und dann eine Rotation um $180^\circ$ um die $x$-Achse. Daher:

\begin{align}
    M &= \overbrace{\begin{pmatrix}1 & 0 & 0 & 0\\
                        0 & \cos 180^\circ & -\sin 180^\circ & 0\\
                        0 & \sin 180^\circ & \cos 180^\circ  & 0\\
                        0 &              0 &              0  & 1\end{pmatrix}}^{\text{Rotation}}
         \cdot
         \overbrace{
         \begin{pmatrix}1 & 0 & 0 & 2\\
                        0 & 1 & 0 & -1\\
                        0 & 0 & 1 & 1\\
                        0 & 0 & 0 & 1\end{pmatrix}}^{\text{Translation}}\\
    M &= \begin{pmatrix}1 &  0 &  0 & 0\\
                        0 & -1 &  0 & 0\\
                        0 &  0 & -1 & 0\\
                        0 &  0 &  0 & 1\end{pmatrix}
         \cdot
         \begin{pmatrix}1 & 0 & 0 & 2\\
                        0 & 1 & 0 & -1\\
                        0 & 0 & 1 & 1\\
                        0 & 0 & 0 & 1\end{pmatrix}\\
    M &= \begin{pmatrix}1 &  0 &  0 & 2\\
                        0 & -1 &  0 & 1\\
                        0 &  0 & -1 & -1\\
                        0 &  0 &  0 & 1\end{pmatrix}
\end{align}

\section*{Aufgabe 6: Texturierung}
\subsection*{Teilaufgabe 6a}
\begin{align}
    \lambda_A &= \frac{A_\Delta(P,B,C)}{A_\Delta(A,B,C)} = \frac{3}{6}
    &\lambda_B &= \frac{A_\Delta(P,A,C)}{A_\Delta(A,B,C)} = \frac{1}{6}
    &\lambda_C &= \frac{A_\Delta(P,A,B)}{A_\Delta(A,B,C)} = \frac{2}{6}
\end{align}

\begin{equation}
\begin{pmatrix}u\\v\end{pmatrix} = \lambda_A \cdot \begin{pmatrix}4\\0\end{pmatrix} + \lambda_B \cdot \begin{pmatrix}12\\6\end{pmatrix} + \lambda_C \cdot \begin{pmatrix}0\\12\end{pmatrix} = \begin{pmatrix}4\\5\end{pmatrix}
\end{equation}


\subsection*{Teilaufgabe 6b}
Siehe Kapitel~4, Folie~56

\textit{Gegeben ist eine Textur mit $8\times8$ Texeln und 4 Mipmap-Stufen sowie ein
Pixel-Footprint und dessen Mittelpunkt (siehe Abbildung). Kreuzen Sie in der
Mipmap-Pyramide die Texel an, die für eine trilineare Interpolation zur
Bestimmung des Farbwertes verwendet werden! Begründen Sie kurz Ihre Wahl der
Mipmap-Stufen!}

\includegraphics*[width=0.8\linewidth, keepaspectratio]{1a.png}

TODO


\subsection*{Teilaufgabe 6c}
\textit{Welchen Vorteil haben Summed-Area-Tables gegenüber Mipmaps bei der Texturfilterung?}
Die Textur kann mit einem beliebigen rechteckigen Filter in konstanter Zeit gefiltert werden. Kapitel 4, Folie 66.

\subsection*{Teilaufgabe 6d}
Das Interpolationsschema heißt \textit{Bilineare Interpolation} und funktionert
wie folgt:

\begin{align}
    t_{12} &= (1-a) \cdot t_1 + a \cdot t_2\\
    t_{34} &= (1-a) \cdot t_3 + a \cdot t_4\\
    t      &= (1-b) \cdot t_{34} + b \cdot t_{12}
\end{align}

\section*{Aufgabe 7: Cube-Maps und Environment-Mapping}
\subsection*{Teilaufgabe 7a}
\subsubsection*{Teilaufgabe 7a (I)}
\textit{Wie wird die Cube-Map-Seite bestimmt, auf die zugegriffen wird? Welche ist es für $\mathbf{r}$?}
Es wird die betragsmäßig größte Komponente gewählt. Diese bestimmt ob es
oben/unten oder links/rechts oder vorne/hinten wird. Das Vorzeichen bestimmt
dann per Konvention die konkrete Fläche.

Im vorliegenden Fall ist $|r_z|$ am größten, also ist es Fläche~1.

\subsubsection*{Teilaufgabe 7a (II)}
\textit{Berechnen Sie die Texturkoordinaten des Zugriffs auf der für $\mathbf{r}$ ausgewählten Cube-Map-Seite!}

\begin{align}
    s &= \nicefrac{1}{2} + \frac{r_x}{2 \cdot r_z} = \frac{1}{4}\\
    t &= \nicefrac{1}{2} + \frac{r_y}{2 \cdot r_z} = \frac{3}{4}
\end{align}

\subsubsection*{Teilaufgabe 7a (III)}
Vorteile von Cube-Maps gegenüber Sphere-Maps (\href{https://en.wikipedia.org/wiki/Cube_mapping}{Quelle}):
\begin{itemize}
    \item Keine Bildverzerrung
    \item Unabhängigkeit vom Viewpoint $P$
    \item Schneller Berechenbar
    \item[$\Rightarrow$] Besser für Echtzeit-Rendering geeignet.
\end{itemize}


\subsection*{Teilaufgabe 7b}
\subsubsection*{Teilaufgabe 7b (I)}
\textit{Was wird in einer Environment-Map gespeichert?}

Ein Bild der Umgebung in einer Textur.

\subsubsection*{Teilaufgabe 7b (II)}
\textit{Nennen Sie ein Anwendungsbeispiel für Environment-Maps!}
Terminator (Reflektion auf dem Terminator im Film, vgl. Kapitel~4, Folie~97)

\subsubsection*{Teilaufgabe 7b (III)}
\textit{Welche grundlegende Annahme wird bei Environment-Mapping gemacht?}
Das Environment ist unendlich weit weg (also: nur die Richtung $r$ wird verwendet,
nicht jedoch der Ausgangspunkt $P$).

\subsubsection*{Teilaufgabe 7b (IV)}
\textit{Was bzw. welcher Effekt kann mit vorgefilterten Environment-Maps nicht korrekt dargestellt werden?}

Selbstverschattung


\section*{Aufgabe 8: Hierarchische Datenstrukturen}
\subsection*{Teilaufgabe 8a}

Reihenfolge der Schnittests:

A B 14 15 A C 5 6 11

\subsection*{Teilaufgabe 8b}
\begin{enumerate}
    \item räumliches Mittel (Spatial Median)
    \item[$\rightarrow$] Nein, da $B$ nicht in der Mitte liegt
    \item Objektmittel (Object Median)
    \item[$\rightarrow$] Nein, da $B$ genau 8~Objekte auf der einen Seite und
                         nur 2~Objekte auf der anderen Seite hat.
    \item Kostenfunktion (Surface Area Heuristic)
    \item[$\rightarrow$] Ja.
\end{enumerate}

\subsection*{Teilaufgabe 8c}
\textit{Nennen Sie jeweils eine Stärke und eine Schwäche der Aufteilung mittels Kostenfunktion (Surface Area Heuristic)!}

\begin{itemize}
    \item[Vorteil] Gut konstruierte kD-Bäume können deutlich schneller sein als schlecht konstruierte
    \item[Nachteil] Konstruktion aufwendig
\end{itemize}


\subsection*{Teilaufgabe 8d}
\begin{tabular}{p{4cm}p{2.2cm}p{2.2cm}p{2.2cm}p{2.2cm}}\toprule
Aussage                                                                 & BVH             & Octree               & kD-Baum               & Gitter               \\\midrule
Datenstruktur wird an Geometrie angepasst                               & Ja (Hüllkörper) & Ja (Rekursionstiefe) & Ja (wo Ebenen liegen) & Ja (Rekursionstiefe) \\
Raum wird immer achsen\-parallel unterteilt                             & Nein            & Ja                   & Ja                    & Ja                   \\
Datenstruktur ist Binär\-baum\footnotemark                              & Nein            & Nein                 & Ja                    & Nein                 \\
Speicherplatz der Daten\-struktur ist abhängig von der Anzahl der Primive & Ja            & Ja                   & Ja                    & Ja                   \\
Bei der Konstruktuion kann die SAH sinnvoll eingesetz werden            & Ja\footnotemark & Nein                 & Ja                    & Nein                 \\\bottomrule
\end{tabular}
\footnotetext{Üblicherweise ist ein BVH ein Binärbaum, muss aber nicht so sein.}
\footnotetext{Foliensatz 5 Folie 98}

\section*{Aufgabe 9: Rasterisierung und OpenGL}
\begin{table}[H]
    \begin{tabular}{cp{12cm}ll}
    \toprule
    \# & \textbf{Aussage} & \textbf{Wahr} & \textbf{Falsch} \\\midrule
    1  & Die Präzision des Tiefenpuffers wird verringert, wenn man die Distanz zwischen Near-Plane und Far-Plane vergrößert.                                    & X &  \\
    2  & Die OpenGL-Pipeline nutzt den Vertex-Cache beim Zeichnen ohne Index-Puffer. & & X   \\
    3  & OpenGL-Puffer vom Typ \texttt{GL\_ELEMENT\_ARRAY\_BUFFER} werden für das Rendering mit der Shared-Vertex-Repräsentation verwendet.                            & X &  \\
    4  & Ein Alpha-Test kann im Fragment-Shader mit dem Befehl \texttt{discard} implementiert werden.                                                                    & X & \\
    5  & Die OpenGL-Funktion \texttt{glBlendFunc(GL\_SRC\_ALPHA, GL\_ONE\_MINUS\_SRC\_ALPHA)} kann verwendet werden, um additives Blending zu implementieren.           & & X \\
    6  & Shadow-Mapping benötigt den Stencil-Puffer.                                                                                                             & & X \\
    7  & Für die Transformation einer Normalen im Vertex-Shader kann immer dieselbe Matrix wie zur Transformation der entsprechenden Vertizes verwendet werden. & & X \\
    8  & Beim Gouraud-Shading werden die im Vertex-Shader berechneten Farben pro Fragment linear interpoliert.                                                   & X & \\\bottomrule
    \end{tabular}
\end{table}

\section*{Aufgabe 10: Tiefenpuffer und Transparenz}
\subsection*{Teilaufgabe 10a}
\textit{In welcher Reihenfolge müssen die drei Funktionen aufgerufen werden, um
eine korrekte Darstellung des Hauses zu erhalten?}

\begin{enumerate}
    \item Lösche Tiefenpuffer
    \item Zeichne Wände
    \item Zeichne Fenster
\end{enumerate}

\subsection*{Teilaufgabe 10b}

\begin{table}[H]
    \begin{tabular}{lllll}
    \toprule
    Sortierung     & keine & vorne nach hinten & hinten nach vorne & Begründung \\\midrule
    ZeichneWände   & TODO  & TODO              & TODO              & TODO       \\
    ZeichneFenster & TODO  & TODO              & TODO              & TODO       \\\bottomrule
    \end{tabular}
\end{table}

\subsection*{Teilaufgabe 10c}
\begin{table}[H]
    \begin{tabular}{lllllll}\toprule
    Sortierung     & Tiefentest & EQUAL & LESS & GREATER & Tiefe schreiben & Blending \\\midrule
    ZeichneWände   & TODO       & TODO  & TODO & TODO    & TODO            & TODO    \\
    ZeichneFenster & TODO       & TODO  & TODO & TODO    & TODO            & TODO   \\\bottomrule
    \end{tabular}
\end{table}

\section*{Aufgabe 11: Phong-Shading und Phong-Beleuchtungsmodell}
\subsection*{Teilaufgabe 11a}
\inputminted[linenos, numbersep=5pt, tabsize=4, frame=lines, label=shader.vert]{glsl}{shader.vert}

\subsection*{Teilaufgabe 11b}
\inputminted[linenos, numbersep=5pt, tabsize=4, frame=lines, label=shader.frag]{glsl}{shader.frag}

\section*{Aufgabe 12: Deformation mit Skelettsystemen}
\inputminted[linenos, numbersep=5pt, tabsize=4, frame=lines, label=skelett.vert]{glsl}{skelett.vert}

\end{document}
