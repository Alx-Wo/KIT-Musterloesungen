\documentclass[a4paper]{scrartcl}
\usepackage[ngerman]{babel}
\usepackage[utf8]{inputenc}
\usepackage{amssymb,amsmath}
\usepackage{graphicx}
\usepackage[inline]{enumitem}
\setlist{noitemsep}
\usepackage[binary-units=true]{siunitx}
\usepackage{hyperref}
\usepackage{parskip}
\usepackage[nameinlink,noabbrev,ngerman]{cleveref} % has to be after hyperref
\usepackage[colorinlistoftodos]{todonotes}
\usepackage{nicefrac}
\usepackage{csquotes}

\usepackage{minted} % needed for the inclusion of source code

\setcounter{secnumdepth}{2}
\setcounter{tocdepth}{2}

\usepackage{microtype}

\begin{document}
\selectlanguage{ngerman}
\title{2014 Nachklausur (WS 2013/14)}

\setcounter{section}{1}
%%%%%%%%%%%%%%%%%%%%%%%%%%%%%%%%%%%%%%%%%%%%%%%%%%%%%%%%%%%%%%%%%%%%%%%%%%%%%%
\section*{Aufgabe 1: Das Phong-Beleuchtungsmodell}
\subsection*{Teilaufgabe 1a}
\begin{figure}[h]
    \centering
    \includegraphics*[width=0.8\linewidth, keepaspectratio]{1a.png}
    \caption{Skizze zu Aufgabe 1}
    \label{fig:1a}
\end{figure}

\subsection*{Teilaufgabe 1b}
TODO
\subsection*{Teilaufgabe 1c}
TODO
\subsection*{Teilaufgabe 1d}
TODO
\subsection*{Teilaufgabe 1e}
TODO
\subsection*{Teilaufgabe 1f}
TODO

\section*{Aufgabe 2: Raytracing}
\subsection*{Teilaufgabe 2a}
\begin{itemize}
    \item Anstelle einen Punkt für einen Pixel abzutasten, tastet man
          $k^2$ mal in äquidistanten Intervallen ab.
    \item Aliasing wird dadurch verringert.
\end{itemize}

\subsection*{Teilaufgabe 2b}
\begin{itemize}
    \item Maximale Rekursionstiefe erreicht
    \item Rekursion bis der Beitrag zur Farbe vernachlässigbar wird
\end{itemize}

\subsection*{Teilaufgabe 2c}

\textit{Was ist der Unterschied zwischen Distributed Raytracing und Whitted-Style Raytracing?}
TODO

\textit{Welchen Lichttransport kann man durch Distributed Raytracing berechnen, den
Whitted-Style Raytracing nicht erfassen kann?}
TODO

\subsection*{Teilaufgabe 2d}
\textit{Nennen Sie kurz und stichpunktartig die zwei Schritte, die zur Berechnung von Vertex-
Normalen bei einem Dreiecksnetz notwendig sind! Gehen Sie dabei davon aus, dass nur
die Vertex-Positionen und die Topologie des Netzes gegeben sind!}
TODO

\clearpage
\section*{Aufgabe 3: Farben und Farbwahrnehmung}
\subsection*{Teilaufgabe 3a}
\subsubsection*{Teilaufgabe 3a (I)}
\textit{Wie berechnet man die Sensorantwort $a$ für ein Spektrum $S(\lambda)$?}
\[a(S(\lambda)) = \int_\lambda E(\lambda) \cdot S(\lambda) \mathrm{d} \lambda \]

\subsubsection*{Teilaufgabe 3a (II)}
Unter einem \textit{Metamerismus} versteht man das Phänomen, das
unterschiedliche Spektren den selben Farbeindruck vermitteln können. Es muss
also
\[a_1 = a_2\]
gelten, damit $S_1(\lambda)$ und $S_2(\lambda)$ bzgl. der gegebenen Kamera
Metamere sind.

\subsection*{Teilaufgabe 3b}
\begin{enumerate}
    \item Das HSV-Farbmodell trennt Farbton von Helligkeit.
    \item[$\Rightarrow$] Richtig (\textit{H}ue (Farbton), \textit{S}aturation (Sättigung),
                         \textit{V}alue (Hellwert)).
    \item Der Farbeindruck einer additiv gemischten Farbe hängt nicht vom Farbeindruck der Ausgangsfarben ab.
    \item[$\Rightarrow$] TODO
    \item Farbige Flächen werden unabhängig von ihrer Umgebung vom menschlichen Auge immer gleich wahrgenommen.
    \item[$\Rightarrow$] Falsch. (TODO: Welche Folie?)
    \item Der Machsche Bandeffekt ist vor allem bei Phong-Shading ein Problem.
    \item[$\Rightarrow$] TODO
\end{enumerate}

\clearpage
\section*{Aufgabe 4: Bézier-Kurven}
\subsection*{Teilaufgabe 4a}
\textit{Gegeben sei die Bézier-Kurve $\mathbf{b}(u) = \sum_{i=0}^3 \mathbf{b}_i B_i^3(u)$ mit den Kontrollpunkten $\mathbf{b}_i$, wobei
$u \in [0, 1]$ und $B_i^3$ das $i$-te Bernstein-Polynom vom Grad 3 ist.}

\subsubsection*{Teilaufgabe 4a (I)}
\textit{Werten Sie die Bézier-Kurve zeichnerisch mit dem de-Casteljau-Algorithmus
an der Stelle $u = \nicefrac{1}{3}$ aus! Markieren Sie den Punkt $\mathbf{b}(\nicefrac{1}{3})$!}

\begin{figure}[h]
    \centering
    \includegraphics*[width=0.8\linewidth, keepaspectratio]{4ab.png}
    \caption{Skizze zu Aufgabe 4a und 4b}
    \label{fig:4ab}
\end{figure}

\subsubsection*{Teilaufgabe 4a (II)}
vgl \cref{fig:4ab} (da bin ich mir aber unsicher, ob das stimmt).

\subsection*{Teilaufgabe 4b}
Siehe Nachklausur 2015, Aufgabe 11b für eine detailierte Erklärung.

\begin{enumerate}
    \item Nein, da die Kontrollpunkte auf den Ecken eines Rechtecks liegen,
          aber die Kurve nicht symmetrisch ist.
    \item Nein, da die Kurve nicht in der konvexen Hülle der Kontrollpunkte
          liegt.
    \item Ja
    \item Nein, da die Kurve nicht tangential an $b_0 b_1$ ist.
\end{enumerate}

\section*{Aufgabe 5: Transformationen}
% Gesucht ist eine homogene Matrix $M \in \mathbb{R}^{4 \times 4}$, für die gilt:

% \begin{align}
%     M \cdot \begin{pmatrix}-2\\1\\-1\\1\end{pmatrix} &= \begin{pmatrix}0\\0\\0\\c\end{pmatrix}\\

% \end{align}

Der Basiswechsel ist eine Verschiebung um $\begin{pmatrix}2\\-1\\1\end{pmatrix}$
und dann eine Rotation um $180^\circ$ um die $x$-Achse. Daher:

\begin{align}
    M &= \overbrace{\begin{pmatrix}1 & 0 & 0 & 0\\
                        0 & \cos 180^\circ & -\sin 180^\circ & 0\\
                        0 & \sin 180^\circ & \cos 180^\circ  & 0\\
                        0 &              0 &              0  & 1\end{pmatrix}}^{\text{Rotation}}
         \cdot
         \overbrace{
         \begin{pmatrix}1 & 0 & 0 & 2\\
                        0 & 1 & 0 & -1\\
                        0 & 0 & 1 & 1\\
                        0 & 0 & 0 & 1\end{pmatrix}}^{\text{Translation}}\\
    M &= \begin{pmatrix}1 &  0 &  0 & 0\\
                        0 & -1 &  0 & 0\\
                        0 &  0 & -1 & 0\\
                        0 &  0 &  0 & 1\end{pmatrix}
         \cdot
         \begin{pmatrix}1 & 0 & 0 & 2\\
                        0 & 1 & 0 & -1\\
                        0 & 0 & 1 & 1\\
                        0 & 0 & 0 & 1\end{pmatrix}\\
    M &= \begin{pmatrix}1 &  0 &  0 & 2\\
                        0 & -1 &  0 & 1\\
                        0 &  0 & -1 & -1\\
                        0 &  0 &  0 & 1\end{pmatrix}
\end{align}

\section*{Aufgabe 6: Texturierung}
\subsection*{Teilaufgabe 6a}
\begin{align}
    \lambda_A &= \frac{A_\Delta(P,B,C)}{A_\Delta(A,B,C)} = \frac{3}{6}
    &\lambda_B &= \frac{A_\Delta(P,A,C)}{A_\Delta(A,B,C)} = \frac{1}{6}
    &\lambda_C &= \frac{A_\Delta(P,A,B)}{A_\Delta(A,B,C)} = \frac{2}{6}
\end{align}

\begin{equation}
\begin{pmatrix}u\\v\end{pmatrix} = \lambda_A \cdot \begin{pmatrix}4\\0\end{pmatrix} + \lambda_B \cdot \begin{pmatrix}12\\6\end{pmatrix} + \lambda_C \cdot \begin{pmatrix}0\\12\end{pmatrix} = \begin{pmatrix}4\\5\end{pmatrix}
\end{equation}


\subsection*{Teilaufgabe 6b}
Siehe Kapitel~4, Folie~56

TODO


\subsection*{Teilaufgabe 6c}
\textit{Welchen Vorteil haben Summed-Area-Tables gegenüber Mipmaps bei der Texturfilterung?}
TODO

\subsection*{Teilaufgabe 6d}
Das Interpolationsschema heißt \textit{Bilineare Interpolation} und funktionert
wie folgt:

\begin{align}
    t_{12} &= (1-a) \cdot t_1 + a \cdot t_2\\
    t_{23} &= (1-a) \cdot t_3 + a \cdot t_4\\
    t      &= (1-b) \cdot t_{12} + b \cdot t_{23}
\end{align}

\section*{Aufgabe 7: Cube-Maps und Environment-Mapping}
\subsection*{Teilaufgabe 7a}
\subsubsection*{Teilaufgabe 7a (I)}
\textit{Wie wird die Cube-Map-Seite bestimmt, auf die zugegriffen wird? Welche ist es für $\mathbf{r}$?}
Es wird die betragsmäßig größte Komponente gewählt. Diese bestimmt ob es
oben/unten oder links/rechts oder vorne/hinten wird. Das Vorzeichen bestimmt
dann per Konvention die konkrete Fläche.

Im vorliegenden Fall ist $|r_z|$ am größten, also ist es Fläche~1.

\subsubsection*{Teilaufgabe 7a (II)}
\textit{Berechnen Sie die Texturkoordinaten des Zugriffs auf der für $\mathbf{r}$ ausgewählten Cube-Map-Seite!}

\begin{align}
    s &= \nicefrac{1}{2} + \frac{r_x}{2 \cdot r_z} = \frac{1}{4}\\
    t &= \nicefrac{1}{2} + \frac{r_y}{2 \cdot r_z} = \frac{3}{4}
\end{align}

\subsubsection*{Teilaufgabe 7a (III)}
\textit{Nennen Sie einen Vorteil von Cube-Maps gegenüber Sphere-Maps!}

TODO

\subsection*{Teilaufgabe 7b}
\subsubsection*{Teilaufgabe 7b (I)}
\textit{Was wird in einer Environment-Map gespeichert?}

Ein Bild der Umgebung in einer Textur.

\subsubsection*{Teilaufgabe 7b (II)}
\textit{Nennen Sie ein Anwendungsbeispiel für Environment-Maps!}
TODO

\subsubsection*{Teilaufgabe 7b (III)}
\textit{Welche grundlegende Annahme wird bei Environment-Mapping gemacht?}
Das Environment ist unendlich weit weg (also: nur die Richtung $r$ wird verwendet,
nicht jedoch der Ausgangspunkt $P$).


\section*{Aufgabe 8: Hierarchische Datenstrukturen}
\subsection*{Teilaufgabe 8a}
TODO
\subsection*{Teilaufgabe 8b}
TODO
\subsection*{Teilaufgabe 8c}
TODO
\subsection*{Teilaufgabe 8d}
TODO

\section*{Aufgabe 9: Rasterisierung und OpenGL}
TODO

\section*{Aufgabe 10: Tiefenpuffer und Transparenz}
\subsection*{Teilaufgabe 10a}
TODO
\subsection*{Teilaufgabe 10b}
TODO
\subsection*{Teilaufgabe 10c}
TODO

\section*{Aufgabe 11: Phong-Shading und Phong-Beleuchtungsmodell}
\inputminted[linenos, numbersep=5pt, tabsize=4, frame=lines, label=shader.vert]{glsl}{shader.vert}


\end{document}
