\documentclass[a4paper]{scrartcl}
\usepackage[ngerman]{babel}
\usepackage[utf8]{inputenc}
\usepackage{amssymb,amsmath}
\usepackage{graphicx}
\usepackage[inline]{enumitem}
\setlist{noitemsep}
\usepackage[binary-units=true]{siunitx}
\usepackage{hyperref}
\usepackage{parskip}
\usepackage[nameinlink,noabbrev,ngerman]{cleveref} % has to be after hyperref
\usepackage{nicefrac}  % for \nicefrac{1}{3}
\usepackage{csquotes}  % for \enquote{what you want to quote}
\usepackage{booktabs}  % for \toprule, \midrule and \bottomrule
\usepackage{minted} % needed for the inclusion of source code

% for \begin{enumerate}[label=(\Alph*)], see http://tex.stackexchange.com/a/129960/5645
\usepackage{enumitem}

\setcounter{secnumdepth}{2}
\setcounter{tocdepth}{2}

\usepackage{wasysym}  % For \CheckedBox
\usepackage{microtype}

\begin{document}
\selectlanguage{ngerman}
\title{2012 Hauptklausur (WS 2011/12)}

\setcounter{section}{1}
%%%%%%%%%%%%%%%%%%%%%%%%%%%%%%%%%%%%%%%%%%%%%%%%%%%%%%%%%%%%%%%%%%%%%%%%%%%%%%
\section*{Aufgabe 1: Wahrnehmung und Farbräume}
\subsection*{Teilaufgabe 1a}
\textit{Eine Grafikkarte ist an ein Anzeigegerät mit einem Gamma-Wert von 2.0
angeschlossen und muss eine entsprechende Gamma-Korrektur durchführen.
Berechnen Sie den Intensitätswert , den die Grafikkarte an das Anzeigegerät
senden muss, um eine Ausgabe mit der Hälfte der Maximalintensität zu erreichen.
(Der Wertebereich der Koeffzienten reicht von 0 bis zur
 Maximalintensität~1.0.)}

Es gilt

\[I_{\text{out}} = I_{\text{in}}^\gamma\]

für $I_{\text{out}} = \nicefrac{1}{2}$ und $\gamma = 2$ muss also gelten:

\[I_{\text{in}} = \frac{1}{\sqrt{2}}\]


\subsection*{Teilaufgabe 1b}
\textit{Sie haben ein Bild im RGB-Farbraum gegeben und wollen den Helligkeitskontrast erhöhen. In welchen der in der Vorlesung vorgestellten Farbräume wandeln Sie es um,
um diese Kontrasterhöhung möglichst einfach durchführen zu können? Welche Berechnung(en) führen Sie dazu auf den Koeffizienten dieses Farbraums aus?}

HSV (oder HSI, HSL). Dann wird einfach der V-Wert (I-Wert, L-Wert) erhöht.

\subsection*{Teilaufgabe 1c}
\begin{tabular}{cp{8cm}llp{4cm}}\toprule
\# & Aussage                                                                                                     & Wahr           & Falsch           & Begründung            \\\midrule
 1 & Um den Farbeindruck für einen Menschen eindeutig zu beschreiben, genügt ein Farbmodell mit 3 Koeffizienten. & \CheckedBox    & $\square$        & Graßmansche Gesetze   \\
 2 & Durch diese 3 Koeffizienten ist dann das Spektrum ebenso eindeutig festgelegt.                              & $\square$      & \CheckedBox      & Metamerie             \\
 3 & Der RGB-Einheitswürfel enthält alle sichtbaren Farben.                                                      & $\square$      & \CheckedBox      & Magenta / Purple-Line \\
 4 & Der RGB-Einheitswürfel enthält Farben, die sich im CIE XYZ-Farbmodell nicht darstellen lassen.              & $\square$      & \CheckedBox      & Es ist umgekehrt  \\\bottomrule
\end{tabular}

\section*{Aufgabe 2: Raytracing und prozedurale Modelle}
\subsection*{Teilaufgabe 2a}
\begin{enumerate}[label=(\arabic*)]
    \item Schattenstrahl zu Lichtquelle verschießen um Verschattung zu testen
    \item Grundfarbwert anhand eines Beleuchtungsmodells (z.B. Phong) bestimmen
    \item Reflektionsstrahl verschießen und ermittelten Farbwert auf Farbwert
          im Punkt addieren.
    \item Refraktionsstrahl verschießen.
\end{enumerate}

\subsection*{Teilaufgabe 2b}
\textit{Was ist der konzeptuelle Unterschied zwischen Raymarching und
Raycasting bzw. Raytracing? Nennen Sie einen Anwendungsfall, bei dem Sie
Raymarching verwenden würden, und begründen Sie, warum Sie für diesen Fall
Raytracing können}

Bei \textbf{Raymarching} wird durch jeden Pixel ein Strahl geschossen und mit
einer bestimmten Größe \enquote{abmarschiert}. Im Gegensatz zu Raytracing
werden keine Sekundärstrahlen verschossen.

Raymarching wird zur Verarbeitung von Volumendaten (Nebel, Wolken, \dots)
eingesetzt, wobei Raytracing und Raycasting für Oberflächen verwendet wird.

\textbf{Anwendungsfall für Raymarching}: Hypertextures. Die Berechnung mittels
Raytracing wäre zu aufwendig, da es sehr viele Sekundärstrahlen gäbe.

TODO: Unterschied zwischen Raycasting und Raytracing?

\subsection*{Teilaufgabe 2c}
\begin{tabular}{cp{8cm}llp{4cm}}\toprule
\# & Aussage                                                                                                                                                                                                       & Wahr & Falsch & Begründung \\\midrule
1  & Prozedurale Modelle erlauben eine kompakte Beschreibung von Texturen oder Objekten, aber die Resultate sind oft nur schwer zu kontrollieren.                                                                  & \CheckedBox    & $\square$        & ~          \\
2  & Eine ideale Noise-Funktion sollte weder bandbegrenzt sein noch räumliche Korrelationen aufweisen, um erkennbare Periodizitäten und monotone Strukturen zu vermeiden.                                          & $\square$      & \CheckedBox      & ~          \\
3  & Um Turbulenz-Texturen zu erstellen, werden immer höhere Oktaven von Noise-Texturen aufsummiert und dabei immer stärker gewichtet, damit die Resultate so weit wie möglich gegen Weißes Rauschen konvergieren. & $\square$      & \CheckedBox      & ~          \\
4  & Constructive Solid Geometry ist eine Technik zum Modellieren von Festkörpern, bei der Objekte durch boolesche Operatoren kombiniert werden.                                                                   & \CheckedBox    & $\square$        & ~          \\\bottomrule
\end{tabular}

\section*{Aufgabe 3: Transformationen}
\subsection*{Teilaufgabe 3a}
\textit{Um welchen Typ Transformation handelt es sich dabei?}

TODO (Basiswechsel?)

\textit{Nennen Sie die effizienteste Methode, die Inverse dieser Transformation zu finden.}

Man kann leicht feststellen, dass $u \times v = u \times w = v \times w = 0$.
Es handelt sich also um eine Orthogonalmatrix. Die inverse $M^{-1}$ einer
Orthogonalmatrix $M$ ist gleich ihrer Transponierten $M^T$: $M^{-1} = M^T$.

\subsection*{Teilaufgabe 3b}
\textit{Nennen Sie zwei Gründe für die Verwendung hierarchischer Modellierung.}

\begin{enumerate}[label=(\arabic*)]
    \item Anordnung von objekten in hierarchischen Gruppen
    \item Anwendung von Transformationen auf ganze Objektgruppen
    \item Natürliche Beschreibung eines Objekts / einer Szene
\end{enumerate}

\textit{Welche Datenstruktur haben Sie in der Vorlesung kennengelernt, die dabei hilft,
Transformationen beim Traversieren des Szenengraphen zu verwalten?}

Matrixstack



\subsection*{Teilaufgabe 3c}
(Homogene 2D-Matrizen)

\textit{Nennen Sie jeweils, um welche Transformation es sich handelt und wie deren Parameter sind.}

\begin{enumerate}[label=(\Alph*)]
    \item Rotation um $45^\circ$ gegen den Urzeigersinn.
    \item Translation um 2.5 in Richtung der $y$-Achse.
    \item Spiegelung an der $y$-Achse.
    \item Scherung um~1 in Richtung der $x$-Achse.
\end{enumerate}


\section*{Aufgabe 4: Texturen}
\subsection*{Teilaufgabe 4a}
\textit{Was ist eine Environment-Map?}

Eine Environment-Map ist eine Textur zur Darstellung der Umgebung.

\textit{Wofür wird sie eingesetzt?}

Durch eine Environment-Map kann die Reflektion/Beleuchtung eines Objekts
bestimmt werden, ohne aufwendiges Ray-Tracing zu betreiben.

\textit{Welche Annahmen trifft man dabei?}

Die Umgebung ist weit genug entfernt, sodass die Position keine Rolle spielt.
Es wird nur die Blickrichtung verwendet.

\subsection*{Teilaufgabe 4b}
\textit{Welches Problem beim Texture-Mapping löst Mip-Mapping? Erklären Sie kurz die Idee!}

Mip-Mapping ist eine Vorfilterung der Textur um Aliasing-Artefakten vorzubeugen.
Aliasing tritt durch Unterabtastung des Bildsignals auf.

Man erzeugt Mip-Maps durch \enquote{zusammenfassen} von jeweilse $2 \times 2$
Texeln. Es werden solange neue Mip-Map-Stufen generiert, bis man eine Mip-Map
erzeugt hat, die nur noch aus einem Texel besteht.

\subsection*{Teilaufgabe 4c: Bilineare Interpolation}

\begin{align}
    c'  &= (1-s) \cdot c_{01} + s \cdot c_{11}\\
    c'' &= (1-s) \cdot c_{00} + s \cdot c_{10}\\
    c   &= (1-t) \cdot c'' + t \cdot c'
\end{align}

\section*{Aufgabe 5: Hierarchische Datenstrukturen}
\subsection*{Teilaufgabe 5a}
\textit{Nennen Sie vier Arten von Hüllkörpern (bounding volumes), die Sie in
der Vorlesung kennengelernt haben, und sortieren Sie sie danach, wie eng sie
den in der Darstellung gegebenen Zylinder umschließen können.}

\begin{enumerate*}[label=(\Alph*)]
    \item Zylinder
    \item Slabs
    \item AABB
    \item Sphäre
\end{enumerate*}


\subsection*{Teilaufgabe 5b}
\textit{Für welche Datenstrukturen stimmt folgende Aussage? Begründen Sie.}

\enquote{Der erste während der Traversierung gefundene Schnittpunkt ist
zugleich der nächste Schnittpunkt zum Ursprung des Strahls.}

\textbf{Octree}: Falsch. Da keine Reihenfolge festgelegt ist in der der Baum
                 traversiert wird.

\textbf{BVH}: Falsch.

\textbf{kD-Baum}: Wahr. Die Traversierungsreihenfolge des Schnittalgorithmus
                  ist so festgelegt, dass immer die naheste Ebene zuerst
                  betrachtet wird.

                  Die Traversierungsreihenfolge ist also von vorne nach hinten.

\subsection*{Teilaufgabe 5c}
\includegraphics*[width=\linewidth, keepaspectratio]{5c.png}

\section*{Aufgabe 6: Projektive Transformationen und Clipping}
\subsection*{Teilaufgabe 6a}
\textit{Transformieren Sie die Vertizes $P_1$ , $P_2$
in homogene Clip-Koordinaten, und geben Sie die homogenen Koordinaten der
transformierten Punkte an.}

\begin{align}
      P_{1h} &= M_{proj} \begin{pmatrix}2\\6\\1\end{pmatrix} = \begin{pmatrix}2\\-2\\6\end{pmatrix}
    & P_{2h} &= M_{proj} \begin{pmatrix}8\\4\\1\end{pmatrix} = \begin{pmatrix}8\\-8\\4\end{pmatrix}
\end{align}

\subsection*{Teilaufgabe 6b}
\textit{Als nächstes müssen Sie die \enquote{window edge coordinates} der
projizierten Punkte in Clip- Koordinaten berechnen. Welche Kanten müssen
potentiell auf Schnitt mit der Strecke überprüft werden?}

\begin{align}
    WEC_{x=w}(P_1)  &= 4     & WEC_{x=w}(P_2)  &= -4\\
    WEC_{x=-w}(P_1) &= 8     & WEC_{x=-w}(P_2) &= 12\\
    WEC_{z=w}(P_1)  &= 8     & WEC_{z=w}(P_2)  &= 12\\
    WEC_{z=-w}(P_1) &= 4     & WEC_{z=-w}(P_2) &= -4\\
    WEC_{x=w}       &= w - x & WEC_{x=-w}      &= w + x
\end{align}

$\Rightarrow$ Untere rechte Kante überprüfen

\subsection*{Teilaufgabe 6c}
\textit{Wenden Sie nun $\alpha$-Clipping an. Geben Sie ebenfalls die Endpunkte
$P_1$ und $P_2$ der Teilstrecke in der Sichtpyramide in normalisierten
Device-Koordinaten an.}

\begin{align}
    a_{\text{min}} = 0\;\;\; a_{\text{max}} = 1\\
    a_1 = \frac{WEC_{x=w}(P_1)}{WEC_{x=w}(P_1) - WEC_{x=w}(P_2)} = \frac{4}{4 - (-4)} = 0.5\\
    \Rightarrow a_{\text{max}} = \min(a_{\text{max}}, a_1) = 0.5 \text{ (weil Optcode } P_2 \text{ gesetzt)}\\
    a_2 = \frac{WEC_{z=-w}(P_1)}{WEC_{z=-w}(P_1) - WEC_{z=-w}(P_2)} = \frac{4}{4 - (-4)} = 0.5\\
    \Rightarrow a_{\text{max}} = \min(a_{\text{max}}, a_2) = 0.5 \text{ (weil Optcode } P_2 \text{ gesetzt)}\\
\end{align}

Neues Liniensegment:

\begin{align}
    (P_1 + a_{\text{min}} \cdot (P_2 - P_1)&, P_1 + a_{\text{max}} \cdot (P_2 - P_1))\\
  = ((2, -2, 6)&, (2, -2, 6) + 0.5 \cdot (6, -10, 2))\\
  = ((2, -2, 6)&, (5, -7, 5))
\end{align}

\begin{align}
    P_1' &= (\nicefrac{1}{3}, -\nicefrac{1}{3}) & P_2' &= (1, -\nicefrac{7}{5})
\end{align}


\section*{Aufgabe 7}
TODO

\section*{Aufgabe 8}
\subsection*{Teilaufgabe 8a}
TODO

\subsection*{Teilaufgabe 8b}
\inputminted[linenos, numbersep=5pt, tabsize=4, frame=lines, label=shader.frag]{glsl}{shader.frag}

\section*{Aufgabe 9}
\subsection*{Teilaufgabe 9a}
TODO
\subsection*{Teilaufgabe 9b}
TODO
\subsection*{Teilaufgabe 9c}
TODO


\end{document}
