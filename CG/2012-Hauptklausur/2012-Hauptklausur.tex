\documentclass[a4paper]{scrartcl}
\usepackage[ngerman]{babel}
\usepackage[utf8]{inputenc}
\usepackage{amssymb,amsmath}
\usepackage{graphicx}
\usepackage[inline]{enumitem}
\setlist{noitemsep}
\usepackage[binary-units=true]{siunitx}
\usepackage{hyperref}
\usepackage{parskip}
\usepackage[nameinlink,noabbrev,ngerman]{cleveref} % has to be after hyperref
\usepackage{nicefrac}  % for \nicefrac{1}{3}
\usepackage{csquotes}  % for \enquote{what you want to quote}
\usepackage{booktabs}  % for \toprule, \midrule and \bottomrule
\usepackage{minted} % needed for the inclusion of source code

% for \begin{enumerate}[label=(\Alph*)], see http://tex.stackexchange.com/a/129960/5645
\usepackage{enumitem}

\setcounter{secnumdepth}{2}
\setcounter{tocdepth}{2}

\usepackage{wasysym}  % For \CheckedBox
\usepackage{microtype}

\begin{document}
\selectlanguage{ngerman}
\title{2012 Hauptklausur (WS 2011/12)}

\setcounter{section}{1}
%%%%%%%%%%%%%%%%%%%%%%%%%%%%%%%%%%%%%%%%%%%%%%%%%%%%%%%%%%%%%%%%%%%%%%%%%%%%%%
\section*{Aufgabe 1: Wahrnehmung und Farbräume}
\subsection*{Teilaufgabe 1a}
% \begin{figure}[h]
%     \centering
%     \includegraphics*[width=0.8\linewidth, keepaspectratio]{1a.png}
%     \caption{Whatever}
%     \label{fig:1a}
% \end{figure}
\textit{Eine Grafikkarte ist an ein Anzeigegerät mit einem Gamma-Wert von 2.0 angeschlossen
und muss eine entsprechende Gamma-Korrektur durchführen.
Berechnen Sie den Intensitätswert , den die Grafikkarte an das Anzeigegerät senden muss, um eine Ausgabe
mit der Hälfte der Maximalintensität zu erreichen. (Der Wertebereich der Koeffzienten
reicht von 0 bis zur Maximalintensität 1.0 .)}

Es gilt

\[I_{\text{out}} = I_{\text{in}}^\gamma\]

für $I_{\text{out}} = \nicefrac{1}{2}$ und $\gamma = 2$ muss also gelten:

\[I_{\text{in}} = \frac{1}{\sqrt{2}}\]


\subsection*{Teilaufgabe 1b}
\textit{Sie haben ein Bild im RGB-Farbraum gegeben und wollen den Helligkeitskontrast erhöhen. In welchen der in der Vorlesung vorgestellten Farbräume wandeln Sie es um,
um diese Kontrasterhöhung möglichst einfach durchführen zu können? Welche Berechnung(en) führen Sie dazu auf den Koeffizienten dieses Farbraums aus?}

HSV (oder HSI, HSL). Dann wird einfach der V-Wert (I-Wert, L-Wert) erhöht.

\subsection*{Teilaufgabe 1c}
\begin{tabular}{cp{8cm}llp{4cm}}\toprule
\# & Aussage                                                                                                     & Wahr           & Falsch           & Begründung            \\\midrule
 1 & Um den Farbeindruck für einen Menschen eindeutig zu beschreiben, genügt ein Farbmodell mit 3 Koeffizienten. & \CheckedBox    & $\square$        & Graßmansche Gesetze   \\
 2 & Durch diese 3 Koeffizienten ist dann das Spektrum ebenso eindeutig festgelegt.                              & $\square$      & \CheckedBox      & Metamerie             \\
 3 & Der RGB-Einheitswürfel enthält alle sichtbaren Farben.                                                      & $\square$      & \CheckedBox      & Magenta / Purple-Line \\
 4 & Der RGB-Einheitswürfel enthält Farben, die sich im CIE XYZ-Farbmodell nicht darstellen lassen.              & $\square$      & \CheckedBox      & Es ist umgekehrt  \\\bottomrule
\end{tabular}

\section*{Aufgabe 2}
TODO

\section*{Aufgabe 3}
TODO

\section*{Aufgabe 4}
TODO

\section*{Aufgabe 5}
TODO

\section*{Aufgabe 6}
TODO

\section*{Aufgabe 7}
TODO

\section*{Aufgabe 8}
\subsection*{Teilaufgabe 8a}
TODO

\subsection*{Teilaufgabe 8b}
\inputminted[linenos, numbersep=5pt, tabsize=4, frame=lines, label=shader.frag]{glsl}{shader.frag}

\section*{Aufgabe 9}
\subsection*{Teilaufgabe 9a}
TODO
\subsection*{Teilaufgabe 9b}
TODO
\subsection*{Teilaufgabe 9c}
TODO


\end{document}
