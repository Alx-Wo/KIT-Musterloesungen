\documentclass[a4paper]{scrartcl}
\usepackage[ngerman]{babel}
\usepackage[utf8]{inputenc}
\usepackage{amssymb,amsmath}
\usepackage{graphicx}
\usepackage[inline]{enumitem}
\setlist{noitemsep}
\usepackage[binary-units=true]{siunitx}
\usepackage{hyperref}
\usepackage{parskip}
\usepackage[nameinlink,noabbrev,ngerman]{cleveref} % has to be after hyperref
\usepackage[colorinlistoftodos]{todonotes}

\usepackage{listings}% http://ctan.org/pkg/listings
\lstset{
  basicstyle=\ttfamily,
  mathescape
}

\setcounter{secnumdepth}{2}
\setcounter{tocdepth}{2}

\usepackage{microtype}

\begin{document}
\selectlanguage{ngerman}
\title{2015 Nachklausur (WS 2014/15)}
\author{Martin Thoma}

\setcounter{section}{1}
%%%%%%%%%%%%%%%%%%%%%%%%%%%%%%%%%%%%%%%%%%%%%%%%%%%%%%%%%%%%%%%%%%%%%%%%%%%%%%
\section*{Aufgabe 1}
\subsection*{Teilaufgabe 1a}
TODO
\subsection*{Teilaufgabe 1b}
TODO

\section*{Aufgabe 2}
\subsection*{Teilaufgabe 2a}

\begin{itemize}
    \item RGB: LCD/CRT-Displays
    \item CMYK: Drucker
    \item HSV: TODO
    \item HSI: TODO
    \item XYZ Color Space: Farbraum für Konversion zwischen Farbräumen
    \item Lab-Farbraum: TODO
\end{itemize}

\subsection*{Teilaufgabe 2b}
TODO

\subsection*{Teilaufgabe 2c}
TODO

\section*{Aufgabe 3: Transformationen}
\subsection*{Teilaufgabe 3a}
Transformationen mit homogenen Koordianten laufen Grundsätzlich nach folgendem Schema ab:
\[\begin{pmatrix}\tilde{x}\\ \tilde{y} \\ 1\end{pmatrix} \gets T \cdot \begin{pmatrix}x\\ y \\ 1\end{pmatrix}\]

Die Transformationsmatrix $T$ für die Translation von homogenen Koordinaten ist von der Form

\[T = \begin{pmatrix}1 & 0 & \Delta x\\0 & 1 & \Delta y\\0 & 0 & 1\end{pmatrix}\]

Die Transformationsmatrix $R$ für eine Rotation um den Punkt $c = (c_x, c_y)$ um den Winkel $\alpha$ ist

\[R_{\alpha, c} = \begin{pmatrix}          1 &            0 & c_x\\          0 &           1 & c_y\\0 & 0 & 1\end{pmatrix} \cdot
  \begin{pmatrix}\cos \alpha & -\sin \alpha &   0\\\sin \alpha & \cos \alpha &   0\\0 & 0 & 1\end{pmatrix} \cdot
  \begin{pmatrix}          1 &            0 &-c_x\\          0 &           1 &-c_y\\0 & 0 & 1\end{pmatrix}\]

Die Idee ist nun, zuerst eine Rotation um $90^\circ$ gegen den Urzeigersinn
um $(0, 0)$ zu machen (Matrix $R$). Dann wird das Rechteck in Richtung der $x$-Achse um
die hälfte gestaucht (Matrix $S$) und schließlich um $0.5$ nach links verschoben (Matrix $T$):

\begin{align}
    R &= \begin{pmatrix}\cos 90 & -\sin 90 &   0\\\sin 90 & \cos 90 &   0\\0 & 0 & 1\end{pmatrix}
       = \begin{pmatrix}0 & -1 &   0\\1 & 0 &   0\\0 & 0 & 1\end{pmatrix}\\
    S &= \begin{pmatrix}0.5 & 0 & 0\\0 & 1 & 0\\0 & 0 & 1\end{pmatrix}\\
    T &= \begin{pmatrix}1 & 0 & -0.5\\0 & 1 & 0\\0 & 0 & 1\end{pmatrix}\\
    M &= T \cdot S \cdot R\\
      &= \begin{pmatrix}0 & -0.5 & -0.5\\1 & 0 & 0\\0 & 0 & 1\end{pmatrix}
\end{align}

Zur Kontrolle:

\begin{align}
    M \cdot \begin{pmatrix}0\\0\\1\end{pmatrix} &= \begin{pmatrix}-0.5\\0\\1\end{pmatrix}
    & M \cdot \begin{pmatrix}1\\0\\1\end{pmatrix} &= \begin{pmatrix}-0.5\\1\\1\end{pmatrix}\\
    M \cdot \begin{pmatrix}1\\1\\1\end{pmatrix} &= \begin{pmatrix}-1\\1\\1\end{pmatrix}
    & M \cdot \begin{pmatrix}0\\1\\1\end{pmatrix} &= \begin{pmatrix}-1\\0\\1\end{pmatrix}
\end{align}




\subsection*{Teilaufgabe 3b}
TODO
\subsection*{Teilaufgabe 3c}
TODO

\section*{Aufgabe 4}
\subsection*{Teilaufgabe 4a}
TODO
\subsection*{Teilaufgabe 4b}
TODO
\subsection*{Teilaufgabe 4c}
\begin{enumerate}
    \item[(i)] Wie verändert sich das Glanzlicht, wenn $e$ größer wird?
    \item[$\Rightarrow$] TODO
    \item[(ii)] Wie verändert sich das Glanzlicht, wenn die Kugel um eine beliebige Achse rotiert?
    \item[$\Rightarrow$] TODO
\end{enumerate}

\section*{Aufgabe 5}
\subsection*{Teilaufgabe 5a}
TODO
\subsection*{Teilaufgabe 5b}
TODO
\subsection*{Teilaufgabe 5c}
TODO

\subsection*{Teilaufgabe 5d}
Gouraud-Shading im Vergleich zu Phong-Shading
\begin{itemize}
    \item Vorteil: Schnellere Berechnung
    \item Nachteil: Schlechtere Ergebnisse
\end{itemize}

\section*{Aufgabe 6}
\subsection*{Teilaufgabe 6a}
TODO
\subsection*{Teilaufgabe 6b}
TODO
\subsection*{Teilaufgabe 6c}
TODO
\subsection*{Teilaufgabe 6d}
TODO
\subsection*{Teilaufgabe 6e}
TODO
\subsection*{Teilaufgabe 6f}
TODO

\section*{Aufgabe 7}
TODO

\section*{Aufgabe 8}
\subsection*{Teilaufgabe 8a}
TODO
\subsection*{Teilaufgabe 8b}
\begin{itemize}
    \item[1] Der Baum einer Hüllkörperhierarchie ist immer balanciert.
    \item[$\Rightarrow$] TODO
    \item[2] Der Speicherbedarf für ein reguläres Gitter ist unabhängig von der Anzahl der Primitive.
    \item[$\Rightarrow$] TODO
    \item[3] Ein kD-Baum hat immer achsenparallele Split-Ebenen.
    \item[$\Rightarrow$] TODO
    \item[4] Ein kD-Baum braucht spezielle Vorkehrungen, um redundante Schnitttests mit demselben Dreieck auszuschliessen.
    \item[$\Rightarrow$] TODO
    \item[5] Ein Verfahren zur Erzeugung eines kD-Baumes erzeugt auch gültige BSP-Bäume.
    \item[$\Rightarrow$] TODO
    \item[6] Reguläre uniforme Gitter leiden nicht unter dem Teapot-in-a-Stadium Problem.
    \item[$\Rightarrow$] TODO
    \item[7] Die Komplexität der Bestimmung eines Schnittpunktes in einem BSP-Baum mit n Primitiven liegt im Optimalfall in $\mathcal{O}(\log n)$.
    \item[$\Rightarrow$] TODO
    \item[8] Das Traversieren einer Hüllkörperhierarchie kann abgebrochen werden sobald ein Schnittpunkt gefunden wurde.
    \item[$\Rightarrow$] Falsch. Es könnte einen Schnitt geben, der näher zur Kamera ist (TODO: Beispiel in Folien raussuchen.)
\end{itemize}

\section*{Aufgabe 9}
TODO

\section*{Aufgabe 10}
TODO

\section*{Aufgabe 11}
\subsection*{Teilaufgabe 11a}
TODO
\subsection*{Teilaufgabe 11b}
TODO


\end{document}
