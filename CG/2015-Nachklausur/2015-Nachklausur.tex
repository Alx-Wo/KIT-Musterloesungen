\documentclass[a4paper]{scrartcl}
\usepackage[ngerman]{babel}
\usepackage[utf8]{inputenc}
\usepackage{amssymb,amsmath}
\usepackage{graphicx}
\usepackage[inline]{enumitem}
\setlist{noitemsep}
\usepackage[binary-units=true]{siunitx}
\usepackage{hyperref}
\usepackage{parskip}
\usepackage[nameinlink,noabbrev,ngerman]{cleveref} % has to be after hyperref
\usepackage[colorinlistoftodos]{todonotes}

\usepackage{listings}% http://ctan.org/pkg/listings
\lstset{
  basicstyle=\ttfamily,
  mathescape
}

\setcounter{secnumdepth}{2}
\setcounter{tocdepth}{2}

\usepackage{microtype}

\begin{document}
\selectlanguage{ngerman}
\title{2015 Nachklausur (WS 2014/15)}
\author{Martin Thoma}

\setcounter{section}{1}
%%%%%%%%%%%%%%%%%%%%%%%%%%%%%%%%%%%%%%%%%%%%%%%%%%%%%%%%%%%%%%%%%%%%%%%%%%%%%%
\section*{Aufgabe 1: Raytracing}
\subsection*{Teilaufgabe 1a}
Phong-Beleuchtungsmodell anwenden:

\begin{itemize}
    \item Berechnung des ambienten Anteils (indirekte Beleuchtung, Licht von anderen Oberflächen)
    \item Berechnung der Reflektion
    \begin{itemize}
        \item Berechnung des spekularen Reflektion. Diese findet nur in Richtung
              $R_L = 2 N \cdot (N \cdot L)$ statt. In alle andren Richtungen
              fällt sie stark ab:
              \[I_s = k_s \cdot I_L \cdot \cos^n \alpha = k_s \cdot I_L (R_L \cdot V)^n\]
              wobei $n$ der Phong-Exponent ist.
        \item Berechnung der diffusen (Lambertschen) Reflektion:
              \[I_d = k_d \cdot I_L \cdot \cos \theta = k_d \cdot I_L \cdot (N \cdot L) \]
    \end{itemize}
\end{itemize}

Also:

\[I = \underbrace{k_a \cdot I_L}_{\text{ambient}} + \underbrace{k_d \cdot I_L \cdot (N \cdot L)}_{\text{diffus}} + \underbrace{k_s \cdot I_L (R_L \cdot V)^n}_{\text{spekular}}\]

Auch:

\begin{itemize}
    \item Berechnung von Schattenstrahlen
    \item Weitere, Strahlen rekursive verschießen
\end{itemize}


\subsection*{Teilaufgabe 1b}
TODO

\section*{Aufgabe 2: Farben und Spektren}
\subsection*{Teilaufgabe 2a}

\begin{itemize}
    \item RGB: LCD/CRT-Displays
    \item CMYK: Drucker
    \item HSV: TODO
    \item HSI: TODO
    \item XYZ Color Space: Farbraum für Konversion zwischen Farbräumen
    \item Lab-Farbraum: TODO
\end{itemize}

\subsection*{Teilaufgabe 2b}
TODO

\subsection*{Teilaufgabe 2c}
Metamerismus ist das Phänomen, dass unterschiedliche Spektren gleich aussehen
können. Dies ist wichtig für Monitore, da sie aufgrund dieses Phänomens mit nur
3~Farben den gleichen Farbeindruck erwecken können wie mit einem komplexeren
Spektrum.

\section*{Aufgabe 3: Transformationen}
\subsection*{Teilaufgabe 3a}
Transformationen mit homogenen Koordianten laufen Grundsätzlich nach folgendem Schema ab:
\[\begin{pmatrix}\tilde{x}\\ \tilde{y} \\ 1\end{pmatrix} \gets T \cdot \begin{pmatrix}x\\ y \\ 1\end{pmatrix}\]

Die Transformationsmatrix $T$ für die Translation von homogenen Koordinaten ist von der Form

\[T = \begin{pmatrix}1 & 0 & \Delta x\\0 & 1 & \Delta y\\0 & 0 & 1\end{pmatrix}\]

Die Transformationsmatrix $R$ für eine Rotation um den Punkt $c = (c_x, c_y)$ um den Winkel $\alpha$ ist

\[R_{\alpha, c} = \begin{pmatrix}          1 &            0 & c_x\\          0 &           1 & c_y\\0 & 0 & 1\end{pmatrix} \cdot
  \begin{pmatrix}\cos \alpha & -\sin \alpha &   0\\\sin \alpha & \cos \alpha &   0\\0 & 0 & 1\end{pmatrix} \cdot
  \begin{pmatrix}          1 &            0 &-c_x\\          0 &           1 &-c_y\\0 & 0 & 1\end{pmatrix}\]

Die Idee ist nun, zuerst eine Rotation um $90^\circ$ gegen den Urzeigersinn
um $(0, 0)$ zu machen (Matrix $R$). Dann wird das Rechteck in Richtung der $x$-Achse um
die hälfte gestaucht (Matrix $S$) und schließlich um $0.5$ nach links verschoben (Matrix $T$):

\begin{align}
    R &= \begin{pmatrix}\cos 90 & -\sin 90 &   0\\\sin 90 & \cos 90 &   0\\0 & 0 & 1\end{pmatrix}
       = \begin{pmatrix}0 & -1 &   0\\1 & 0 &   0\\0 & 0 & 1\end{pmatrix}\\
    S &= \begin{pmatrix}0.5 & 0 & 0\\0 & 1 & 0\\0 & 0 & 1\end{pmatrix}\\
    T &= \begin{pmatrix}1 & 0 & -0.5\\0 & 1 & 0\\0 & 0 & 1\end{pmatrix}\\
    M &= T \cdot S \cdot R\\
      &= \begin{pmatrix}0 & -0.5 & -0.5\\1 & 0 & 0\\0 & 0 & 1\end{pmatrix}
\end{align}

Zur Kontrolle:

\begin{align}
    M \cdot \begin{pmatrix}0\\0\\1\end{pmatrix} &= \begin{pmatrix}-0.5\\0\\1\end{pmatrix}
    & M \cdot \begin{pmatrix}1\\0\\1\end{pmatrix} &= \begin{pmatrix}-0.5\\1\\1\end{pmatrix}\\
    M \cdot \begin{pmatrix}1\\1\\1\end{pmatrix} &= \begin{pmatrix}-1\\1\\1\end{pmatrix}
    & M \cdot \begin{pmatrix}0\\1\\1\end{pmatrix} &= \begin{pmatrix}-1\\0\\1\end{pmatrix}
\end{align}




\subsection*{Teilaufgabe 3b}
TODO

\subsection*{Teilaufgabe 3c}
Die Transformation von Welt- in Kamerakoordinaten wird auch
\textit{Kameratransformation} genannt. Die virtuelle Kamera ist durch die
Position $\mathbf{e}$ und die negative Blickrichtung $\mathbf{w}$ definiert.
Mithilfe des Up-Vektors $\mathbf{up}$ ergibt sich

\[\mathbf{u} = \mathbf{up} \times \mathbf{w} \;\;\;\;\; \mathbf{v} = \mathbf{w} \times \mathbf{u}\]

Es wird zuerst eine Translation um $-\mathbf{e}$ durchgeführt und dann
eine Transformation in das Kamera-Koordiantensystem durchgeführt. Die Basis
des Kamera-Koordinatensystems ist $\mathbf{u}, \mathbf{v}, \mathbf{w}$.

Das Verschieben ist einfach die Matrix

\[T = \begin{pmatrix}
1 & 0 & 0 & e_x\\
0 & 1 & 0 & e_y\\
0 & 0 & 1 & e_z\\
0 & 0 & 0 & 1\end{pmatrix}\]

nun muss noch rotiert werden:

\[R = \begin{pmatrix}
u_x & u_y & u_z & 0\\
v_x & v_y & v_z & 0\\
w_x & w_y & w_z & 0\\
0 & 0 & 0 & 1\end{pmatrix}\]

Die Gesamttrasformationsmatrix ist also $V = R \cdot T$.

(vgl. \texttt{03\_ Transformationen und homogene Koordinaten.pdf}, Folie 50)

\section*{Aufgabe 4: Phong-Beleuchtungsmodell}
\subsection*{Teilaufgabe 4a}

\begin{figure}[h]
    \centering
    \includegraphics*[width=0.8\linewidth, keepaspectratio]{reflection-angles.png}
    \caption{Situationsskizze. $r_l$ ist senkrecht auf $L$ in $P$.}
    \label{fig:your-reference}
\end{figure}

\[\mathbf{r}_l = 2 \cdot \mathbf{n} \cdot (\mathbf{n} \cdot \mathbf{l}) - \mathbf{l}\]

Kaptiel 2, Folie 96.


\subsection*{Teilaufgabe 4b}
Die spekulare Komponente im Phong Beleuchtungsmodell lautet
\[I_s = k_s \cdot I_L \cdot \cos^n \alpha = k_s \cdot I_L (\mathbf{r}_l \cdot \mathbf{v})^n\]

\subsection*{Teilaufgabe 4c}
\begin{enumerate}
    \item[(i)] Wie verändert sich das Glanzlicht, wenn $e$ größer wird?
    \item[$\Rightarrow$] TODO
    \item[(ii)] Wie verändert sich das Glanzlicht, wenn die Kugel um eine beliebige Achse rotiert?
    \item[$\Rightarrow$] TODO
\end{enumerate}


\section*{Aufgabe 5: Dreiecke und Schattierung}
\subsection*{Teilaufgabe 5a}
Drei Punkte $P_1, P_2, P_3$ definieren eine Ebene. Diese Ebene ist eine Menge
von Punkten

\[\{x \in \mathbb{R}^3 | x \text{ liegt auf der Ebene von } P_1, P_2, P_3\} = \{x \in \mathbb{R}^3 = n \cdot x = d\}\]

Es gilt also folgendes Gleichungssystem zu lösen:

\begin{align}
    \|n\| &= 1\\
    n_x \cdot P_{1, x} + n_y \cdot P_{1, y} + n_z \cdot P_{1, z} &= d\\
    n_x \cdot P_{2, x} + n_y \cdot P_{2, y} + n_z \cdot P_{2, z} &= d\\
    n_x \cdot P_{3, x} + n_y \cdot P_{3, y} + n_z \cdot P_{3, z} &= d
\end{align}

\subsection*{Teilaufgabe 5b}
Die Normalen der Vertices eines Dreiecksnetzes können wie folgt berechnet werden:

\begin{enumerate}
    \item Berechne Normale jedes Dreiecks
    \item Für jeden Vertex wird nun die Summe der Normalen der angrenzenden Dreiecke gebildet.
    \item Die Vertex-Normalen werden normalisiert, indem sie durch ihre Länge geteilt werden.
\end{enumerate}

\subsection*{Teilaufgabe 5c}
TODO

\subsection*{Teilaufgabe 5d}
Gouraud-Shading im Vergleich zu Phong-Shading
\begin{itemize}
    \item Vorteil: Schnellere Berechnung
    \item Nachteil: Schlechtere Ergebnisse
\end{itemize}


\section*{Aufgabe 6: Texturen und Texturfilterung}
\subsection*{Teilaufgabe 6a}
vgl. geometrische Interpretation (Kaptiel 2, Folie 43)

\begin{align}
    \lambda_1 &= \frac{A_\Delta(S, P_2, P_3)}{A_\Delta(P_1, P_2, P_3)}\\
    \lambda_2 &= \frac{A_\Delta(P_1, S, P_3)}{A_\Delta(P_1, P_2, P_3)}\\
    \lambda_3 &= \frac{A_\Delta(P1, P_2, S)}{A_\Delta(P_1, P_2, P_3)}\\
\end{align}


\subsection*{Teilaufgabe 6b}
\[T_S = \sum_{i=1}^3 \lambda_i T_i\]

\subsection*{Teilaufgabe 6c}
\begin{itemize}
    \item Aliasing-Artefakte durch Unterabtastung (Kapitel 4, Folie 45): Beispiel TODO
    \item TODO: Beispiel TODO
\end{itemize}

\subsection*{Teilaufgabe 6d}
TODO

\subsection*{Teilaufgabe 6e}
Unter trilinearer Texturfilterung versteht man eine bilineare Interpolation
der Stufe $n$, eine bilineare Interpolation der Stufe $n+1$ und dann eine
lineare Interpolation dieser beiden Farben.

\subsection*{Teilaufgabe 6f}
TODO


\section*{Aufgabe 7: Projektionen}
TODO

\section*{Aufgabe 8: Beschleunigungsstrukturen}
\subsection*{Teilaufgabe 8a}
TODO
\subsection*{Teilaufgabe 8b}
\begin{itemize}
    \item[1] Der Baum einer Hüllkörperhierarchie ist immer balanciert.
    \item[$\Rightarrow$] TODO
    \item[2] Der Speicherbedarf für ein reguläres Gitter ist unabhängig von der Anzahl der Primitive.
    \item[$\Rightarrow$] TODO
    \item[3] Ein kD-Baum hat immer achsenparallele Split-Ebenen.
    \item[$\Rightarrow$] TODO
    \item[4] Ein kD-Baum braucht spezielle Vorkehrungen, um redundante Schnitttests mit demselben Dreieck auszuschliessen.
    \item[$\Rightarrow$] TODO
    \item[5] Ein Verfahren zur Erzeugung eines kD-Baumes erzeugt auch gültige BSP-Bäume.
    \item[$\Rightarrow$] TODO
    \item[6] Reguläre uniforme Gitter leiden nicht unter dem Teapot-in-a-Stadium Problem.
    \item[$\Rightarrow$] TODO
    \item[7] Die Komplexität der Bestimmung eines Schnittpunktes in einem BSP-Baum mit n Primitiven liegt im Optimalfall in $\mathcal{O}(\log n)$.
    \item[$\Rightarrow$] TODO
    \item[8] Das Traversieren einer Hüllkörperhierarchie kann abgebrochen werden sobald ein Schnittpunkt gefunden wurde.
    \item[$\Rightarrow$] Falsch. Es könnte einen Schnitt geben, der näher zur Kamera ist (TODO: Beispiel in Folien raussuchen.)
\end{itemize}

\section*{Aufgabe 9: Instancing (GLSL)}
TODO

\section*{Aufgabe 10: Normal Mapping in Objektkoordinaten (GLSL)}
TODO

\section*{Aufgabe 11: Bézier-Kurven und Bézier-Splines}
\subsection*{Teilaufgabe 11a}
TODO
\subsection*{Teilaufgabe 11b}
TODO


\end{document}
