\documentclass[a4paper]{scrartcl}
\usepackage[ngerman]{babel}
\usepackage[utf8]{inputenc}
\usepackage{amssymb,amsmath}
\usepackage{graphicx}
\usepackage[inline]{enumitem}
\setlist{noitemsep}
\usepackage[binary-units=true]{siunitx}
\usepackage{hyperref}
\usepackage{parskip}
\usepackage[nameinlink,noabbrev,ngerman]{cleveref} % has to be after hyperref
\usepackage[colorinlistoftodos]{todonotes}
\usepackage{nicefrac}
\usepackage{csquotes}

\usepackage{minted} % needed for the inclusion of source code

\setcounter{secnumdepth}{2}
\setcounter{tocdepth}{2}

\usepackage{microtype}

\begin{document}
\selectlanguage{ngerman}
\title{2015 Nachklausur (WS 2014/15)}
\author{Martin Thoma}

\setcounter{section}{1}
%%%%%%%%%%%%%%%%%%%%%%%%%%%%%%%%%%%%%%%%%%%%%%%%%%%%%%%%%%%%%%%%%%%%%%%%%%%%%%
\section*{Aufgabe 1: Raytracing}
\subsection*{Teilaufgabe 1a}
Phong-Beleuchtungsmodell anwenden:

\begin{itemize}
    \item Berechnung des ambienten Anteils (indirekte Beleuchtung, Licht von anderen Oberflächen)
    \item Berechnung der Reflektion
    \begin{itemize}
        \item Berechnung des spekularen Reflektion. Diese findet nur in Richtung
              $R_L = 2 N \cdot (N \cdot L)$ statt. In alle andren Richtungen
              fällt sie stark ab:
              \[I_s = k_s \cdot I_L \cdot \cos^n \alpha = k_s \cdot I_L (R_L \cdot V)^n\]
              wobei $n$ der Phong-Exponent ist.
        \item Berechnung der diffusen (Lambertschen) Reflektion:
              \[I_d = k_d \cdot I_L \cdot \cos \theta = k_d \cdot I_L \cdot (N \cdot L) \]
    \end{itemize}
\end{itemize}

Also:

\[I = \underbrace{k_a \cdot I_L}_{\text{ambient}} + \underbrace{k_d \cdot I_L \cdot (N \cdot L)}_{\text{diffus}} + \underbrace{k_s \cdot I_L (R_L \cdot V)^n}_{\text{spekular}}\]

Auch:

\begin{itemize}
    \item Berechnung von Schattenstrahlen
    \item Weitere, Strahlen rekursive verschießen
\end{itemize}


\subsection*{Teilaufgabe 1b}
TODO

\section*{Aufgabe 2: Farben und Spektren}
\subsection*{Teilaufgabe 2a}

\begin{itemize}
    \item RGB: LCD/CRT-Displays
    \item CMYK: Drucker
    \item HSV: TODO
    \item HSI: TODO
    \item XYZ Color Space: Farbraum für Konversion zwischen Farbräumen
    \item Lab-Farbraum: TODO
\end{itemize}

\subsection*{Teilaufgabe 2b}
TODO

\subsection*{Teilaufgabe 2c}
Metamerismus ist das Phänomen, dass unterschiedliche Spektren gleich aussehen
können. Dies ist wichtig für Monitore, da sie aufgrund dieses Phänomens mit nur
3~Farben den gleichen Farbeindruck erwecken können wie mit einem komplexeren
Spektrum.

\section*{Aufgabe 3: Transformationen}
\subsection*{Teilaufgabe 3a}
Transformationen mit homogenen Koordianten laufen Grundsätzlich nach folgendem Schema ab:
\[\begin{pmatrix}\tilde{x}\\ \tilde{y} \\ 1\end{pmatrix} \gets T \cdot \begin{pmatrix}x\\ y \\ 1\end{pmatrix}\]

Die Transformationsmatrix $T$ für die Translation von homogenen Koordinaten ist von der Form

\[T = \begin{pmatrix}1 & 0 & \Delta x\\0 & 1 & \Delta y\\0 & 0 & 1\end{pmatrix}\]

Die Transformationsmatrix $R$ für eine Rotation um den Punkt $c = (c_x, c_y)$ um den Winkel $\alpha$ ist

\[R_{\alpha, c} = \begin{pmatrix}          1 &            0 & c_x\\          0 &           1 & c_y\\0 & 0 & 1\end{pmatrix} \cdot
  \begin{pmatrix}\cos \alpha & -\sin \alpha &   0\\\sin \alpha & \cos \alpha &   0\\0 & 0 & 1\end{pmatrix} \cdot
  \begin{pmatrix}          1 &            0 &-c_x\\          0 &           1 &-c_y\\0 & 0 & 1\end{pmatrix}\]

Die Idee ist nun, zuerst eine Rotation um $90^\circ$ gegen den Urzeigersinn
um $(0, 0)$ zu machen (Matrix $R$). Dann wird das Rechteck in Richtung der $x$-Achse um
die hälfte gestaucht (Matrix $S$) und schließlich um $0.5$ nach links verschoben (Matrix $T$):

\begin{align}
    R &= \begin{pmatrix}\cos 90 & -\sin 90 &   0\\\sin 90 & \cos 90 &   0\\0 & 0 & 1\end{pmatrix}
       = \begin{pmatrix}0 & -1 &   0\\1 & 0 &   0\\0 & 0 & 1\end{pmatrix}\\
    S &= \begin{pmatrix}0.5 & 0 & 0\\0 & 1 & 0\\0 & 0 & 1\end{pmatrix}\\
    T &= \begin{pmatrix}1 & 0 & -0.5\\0 & 1 & 0\\0 & 0 & 1\end{pmatrix}\\
    M &= T \cdot S \cdot R\\
      &= \begin{pmatrix}0 & -0.5 & -0.5\\1 & 0 & 0\\0 & 0 & 1\end{pmatrix}
\end{align}

Zur Kontrolle:

\begin{align}
    M \cdot \begin{pmatrix}0\\0\\1\end{pmatrix} &= \begin{pmatrix}-0.5\\0\\1\end{pmatrix}
    & M \cdot \begin{pmatrix}1\\0\\1\end{pmatrix} &= \begin{pmatrix}-0.5\\1\\1\end{pmatrix}\\
    M \cdot \begin{pmatrix}1\\1\\1\end{pmatrix} &= \begin{pmatrix}-1\\1\\1\end{pmatrix}
    & M \cdot \begin{pmatrix}0\\1\\1\end{pmatrix} &= \begin{pmatrix}-1\\0\\1\end{pmatrix}
\end{align}




\subsection*{Teilaufgabe 3b}
TODO

\subsection*{Teilaufgabe 3c}
Die Transformation von Welt- in Kamerakoordinaten wird auch
\textit{Kameratransformation} genannt. Die virtuelle Kamera ist durch die
Position $\mathbf{e}$ und die negative Blickrichtung $\mathbf{w}$ definiert.
Mithilfe des Up-Vektors $\mathbf{up}$ ergibt sich

\[\mathbf{u} = \mathbf{up} \times \mathbf{w} \;\;\;\;\; \mathbf{v} = \mathbf{w} \times \mathbf{u}\]

Es wird zuerst eine Translation um $-\mathbf{e}$ durchgeführt und dann
eine Transformation in das Kamera-Koordiantensystem durchgeführt. Die Basis
des Kamera-Koordinatensystems ist $\mathbf{u}, \mathbf{v}, \mathbf{w}$.

Das Verschieben ist einfach die Matrix

\[T = \begin{pmatrix}
1 & 0 & 0 & e_x\\
0 & 1 & 0 & e_y\\
0 & 0 & 1 & e_z\\
0 & 0 & 0 & 1\end{pmatrix}\]

nun muss noch rotiert werden:

\[R = \begin{pmatrix}
u_x & u_y & u_z & 0\\
v_x & v_y & v_z & 0\\
w_x & w_y & w_z & 0\\
0 & 0 & 0 & 1\end{pmatrix}\]

Die Gesamttrasformationsmatrix ist also $V = R \cdot T$.

(vgl. \texttt{03\_ Transformationen und homogene Koordinaten.pdf}, Folie 50)

\section*{Aufgabe 4: Phong-Beleuchtungsmodell}
\subsection*{Teilaufgabe 4a}

\begin{figure}[h]
    \centering
    \includegraphics*[width=0.8\linewidth, keepaspectratio]{reflection-angles.png}
    \caption{Situationsskizze. $r_l$ ist senkrecht auf $L$ in $P$.}
    \label{fig:situation-4a}
\end{figure}

\[\mathbf{r}_l = 2 \cdot \mathbf{n} \cdot (\mathbf{n} \cdot \mathbf{l}) - \mathbf{l}\]

Kaptiel 2, Folie 96.


\subsection*{Teilaufgabe 4b}
Die spekulare Komponente im Phong Beleuchtungsmodell lautet
\[I_s = k_s \cdot I_L \cdot \cos^n \alpha = k_s \cdot I_L (\mathbf{r}_l \cdot \mathbf{v})^n\]

\subsection*{Teilaufgabe 4c}
\begin{enumerate}
    \item[(i)] Wie verändert sich das Glanzlicht, wenn $e$ größer wird?
    \item[$\Rightarrow$] TODO
    \item[(ii)] Wie verändert sich das Glanzlicht, wenn die Kugel um eine beliebige Achse rotiert?
    \item[$\Rightarrow$] TODO
\end{enumerate}


\section*{Aufgabe 5: Dreiecke und Schattierung}
\subsection*{Teilaufgabe 5a}
Drei Punkte $P_1, P_2, P_3$ definieren eine Ebene. Diese Ebene ist eine Menge
von Punkten

\[\{x \in \mathbb{R}^3 | x \text{ liegt auf der Ebene von } P_1, P_2, P_3\} = \{x \in \mathbb{R}^3 = n \cdot x = d\}\]

Es gilt also folgendes Gleichungssystem zu lösen:

\begin{align}
    \|n\| &= 1\\
    n_x \cdot P_{1, x} + n_y \cdot P_{1, y} + n_z \cdot P_{1, z} &= d\\
    n_x \cdot P_{2, x} + n_y \cdot P_{2, y} + n_z \cdot P_{2, z} &= d\\
    n_x \cdot P_{3, x} + n_y \cdot P_{3, y} + n_z \cdot P_{3, z} &= d
\end{align}

\subsection*{Teilaufgabe 5b}
Die Normalen der Vertices eines Dreiecksnetzes können wie folgt berechnet werden:

\begin{enumerate}
    \item Berechne Normale jedes Dreiecks
    \item Für jeden Vertex wird nun die Summe der Normalen der angrenzenden Dreiecke gebildet.
    \item Die Vertex-Normalen werden normalisiert, indem sie durch ihre Länge geteilt werden.
\end{enumerate}

\subsection*{Teilaufgabe 5c}
TODO

\subsection*{Teilaufgabe 5d}
Gouraud-Shading im Vergleich zu Phong-Shading
\begin{itemize}
    \item Vorteil: Schnellere Berechnung
    \item Nachteil: Schlechtere Ergebnisse
\end{itemize}


\section*{Aufgabe 6: Texturen und Texturfilterung}
\subsection*{Teilaufgabe 6a}
vgl. geometrische Interpretation (Kaptiel 2, Folie 43)

\begin{align}
    \lambda_1 &= \frac{A_\Delta(S, P_2, P_3)}{A_\Delta(P_1, P_2, P_3)}\\
    \lambda_2 &= \frac{A_\Delta(P_1, S, P_3)}{A_\Delta(P_1, P_2, P_3)}\\
    \lambda_3 &= \frac{A_\Delta(P1, P_2, S)}{A_\Delta(P_1, P_2, P_3)}
\end{align}


\subsection*{Teilaufgabe 6b}
\[T_S = \sum_{i=1}^3 \lambda_i T_i\]

\subsection*{Teilaufgabe 6c}
\begin{itemize}
    \item Aliasing-Artefakte durch Unterabtastung (Kapitel 4, Folie 45): Beispiel TODO
    \item TODO: Beispiel TODO
\end{itemize}

\subsection*{Teilaufgabe 6d}
TODO

\subsection*{Teilaufgabe 6e}
Unter trilinearer Texturfilterung versteht man eine bilineare Interpolation
der Stufe $n$, eine bilineare Interpolation der Stufe $n+1$ und dann eine
lineare Interpolation dieser beiden Farben.

\subsection*{Teilaufgabe 6f}
\begin{itemize}
    \item \texttt{GL\_TEXTURE\_MAG\_FILTER}: The texture magnification function
    is used when the pixel being textured maps to an area less than or equal to
    one texture element. It sets the texture magnification function to either
    \texttt{GL\_NEAREST} or \texttt{GL\_LINEAR}. \texttt{GL\_NEAREST} is
    generally faster than
    \texttt{GL\_LINEAR}, but it can produce textured images with sharper edges
    because the transition between texture elements is not as smooth. The
    initial value of \texttt{GL\_TEXTURE\_MAG\_FILTER} is \texttt{GL\_LINEAR}.\\
    Quelle: \href{https://www.talisman.org/opengl-1.1/Reference/glTexParameter.html}{www.talisman.org/opengl-1.1/Reference/glTexParameter.html}
    \item \texttt{GL\_TEXTURE\_MIN\_FILTER}: The texture minifying function is used whenever the pixel being textured maps to an area greater than one texture element. There are six defined minifying functions. Two of them use the nearest one or nearest four texture elements to compute the texture value. The other four use mipmaps. [...]\\
    Quelle: \href{https://www.talisman.org/opengl-1.1/Reference/glTexParameter.html}{www.talisman.org/opengl-1.1/Reference/glTexParameter.html}
\end{itemize}

Bilineare Filterung (Kaptiel 4, Folie 38):

\begin{align}
    a &= \frac{3}{8} - \frac{1}{4} = \frac{1}{8}\\
    b &= \frac{1}{2} - \frac{1}{4} = \frac{1}{4}\\
    t_{12} &= t_1 + a(t_2 - t_1) = 16 + \frac{1}{8} \cdot (-16) = 14\\
    t_{34} &= (1-a) t_3 + a t_4 = \frac{7}{8} \cdot 0 + \frac{1}{8} \cdot 32 = 4\\
    t &= (1-b) t_{12} + b t_{34} = \frac{3}{4} \cdot 14 + \frac{1}{4} \cdot 4 = 10.5 + 1 = 11.5
\end{align}

\clearpage
\section*{Aufgabe 7: Projektionen}
Wir müssen eine Matrix $M \in \mathbb{R}^{3 \times 3}$ finden, welche die
Projektion von Punkten $P \in [0, \infty)^2$ auf den Unterraum
$\{P \in [0, \infty)^2 | P_x + P_y = 1\}$ transformiert.

Die Homogenen Koordinaten $(x, y, z)$ entsprechen den 2D-Koordinaten
$(\frac{x}{z}, \frac{y}{z})$.

% Eine homogene Matrix hat in der letzten Zeile immer $(0, 0, 1)$ stehen. Damit
% bleiben noch 6~Parameter zu finden.
\goodbreak
Es gilt, dass alle Punkte auf der $x$-Achse auf (1, 0) abgebildet werden:\nobreak
\begin{align}
    m_{1,1} \cdot x_1 + m_{1,2} \cdot 0 + m_{1,3} \cdot 1 &= m_{3,1} \cdot x_1 + m_{3,2} \cdot 0 + m_{3,3} \cdot 1 \;\;\; \forall x_1 \in [0, \infty) \label{eq:x-1}\\
    m_{2,1} \cdot x_1 + m_{2,2} \cdot 0 + m_{2,3} \cdot 1 &= 0 \;\;\; \forall x_1 \in [0, \infty)
\end{align}

Daraus folgt: $m_{2,1} = 0, m_{2,3} = 0$. Außerdem lässt sich \cref{eq:x-1}
vereinfachen:

\begin{align}
    m_{1,1} \cdot x_1 +  m_{1,3} &= m_{3,1} \cdot x_1 + m_{3,3} \;\;\; \forall x_1 \in [0, \infty)\\
    (m_{1,1} - m_{3,1}) \cdot x_1 + (m_{1,3} - m_{3,3}) &= 0 \;\;\; \forall x_1 \in [0, \infty)\label{eq:x-1.1}
\end{align}

Daher gilt:

\begin{itemize}
    \item $m_{1,1} = m_{3,1}$
    \item $m_{1,3} = m_{3,3}$
\end{itemize}

Bisher wissen wir:\nobreak
\[M = \begin{pmatrix}            m_{1,1} & m_{1,2} & m_{3,3}\\
                                       0 & m_{2,2} & 0\\
                                 m_{1,1} & m_{3,2} & m_{3,3}\end{pmatrix}\]

\goodbreak
Des weiteren gilt, dass alle Punkte auf der $y$-Achse auf (0, 1) abgebildet werden:\nobreak
\begin{align}
    m_{1,1} \cdot 0 + m_{1,2} \cdot x_2 + m_{1,3} \cdot 1 &= 0 \;\;\; \forall x_2 \in [0, \infty)\\
    m_{2,1} \cdot 0 + m_{2,2} \cdot x_2 + m_{2,3} \cdot 1 &= m_{3,1} \cdot 0 + m_{3,2} \cdot x_2 + m_{3,3} \cdot 1 \;\;\; \forall x_2 \in [0, \infty)
\end{align}

Wie zuvor, folgt: $m_{1,2} = m_{1,3} = 0$. Es verbleiben 3~Parameter.
\goodbreak
Bisher wissen wir:\nobreak
\[M = \begin{pmatrix}      m_{1,1} &       0 & 0\\
                                 0 & m_{2,2} & 0\\
                           m_{1,1} & m_{3,2} & 0\end{pmatrix}\]

Es gilt, dass alle Punkte auf der Diagonalen auf $(\nicefrac{1}{2}, \nicefrac{1}{2})$
abgebildet werden:

\begin{align}
    m_{1,1} \cdot x &= \nicefrac{1}{2} \cdot (m_{1,1} x + m_{3,2} x) \;\;\; \forall x \in [0, \infty)\\
    m_{2,2} \cdot x &= \nicefrac{1}{2} \cdot (m_{1,1} x + m_{3,2} x) \;\;\; \forall x \in [0, \infty)
\end{align}

Für $x \neq 0$ kann man hier beide Gleichungen jeweils durch $x$ teilen und erhält:

\begin{align}
    m_{1,1} &= m_{3,2} \;\;\; \forall x \in (0, \infty)\\
    m_{2,2} &= \nicefrac{1}{2} \cdot (m_{1,1} + m_{3,2}) \;\;\; \forall x \in (0, \infty)\label{eq:m22}\\
    \overset{\ref{eq:m22}}{\Rightarrow} m_{2,2} &= m_{1,1}
\end{align}

Bisher wissen wir:\nobreak
\[M = \begin{pmatrix}      m_{1,1} &       0 & 0\\
                                 0 & m_{1,1} & 0\\
                           m_{1,1} & m_{1,1} & 0\end{pmatrix}\]

Nun gilt für die Transformation:

\begin{align}
    T(\begin{pmatrix}x\\y\end{pmatrix}) &= M \cdot \begin{pmatrix}x\\y\\1\end{pmatrix}\\
      &= \begin{pmatrix}m_{1,1} \cdot x\\m_{1,1} \cdot y\\m_{1,1} \cdot (x+y)\end{pmatrix}\\
      &= \begin{pmatrix}\frac{x}{x+y}\\\frac{y}{x+y}\end{pmatrix}
\end{align}

Die Konkrete Wahl von $m_{1,1}$ ist also egal, solange $m_{1,1} \neq 0$. Wähle
oBdA $m_{1,1} = 1$ und daher:

\[M = \begin{pmatrix}      1 & 0 & 0\\
                           0 & 1 & 0\\
                           1 & 1 & 0\end{pmatrix}\]

Für den Punkt (2,1) gilt also:

\[\begin{pmatrix}1 & 0 & 0\\
                 0 & 1 & 0\\
                 1 & 1 & 0\end{pmatrix} \cdot \begin{pmatrix}2\\1\\1\end{pmatrix}_H = \begin{pmatrix}2\\1\\3\end{pmatrix}_H = \begin{pmatrix}\nicefrac{2}{3}\\\nicefrac{1}{3}\end{pmatrix}\]


\section*{Aufgabe 8: Beschleunigungsstrukturen}
\subsection*{Teilaufgabe 8a}
Um Schnitttests zu beschleunigen, kann man den Raum durch ein Gitter in Zellen
unterteilen. Für jede Zelle nimmt man die AABB des Objekts um zu bestimmen,
ob ein Objekt in der Zelle ist. Wenn man dann den Schnittest macht, schaut man
zuerst welche Zellen der Strahl traversiert. Für jede Zelle schaut man ob dort
Objekte sind und macht den Schnittest mit den Objekten. Nun kann ein Objekt in
mehreren Zellen sein, aber den Schnitttest macht man nur das erste mal. Dannach
speichert man sich, dass der Schnitttest des Strahls mit dem Objekt bereits
gemacht wurde. Dieses Speichern nennt man \textit{mailboxing} (vgl. Kapitel~5, Folie~56).

Die Schnitttests sind in \cref{fig:schnitttests} dargestellt.

\begin{figure}[h]
    \centering
    \includegraphics*[width=0.8\linewidth, keepaspectratio]{schnittests.png}
    \caption{Alle durchgeführten Schnitttests.}
    \label{fig:schnitttests}
\end{figure}

Die Anzahl der Schnittests ist:

\begin{itemize}
    \item Q1 mit Strahl: Schnitt
    \begin{itemize}
        \item Q1.1 mit Strahl: Schnitt, aber hat kein Kind
        \item Q1.2 mit Strahl: kein Schnitt
        \item Q1.3 mit Strahl: Schnitt
        \begin{itemize}
            \item Q1.3.1 mit Strahl: Schnitt. Schnitttest mit I.
            \item Q1.3.2 mit Strahl: kein Schnitt.
            \item Q1.3.3 mit Strahl: Schnitt. Schnitttest mit H.
            \item Q1.3.4 mit Strahl: Schnitt. Schnitttest mit I.
        \end{itemize}
        \item Q1.4 mit Strahl: Schnitt. Schnittest mit G.
    \end{itemize}
    \item Q2 mit Strahl: kein Schnitt
    \item Q3 mit Strahl: Schnitt
    \begin{itemize}
        \item Q3.1 mit Strahl: Schnitt
        \begin{itemize}
            \item Q3.1.1: Schnitt.
            \item Q3.1.2 mit Strahl: Kein Schnitt.
            \item Q3.1.3 mit Strahl: Schnitt. Schnitttest mit C.
            \item Q3.1.4 mit Strahl: Schnitt. Schnitttest mit C.
        \end{itemize}
        \item Q3.2 mit Strahl: kein Schnitt.
        \item Q3.3 mit Strahl: Schnitt.
        \begin{itemize}
            \item Q3.3.1 mit Strahl: Schnitt.
            \item Q3.3.2 mit Strahl: Kein Schnitt
            \item Q3.3.3 mit Strahl: Schnitt
            \begin{itemize}
                \item Q3.3.3.1 mit Strahl: Schnitt. Schnitttest mit A.
                \item Q3.3.3.2 mit Strahl: kein Schnitt
                \item Q3.3.3.3 mit Strahl: Schnitt. Schnitttest mit A.
                \item Q3.3.3.4 mit Strahl: Schnitt. Schnitttest mit A.
            \end{itemize}
            \item Q3.3.4 mit Strahl: Schnitt.
        \end{itemize}
        \item Q3.4 mit Strahl: Schnitt. Schnitttest mit C.
    \end{itemize}
    \item Q4 mit Strahl: Schnitt.
\end{itemize}

\subsection*{Teilaufgabe 8b}
\begin{itemize}
    \item[1] Der Baum einer Hüllkörperhierarchie ist immer balanciert.
    \item[$\Rightarrow$] Falsch.
    \item[2] Der Speicherbedarf für ein reguläres Gitter ist unabhängig von der Anzahl der Primitive.
    \item[$\Rightarrow$] Falsch. Man muss pro Zelle speichern welche Objekte diese Zelle enthält.
    \item[3] Ein kD-Baum hat immer achsenparallele Split-Ebenen.
    \item[$\Rightarrow$] Richtig. Siehe Folie 85.
    \item[4] Ein kD-Baum braucht spezielle Vorkehrungen, um redundante Schnitttests mit demselben Dreieck auszuschliessen.
    \item[$\Rightarrow$] TODO
    \item[5] Ein Verfahren zur Erzeugung eines kD-Baumes erzeugt auch gültige BSP-Bäume.
    \item[$\Rightarrow$] Richtig. kD-Bäume sind spezialfälle von BSP-Bäumen.
    \item[6] Reguläre uniforme Gitter leiden nicht unter dem Teapot-in-a-Stadium Problem.
    \item[$\Rightarrow$] Falsch. Das \enquote{Teapot-in-a-Stadium} Problem bezeichnet
    das Problem, dass Resourcen bei regulären Gittern verschwendet werden weil
    die meisten Zellen leer sind und einige wenige die gesamte Komplexität
    beinhalten. Ein riesiges Objekt (das Stadium) und ein deutlich kleineres,
    aber komplexes Objekt (Teapot) rufen es hervor.
    \item[7] Die Komplexität der Bestimmung eines Schnittpunktes in einem BSP-Baum mit $n$ Primitiven liegt im Optimalfall in $\mathcal{O}(\log n)$.
    \item[$\Rightarrow$] Richtig. Im Optimalfall ist der BSP-Baum balanciert.
    \item[8] Das Traversieren einer Hüllkörperhierarchie kann abgebrochen werden sobald ein Schnittpunkt gefunden wurde.
    \item[$\Rightarrow$] Falsch. Es könnte einen Schnitt geben, der näher zur Kamera ist (TODO: Beispiel in Folien raussuchen.)
\end{itemize}

\clearpage
\section*{Aufgabe 9: Instancing (GLSL)}
\inputminted[linenos, numbersep=5pt, tabsize=4, frame=lines, label=shader.vert]{glsl}{shader.vert}

\clearpage
\section*{Aufgabe 10: Normal Mapping in Objektkoordinaten (GLSL)}
\inputminted[linenos, numbersep=5pt, tabsize=4, frame=lines, label=shader.frag]{glsl}{shader.frag}

\clearpage
\section*{Aufgabe 11: Bézier-Kurven und Bézier-Splines}
\subsection*{Teilaufgabe 11a}
Damit ein Bézier-Spline, welcher aus $C^2$-stetig ist, muss er $C^0$ und $C^1$ stetig sein.
Für $C^0$-Stetigkeit muss nur $\mathbf{b}_3 = \mathbf{c}_0$ gelten.

Für $C^1$-Stetigkeit muss $\mathbf{b}_3 - \mathbf{b}_2 = \mathbf{c}_1 - \mathbf{c}_0$ gelten. Das heißt:

\begin{align}
    \mathbf{b}_3 - \mathbf{b}_2 &\overset{!}{=} \mathbf{c}_1 - \mathbf{c}_0\\
    \begin{pmatrix}7\\8\end{pmatrix} - \begin{pmatrix}5\\9\end{pmatrix}
    &\overset{!}{=}
    \mathbf{c}_1 - \begin{pmatrix}7\\8\end{pmatrix}\\
    \Leftrightarrow \mathbf{c}_1 &\overset{!}{=} \begin{pmatrix}9\\7\end{pmatrix}
\end{align}

Für $C^2$-Stetigkeit muss zusätzlich noch folgende Bedingung gelten:

\begin{align}
    \mathbf{b}_{2} + (\mathbf{b}_{2} - \mathbf{b}_{1}) &\overset{!}{=} \mathbf{c}_{1} + (\mathbf{c}_{1} - \mathbf{c}_{2})\\
    \begin{pmatrix}5\\9\end{pmatrix} + \left (\begin{pmatrix}5\\9\end{pmatrix} - \begin{pmatrix}3\\9\end{pmatrix} \right )
    &\overset{!}{=}
    \begin{pmatrix}9\\7\end{pmatrix} + \left (\begin{pmatrix}9\\7\end{pmatrix} - \mathbf{c}_{2} \right)\\
    \Leftrightarrow \begin{pmatrix}7\\9\end{pmatrix} &\overset{!}{=} \begin{pmatrix}18\\14\end{pmatrix} - \mathbf{c}_{2}\\
    \Leftrightarrow \mathbf{c}_{2} &\overset{!}{=} \begin{pmatrix}11\\5\end{pmatrix}
\end{align}

Das sieht dann wie folgt aus:

\begin{figure}[h]
    \centering
    \includegraphics*[width=0.5\linewidth, keepaspectratio]{11a.png}
    \caption{Ergebnisspline. TODO: Ich glaube $c_2$ ist falsch.}
    \label{fig:situation-11a}
\end{figure}
\clearpage

\subsection*{Teilaufgabe 11b}
Bézierkurven sind immer:

\begin{itemize}
    \item \textbf{Wertebereich}: Bézierkurven liegen innerhalb der konvexen
          Hülle, die durch die 4~Kontrollpunkte gebildet werden.
    \item \textbf{Endpunktinterpolation}: Bézierkurven beginnen immer beim
          ersten Kontrollpunkt und enden beim letzten Kontrollpunkt.
    \item \textbf{Variationsredukion}: Eine Bézierkurve $F$ wackelt nicht stärker
          als ihr Kontrollpolygon $B$ ($\sharp (H \cap F) \leq \sharp (H \cap B)$).
\end{itemize}

Die ersten zwei Eigenschaften gelten für alle 3 dargestellten Kurven, die
dritte ist bei der dritten Kurve jedoch verletzt. Also ist die dritte Kurve
mit den dargestellten Kontrollpunkten keine Bézierkurve.

Begründung: Legt man eine Hyperebene $H$ (Gerade) so, dass sie kurz über dem
zweiten Kontrollpunkt und leicht unter dem zweiten Kontrollpunkt liegt, so
schneidet $H$ das Kontrollpolygon genau 1~mal, aber 3~mal die Bézierkurve.

\end{document}
