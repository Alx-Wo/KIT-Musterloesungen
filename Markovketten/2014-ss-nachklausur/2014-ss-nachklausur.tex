\documentclass[a4paper]{scrartcl}
\usepackage[ngerman]{babel}
\usepackage[utf8]{inputenc}
\usepackage{amssymb,amsmath}
\usepackage{graphicx}
\usepackage[inline]{enumitem}
\setlist{noitemsep}
\usepackage[binary-units=true]{siunitx}
\usepackage{hyperref}
\usepackage{parskip}
\usepackage{braket}         % needed for \Set
\usepackage{csquotes}         % needed for \enquote
\usepackage[nameinlink,noabbrev,ngerman]{cleveref} % has to be after hyperref
\usepackage[colorinlistoftodos]{todonotes}

\usepackage{listings}% http://ctan.org/pkg/listings
\lstset{
  basicstyle=\ttfamily,
  mathescape
}

\setcounter{secnumdepth}{2}
\setcounter{tocdepth}{2}

\begin{document}
 \selectlanguage{ngerman}
 \title{2013 SS}
 \author{Martin Thoma}

 \setcounter{section}{1}
 %%%%%%%%%%%%%%%%%%%%%%%%%%%%%%%%%%%%%%%%%%%%%%%%%%%%%%%%%%%%%%%%%%%%%%%%%%%%%%
 \section*{Aufgabe 6}
 \subsection*{Teilaufgabe 6a}
Es handelt sich um ein M/M/100/$\infty$ Warteschlangenmodell, wenn man davon
ausgeht, dass die Personen vor der Diskothek warten, falls diese zu voll ist.

Wenn man davon ausgeht, dass die Leute nicht warten, handelt es sich um ein
M/M/100/100 Warteschlangenmodell.

\subsection*{Teilaufgabe 6b}
Sei $K=100$ die Kapazität.
\begin{align}
    \lambda_i &= \lambda \text{ für } i=0, 1, \dots, K-1\\
    \mu_i     &= \begin{cases}\mu \cdot i &\text{für } i=0, 1, \dots, K\\
                              0           &\text{für } K+1, K+2, \dots \end{cases}
\end{align}

\subsection*{Teilaufgabe 6c}
Das Erlangsche Verlustmodell mit $K=100$ hat folgende Intensitätsmatrix $Q$:
\[Q = \begin{pmatrix}
- \lambda &        \lambda &               0 & 0       & \dots   & 0\\
      \mu & -(\mu+\lambda) &         \lambda & 0       & \dots   & 0\\
        0 &          2 \mu & -(2\mu+\lambda) & \lambda & \ddots  & \vdots\\
   \vdots &  \ddots        &          \ddots &         & \ddots  & 0\\
   \vdots &                &          \ddots &         &         & \lambda\\
        0 &              0 &           \dots & 0       & 100 \mu & -100\mu
\end{pmatrix} \in \mathbb{R}^{101 \times 101}\]


\subsection*{Teilaufgabe 6d}
Es seien $X_1, \dots, X_n$ exponentialverteilte Zufallsvariablen mit $X_i \sim Exp(\lambda_i)$.
Dann gilt:
\[\min\Set{X_1, \dots, X_n} \sim Exp(\sum_{i=1}^n \lambda_i)\]

Daher gilt für die Zufallsvariable $Y := $ \enquote{Dauer bis der erste von 100 Leuten geht}:
\[Y \sim Exp(100 \cdot \mu)\]

Es folgt:
\[\mathbb{E}(Exp(100 \cdot \frac{1}{50})) = \frac{1}{2}\]
\end{document}
